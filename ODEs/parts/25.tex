\subsection{Dirac Delta Function}

\begin{definition}[Dirac Delta]
    Denote $\delta(t - t_0) \defeq \begin{cases}
        0 & t \neq t_0\\
        \text{ unbounded} & t = t_0
    \end{cases}$,
    in such a way that for any $\epsilon > 0$, $\int_{t_0-\epsilon}^{t_0 + \epsilon} \delta(t - t_0) f(t) \dd{t} = \int_{-\infty}^\infty f(t) \delta(t - t_0) \dd{t} = f(t_0)$. Ie, $\delta(t-t_0)$ "picks out" the function's value at $t_0$.

    In particular, letting $f(t) \equiv 1$, we see that \begin{align*}
        \int_0^t \delta(s - t_0) \dd{s} = \begin{cases}
            0 & t < t_0\\
            1 & t > t_0
        \end{cases}.
    \end{align*}
\end{definition}

\begin{remark}
    This is not a very rigorous definition. Sorry.
\end{remark}

\begin{theorem}
    For $t_0 > 0$, $\laplace{\delta(t - t_0)} = e^{-st_0}$.
\end{theorem}
\begin{proof}
    $\laplace{\delta(t - t_0)} = \int_0^\infty e^{-st} \delta(t - t_0) \dd{t} = e^{-s t_0}$.
\end{proof}

\begin{corollary}
    $\laplace{\delta(t)} = 1$
\end{corollary}

\subsection{Convolutions, Green's Function}

Recall that we can write $L[y] = \sum_{k=0}^n a_k y^{(k)}(t) = f(t)$ (with IVPs) as $P(s) Y(s) - Q(s) = F(s)$, where $P(s) = \sum_{k=0}^n a_k s^k$, $Y(s) = \laplace{y(t)}$, $F(s) = \laplace{f(t)}$, and $Q(s)$ of degree $n - 1$ and dependent on the ICs. Letting $G(s) \defeq \frac{1}{P(s)}$, then, we can rewrite this as \[
y(t) = \ilaplace{F(s)G(s)} + \ilaplace{\frac{Q(s)}{P(s)}}    .
\]
$\deg(Q) < \deg(P)$ so we can find the RHS of this using typical partial fractions techniques, and we can solve the LHS using the convolution theorem, namely $\ilaplace{F(s)G(s)} = (f \ast g)(t)$.

\begin{definition}[Green's function]
    The function $g(t)$ that solves $L[g(t)] = \delta(t)$ with IC $g(0) = g'(0) = \cdots = g^{(n-1)}(0)$ is called the \emph{Green's function} of $L$.
\end{definition}

\begin{theorem}
    Let $g(t)$ be the Green's function of $L$. Then, $L[g(t)] = G(s) = \frac{1}{P(s)}$.
\end{theorem}

\begin{proof}
    $L[g] = \delta(t) \implies P(s)G(s) - Q(s) = 1 \implies P(s)G(s) = 1$.
\end{proof}

\begin{example}
    Find an expression for $y(t)$ with respect to a convolution integral and $Q(s)/P(s)$ for the ODE\[
    y'' + \omega^2 y = f(t), 
    \]
    for arbitrary $y(0) =\alpha_0, y'(0) = \alpha_1$, and then when $\alpha_0 = \alpha_1 = 0$.
\end{example}

\subsection{Transforms of Periodic Functions}

\begin{definition}[Periodic function]
    We say a function $f(t)$ is periodic of period $T$ if $f(t) = f(t + T)$ for some minimal $T > 0$ for all $t > 0$.
\end{definition}
\begin{remark}
    This definition excludes the constant function as a periodic (why?).
\end{remark}

\begin{theorem}
    Let $f$-periodic of period $T$ and pw-cont on $[0, \infty)$. Then, \[
    \laplace{f(t)}     = \frac{1}{1 - e^{-sT}}\int_0^T e^{-st} f(t)\dd{t}.
    \]
\end{theorem}

\begin{proof}
    Straightforward computation (hint: split up the integral in $\laplace{f(t)}$ into two integrals with $T$ as the upper, lower limits resp.).
\end{proof}

\begin{example}
    Find the Laplace transform of $f(t) \defeq \sum_{n=0}^\infty (-1)^n \unit{t - n}$ (remarking that $f$ periodic with $T = 2$) using the previous theorem. Then find it using the linearity of $f$.
\end{example}

\begin{example}[Cursed]
   We consider $y'' + y' + y = f(t) = \delta(t - 1) + \unit{t - 2}e^{-(t-2)}$ with $y(0) = 0$, $y'(0) = 1$. Taking the Laplace of both sides \begin{align*}
    s^2Y(s) - s y(0) - y'(0) +& sY(s) - y(0) + Y(s) = e^{-s} + e^{-2s}\laplace{e^{-t}}\\
    &\implies Y(s)(s^2 + s + 1) - 1 = e^{-s}+e^{-2s}(\frac{1}{s+1})\\
    &\implies Y(s) = \frac{1}{s^2 + s + 1} + e^{-s} \frac{1}{s^2 + s + 1} + e^{-2s} \frac{1}{(s^2+s+1)(s+1)}
   \end{align*}
   Unlike other examples, $s^2 + s + 1$ not reducible (over $\R$) so we have some difficulties. Completing the square, we find $s^2 + s + 1 = (s + \frac{1}{2})^2 + \frac{3}{4}$, and so \[\frac{1}{s^2 + s + 1} = \frac{1}{(s + \frac{1}{2})^2 + \frac{3}{4}}, \qquad \frac{1}{(s^2 + s + 1)(s + 1)} = \frac{1}{((s+\frac{1}{2})^2 + \frac{3}{4})(s + 1)}.\]

   Using partial fractions on the second expression, \begin{align*}
    \frac{1}{(s^2+s+1)(s+1)} = \frac{As + B}{s^2+s+1} + \frac{C}{s+1}&\\
    \implies 1 = (As + B)(s + 1) + C(s^2 + s + 1)&\\
    s = -1]& \quad 1 = C\\
    s^2]& \quad 0 = A + C \implies A = -1\\
    s^0]& \quad 1 = B + C \implies B = 0\\
   \end{align*}
   Bring all the "simplifications" together, we have \begin{align*}
    Y(s) = \frac{1}{(s + \frac{1}{2})^2 + \frac{3}{4}} + e^{-s}[\frac{1}{(s+\frac{1}{2})^2 + \frac{3}{4}}] + e^{-2s}[\frac{-s}{(s + \frac{1}{2})^2 + \frac{3}{4}} + \frac{1}{s+1}]
   \end{align*}
   For the first term, we need to use the first translation theorem, and for the other two we need to use both the first and second theorems.
   \begin{align*}
    \frac{1}{(s+1/2)^2 + 3/4} = \frac{2}{\sqrt{3}}(\frac{\sqrt{3}/2}{(s + 1/2)^2 + 3/4}) &\overset{\ilaplace{\dots}}{\rightsquigarrow} \frac{2}{\sqrt{3}}e^{-1/2 t} \sin(\frac{\sqrt{3}}{2}t)\\
    e^{-s}[\frac{1}{(s+1/2)^2 + 3/4}]  &\overset{\ilaplace{\dots}}{\rightsquigarrow} \unit{t - 1}\ilaplace{\frac{2}{\sqrt{3}}(\frac{\sqrt{3}/2}{(s + 1/2)^2 + 3/4})}_{t \mapsto t -1} \\
    &\overset{\ilaplace{\dots}}{\rightsquigarrow} \unit{t - 1}\frac{2}{\sqrt{3}}e^{-1/2 (t-1)} \sin(\frac{\sqrt{3}}{2}(t-1))\\
   \end{align*}
\end{example}