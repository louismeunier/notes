\documentclass[12pt, oneside]{article}
\usepackage[margin=1in, includefoot]{geometry}
\usepackage[pagestyles]{titlesec}
\usepackage[]{csquotes}

\usepackage{amsthm}
\usepackage{amsmath,amssymb}
\usepackage{mathrsfs}
\usepackage{xfrac}

% \usepackage{libertine}
\usepackage{newpxtext}
\usepackage{newpxmath}

\usepackage{tabularx}
\usepackage{multicol}
\usepackage[shortlabels]{enumitem}
\usepackage{listings}
\usepackage{tocloft}

\usepackage{setspace}
\usepackage{cancel}
\usepackage{graphicx}

\usepackage{thmtools}
\usepackage{thm-restate}
\usepackage[framemethod=TikZ]{mdframed}

\usepackage[ragged,multiple,flushmargin]{footmisc}

\usepackage{xcolor-solarized}
\usepackage[hidelinks,colorlinks=true,allcolors=solarized-blue]{hyperref}
\usepackage{cleveref}
\usepackage{nameref}

\usepackage{physics}

\usepackage{tikz}

\usepackage{contour}
\usepackage[normalem, normalbf]{ulem}

\usepackage{import}
\import{/users/louismeunier/Documents/templates/}{shortcuts.sty}


\newcommand*{\important}{\textcolor{red}{$\star$} }
% https://alexwlchan.net/2017/latex-underlines/
\renewcommand{\ULdepth}{1.8pt}
\contourlength{0.8pt}

\newcommand{\myuline}[1]{%
  \uline{\phantom{#1}}%
  \llap{\contour{white}{#1}}%
}
\newcommand{\pageauthor}{Louis Meunier}
\newcommand{\pagetitle}{Honours Analysis 2}
\newcommand{\pagesubtitle}{MATH255}
\newcommand{\pagedescription}{Summary of Results}
\newcommand{\pagesemester}{Winter, 2024 }
\newcommand{\pageprofessor}{Prof. Dmitry Jakobson}

\newcommand{\thetitle}{
  \noindent
% \begin{center}
  \vspace*{5em}\\
  {\Large\textbf{\pagesubtitle} - \pagetitle}\\
  {\small{\pagedescription}}
  \vspace*{2em}\\
  {\small \pagesemester\\
  Notes by \pageauthor}\\
  {\small \href{https://notes.louismeunier.net/Analysis 2/analysis2.pdf}{Complete notes}}
  % \vspace*{2em}
% \end{center}
}
\renewcommand{\contentsname}{}

\titleformat{\section}
{\centering\bfseries\scshape}
{\thesection}{1em}{}

\theoremstyle{definition}
\newtheorem{defn}{Definition}

\theoremstyle{plain}
\newtheorem{thm}{Theorem}
\newtheorem{cor}{Corollary}
\newtheorem{prop}{Proposition}
\newtheorem{lemma}{Lemma}
\theoremstyle{remark}
\newtheorem{remark}{Remark}

\DeclareMathOperator{\proj}{proj}
\DeclareMathOperator{\orth}{orth}

\let\origsection=\section
\let\origsubsection=\subsection

\renewcommand\section[1]{\origsection{#1}\label{sec:\thesection}}
\renewcommand\subsection[1]{\origsubsection{#1}\label{\thesubsection}}

\begin{document}
\thetitle
% \begin{center}
\tableofcontents
\setstretch{1.5}
% \end{center}

\section{Point-Set Topology}

\textbf{Topology is about abstracting openness. It can typically suffice to consider open, closed sets in $\R$ for intuition, but is obviously not all-general.}

\begin{defn}[Metric Space]
  A space $X$ equipped with a function $d : X \times X \to [0, \infty)$ is called a metric space and $d$ a metric or distance if \begin{itemize}
    \item $d(x, y) = d(y, x) \geq 0$
    \item $d(x, y) = 0 \iff x = y$
    \item $d(x, y) + d(y, z) \geq d(x, z)$
  \end{itemize}
  for any $x, y, z \in X$.
\end{defn}

\begin{defn}[Normed Vector Space]
  A function $\norm{\cdot} : X \to \R$ defined on a vector space $X$ over $\R$ is a norm if \begin{itemize}
    \item $\norm{x} \geq 0$
    \item $\norm{x} = 0 \iff x = 0$
    \item $\norm{c \cdot x} = \abs{c} \norm{x}$
    \item $\norm{x + y} \leq \norm{x} + \norm{y}$,
  \end{itemize}
  for any $x, y \in X, c \in \R$.
\end{defn}

\begin{remark}
  We can naturally extend this to arbitary fields, but seeing as this is a course in Real Analysis, we won't.
\end{remark}

\begin{prop}
  For a normed vector space $(X, \norm{\cdot})$, $d(x, y) \defeq \norm{x - y}$ is a metric on $X$. We call such a metric the one "induced" by the norm.
\end{prop}

\begin{defn}[Topological Set]
  A set $X$ is a topological set if we have a collection $\tau$ of subsets of $X$, called open sets, such that \begin{itemize}
    \item $\varnothing \in \tau, X \in \tau$
    \item For $A_\alpha \in \tau$ for $\alpha$ in any $I$ (potentially infinite), $\bigcup_{\alpha \in I} A_\alpha \in \tau$
    \item For $A_\alpha \in \tau$ for $\alpha \in J$ where \emph{J finite}, then $\bigcap_{\alpha \in J} A_\alpha \in \tau$
  \end{itemize}
  ie, arbitrary unions of open sets are open, and finite intersections of open sets are open.
\end{defn}

\begin{remark}
  Keep $\R$ in mind when initially working with these definitions; for instance, the set $A_n \defeq (0, \frac{1}{n})$ open in $\R$ for any $n \in \mathbb{N}$, but $\bigcap_{n \in \mathbb{N}} A_n = \{0\}$ which is closed.
\end{remark}

\begin{remark}
  Complemented each of these requirements gives similar definitions for closed sets of $X$.
\end{remark}

\begin{defn}[Topology on a Metric Space]
  A subset $A \subseteq X$ open iff $\forall x \in A, \exists r = r(x) \in \R,$ where $r(x) > 0$, such that $B(x, r(x)) \defeq \{y \in x : d(x, y) < r(x)\} \subseteq A$. We call such a $B$ an open ball, and $\overline{B}$ a closed ball with the same definition replacing the strict inequality with $\leq$.
\end{defn}

\begin{remark}
  While many of the spaces we look at our metric spaces that induce a topology as such, \textbf{not all topological spaces are metric spaces}. Indeed, "metrizability" (ie, equipping a topological space $X$ with a metric that respects the open sets) is not a trivial activity.
\end{remark}


\begin{defn}[Equivalence of Metrics]
  We say two metrics on $X$ are equivalent if they admit the same topology; a sufficient condition is that, $\forall x \neq y \in X$, $\exists 1 < C < \infty$ such that $\frac{1}{C} < \frac{d_1(x,y)}{d_2(x,y)}< C$, then $d_1, d_2$ equivalent, where $C$ independent of $x,y$.
\end{defn}


\begin{figure*}[!ht]
  \centering
  \includegraphics*[width=0.7\textwidth]{topoanatomy.png}
\end{figure*}

\begin{defn}[\important Interior, Boundary, Closure]
  Let $X$-topological space, $A \subseteq X$, $x \in X$. \begin{itemize}
    \item If $\exists U$-open \st $x \in U \subseteq A$, then we write $x \in \interior(A)$, the interior of $A$.
    \item If $\exists V$-open \st $x \in V \subseteq A^C$, then $x \in \interior(A^C)$.
    \item If $\forall U$-open \st $x \in U, U \cap A \neq \varnothing$ and $U \cap A^C \neq \varnothing$, then $x \in \partial A$, the boundary of $A$.
  \end{itemize}
  We put $\overline{A} \defeq \interior(A) \cup \partial A$, the closure of $A$. Equivalently, $x \in \overline{A} \iff$ for every open set $U : x \in U$, $U \cap A \neq \varnothing$. We call $x \in \overline{A}$ the limit points of $A$.
\end{defn}

\begin{remark}
  The limit point interpretation of the closure can be more intuitive; the points that we can get "arbitrary close to" are the closure. For instance, $\overline{(a, b)} = [a, b] \subseteq \R$ with the standard topology. 
\end{remark}

\begin{prop}
  Let $A \subseteq X$-topological space. Then, $\interior(A)$ is open, the largest open set contained in $A$, the union of all open sets contained in $A$, and $\interior(\interior(A)) = \interior(A)$. Also, $\overline{A}$ closed, the smallest closed set that contains $A$, $\overline{A}$ the intersection of all closed sets that $A$ is contained in, and $\overline{\overline{A}} = \overline{A}$.
\end{prop}

\begin{cor}
  $A$ open $\iff A = \interior(A)$ and $A$ closed $\iff A = \overline{A}$
\end{cor}

\begin{remark}
  Remark that these are not exclusive, nor indeed the only possibilities.
\end{remark}

\begin{defn}[Basis]
  A basis for a topology $X$ with open sets $\tau$ is a collection $B \subseteq \tau$ such that every $U\in \tau$ a union of sets in $B$.
\end{defn}

\begin{remark}
  Don't think about bases for vector spaces in this regard - there is no "minimality" requirement.

  Keep in mind $\{(a, b) : -\infty < a < b < \infty\}$, a basis of topology on $\R$.
\end{remark}

\begin{prop}
  For a metric space $(X, d)$, $\{B(x, r) : x \in X, r > 0\}$ a basis of topology.
\end{prop}

\begin{defn}[Subspace Topology]
  For a subset $Y \subseteq X$-topological space, we define the subspace topology on $Y$ as $\tau_Y \defeq \{Y \cap U : U \in \tau\}$.
\end{defn}

\begin{defn}[\important Continuous]
  For $X, Y$-topological spaces, a function $f : X \to Y$ is continuous iff $\forall V$-open in $Y$, $f^{-1}(V)$-open in $X$.
\end{defn}

\begin{remark}
  One can verify that this is consistent with the $\epsilon-\delta$ definition of continuity for functions on $\R$.
\end{remark}

\begin{thm}[Continuity of Composition]
  If $f : X \to Y$, $g : Y \to Z$ continuous, $g \circ f$ continuous.
\end{thm}
\begin{remark}
  Note how much easier this is to prove via toplogical spaces than the $\epsilon-\delta$ definition.
\end{remark}

% \begin{prop}
%   For $f : X \to Y$ continuous and $A \subseteq X$ equipped with the subspace topology, $f_A$ also continuous.
% \end{prop}

\begin{defn}[Product Space]
  For an index set $I$ and $X_\alpha, \alpha \in I$, we define $\prod_{\alpha \in I} X_\alpha$ as a product space; $I$ may be finite or infinite.
\end{defn}

\begin{prop}
  A basis for the product space is given by cyliders of the form $A = \prod_{\alpha \in J} A_\alpha \times \prod{\alpha \notin J} X_\alpha$ for $A_\alpha$-open in $X_\alpha$, where $J \subseteq I$-finite.
\end{prop}

\begin{defn}[Compact]
  A set $A \subseteq X$ is compact if every cover has a finite subcover, that is \[
  A \subseteq \bigcup_{\alpha \in I} U_\alpha \text{-open} \implies \exists \{\alpha_1, \dots, \alpha_n\}   \subseteq I \st A \subseteq \bigcup_{i=1}^n U_{\alpha_i}.
  \]
\end{defn}

\begin{prop}
  Closed intervals $[a, b]$ compact in $\R$.
\end{prop}

\begin{prop}
  $A \subseteq \R^n$ compact $\iff $ closed and bounded.
\end{prop}

\begin{defn}[Connected]
  $X$ is said to not be connected if $X = U \cup V$ for $U, V$ open, nonempty, disjoint. If $X$ cannot be written as such, $X$ is said to be connected.
\end{defn}

\begin{thm}
  If $X$ connected and $f : X \to Y$, then $f(X)$ connected in $Y$.
\end{thm}

\begin{prop}
  Intervals in $\R$ are connected.
\end{prop}

\begin{thm}[Intermediate Value Theorem]
  If $X$ connected, $f : X \to \R$ continuous, then $f$ takes intermediate value; if $a = f(x), b = f(y)$ for $x, y \in X$ with $a < b$, then for any $a < c < b$ $\exists z \in X \st f(z) = c$.
\end{thm}

\begin{thm}
  For $X$ compact, $f : X \to Y$ continuous, $f(X)$ compact in $Y$.
\end{thm}

\begin{prop}
  For $X$ compact and $f: X \to \R$, $f$ attains both max and min on $X$.
\end{prop}

\begin{defn}[Path Connected]
  A set $A \subseteq X$ is path connected if for any $x, y \in A, \exists f : [a, b] \to X$ continuous such that $f(a) = x, f(b) = y f([a, b]) \subseteq A$.
\end{defn}

\begin{thm}
  Path connected $\implies$ connected.

  For open sets in $\R^n$, the converse holds too.
\end{thm}

\begin{defn}[Connected Component, Path Component]
  For $x \in X$, the connected component of $x$ is the largest connected subset of $X$ containing $x$ and the path component of $x$ is the largest path connected subset of $X$ containing $x$.
\end{defn}

\section{Metric Spaces}

\textbf{We discuss mostly the metric on $\ell_p$ space and notions of completeness, as well as some topological results specific to metric spaces, namely compactness.}

\begin{defn}[$\ell_p$]
  For $x = (x_1, \dots, x_n) \in \R^n$ and $1 \leq p \leq + \infty$, we define \[
  \norm{x}_p \defeq \left(\sum_{i=1}^n \abs{x_i}^p\right)^{\frac{1}{p}}, \quad \norm{x}_\infty \defeq \max_{i=1}^n \abs{x_i},
  \]
  and similarly, for sequences $x = (x_1, \dots, x_n, \dots)$, \[
  \norm{x}_p \defeq   \left(\sum_{i=1}^\infty \abs{x_i}^p\right)^{\frac{1}{p}}, \quad \norm{x}_\infty \defeq \sup_{i=1}^\infty \abs{x_i},
  \]
  and define $\ell_p \defeq \{x : \norm{x}_p < +\infty\}$. It can be shown that these are well-defined norms on their respective spaces.
\end{defn}

\begin{thm}[Holder, Minkowski's Inequalities]
  For $x = (x_1, \dots, x_n), y = (y_1, \dots, y_n)$ and $p, q$ such that $\frac{1}{p} + \frac{1}{q} = 1$, then \[
  \text{Holder's:} \qquad \iprod{x, y} = \abs{\sum_{i=1}^n x_i y_i} \leq \norm{x}_p \norm{y}_q  
  \]
  and \[
  \text{Minkowski's:} \qquad \norm{x + y}_p \leq \norm{x}_p + \norm{y}_p .
  \]
  The identical inequalities hold for infinite sequences.
\end{thm}

\begin{defn}[Completeness]
  We say a metric space is complete if every Cauchy sequence converges to a limit point in the space.
\end{defn}
\begin{prop}
  For $\{x_n\}_{n \in \mathbb{N}}$, $\ell_p$ complete for any $1 \leq p \leq +\infty$.
\end{prop}

\begin{prop}
  If $p < q$, $\ell_p \subseteq \ell_q$.
\end{prop}

\begin{defn}[Contraction Mapping]
  For a metric space $(X, d)$, a function $f : X \to X$ is a contraction mapping if there exists $0 < c < 1$ such that \[
  d(f(x), f(y)) \leq c \cdot d(x,y)  
  \]
  for any $x, y \in X$.
\end{defn}

\begin{thm}
  Let $(X, d)$ be a complete metric space, $f : X \to X$ a contraction. Then, there exist a unique fixed point $z$ of $f$ such that $f(z) = z$; ie $\lim_{n \to \infty} f^{n}(x) = \lim_{n \to \infty} f\circ f \circ \cdots \circ f(x) = z$ for any $x \in X$.
\end{thm}

\begin{thm}
  $\ell_p$ complete.
\end{thm}
\begin{remark}
  It can be kind of funky to work with sequences in $\ell_p$, since the elements of $\ell_p$ themselves sequences so we have "sequences of sequences".  
\end{remark}

\begin{defn}[Totally bounded]
  A metric space $X$ is said to be totally bounded if $\forall \epsilon > 0 \exists x_1, \dots, x_n \in X, n = n(\epsilon)$ such that $\bigcup_{i=1}^n B(x_i, \epsilon) = X$.
\end{defn}

\begin{defn}[Sequentially compact]
  A metric space $X$ is said to be sequentially compact if every sequence has a convergent subsequence.
\end{defn}

\begin{thm}[\textcolor{red}{$\star$} Equivalent Notions of Compactness in Metric Spaces]
  Let $(X, d)$ a metric space. TFAE: \begin{itemize}
    \item $X$ compact
    \item $X$ complete and totally bounded
    \item $X$ sequentially compact
  \end{itemize}
\end{thm}

\begin{remark}
  This is for a metric space, not a general topological space! Hopefully this is clear because some of the requirements necessitate a distance.
\end{remark}


\section{Differentiation}
\begin{defn}[Differentiable]
  $f(x)$ differentiable at $c$ if $\lim_{x \to c} \frac{f(x) -f(c)}{x - c}$ exists, and if so we denote the limit $f'(c)$.

  Alternatively, one can view differentiation as a linear map between spaces of differentiable functions.
\end{defn}

\begin{thm}
  Differentiable $\implies$ continuous.
\end{thm}
\begin{proof}
  Short enough to write the full proof; $\lim_{x \to c} (f(x) - f(c)) = \lim_{x\to c} (x - c) \frac{f(x) - f(c)}{x-c} = 0 \cdot f'(c) = 0$.
\end{proof}

\begin{thm}[Caratheodory's]
  For $f : I \to \R, c \in I$, $f$ differentiable at $c$ iff $\exists \varphi : I \to \R : \varphi$ continuous at $c$, $f(x) - f(c) = \varphi(x) (x-c)$.
\end{thm}

\begin{proof}[Sketch]
  Its worth recalling the definition of $\varphi$ for the forward implication, $$\varphi(x) \defeq \begin{cases}
    \frac{f(x) - f(c)}{x - c} & x \neq c\\
    f'(c) & x = c
  \end{cases}.$$ The converse follows by taking limits.
\end{proof}

\begin{remark}
  While not a particularly enlightening result, used in proofs of the chain rule, etc.
\end{remark}

\begin{thm}[Chain Rule]
  Let $f : J \to \R, g : I \to R \st f(J) \subseteq I$. If $f(x)$ differentiable at $c$ and $g(y)$ at $f(c)$, $g \circ f$ differentiable at $c$ with $(g \circ f)'(c) = g'(f(c))\cdot f'(c)$.
\end{thm}
\begin{proof}[Sketch]
  Apply Caratheodory's to $f$ at $c$ and $g$ at $f(c)$, and compose.
\end{proof}

\begin{thm}[Rolle's]
  Let $f : [a, b] \to \R$ continuous. If $f'(x)$ exists on $(a, b)$ and $f(a) = f(b) = 0$, $\exists c \in (a, b) : f'(c) = 0$.
\end{thm}
  
\begin{proof}[Sketch]
  If constant function, done. Else, assuming function positive, it obtains a maximum, and thus its derivative 0 at this point.
\end{proof}

\begin{thm}[\textcolor{red}{$\star$} Mean Value]
  Let $f$ continuous on $[a, b]$ and differentiable on $(a, b)$. Then, $\exists c \in (a, b)$ such that $f(b) - f(a) = f'(c) (b - a)$.
\end{thm}
\begin{proof}[Sketch]
  Let $\phi(x) \defeq f(x) - f(a) - \frac{f(b) - f(a)}{(b-a)}(x-a)$. Then $\phi(a) = \phi(b) = 0$ so applying Rolle's $\exists c \in (a, b) : \varphi'(c) = 0 = f'(x) - \frac{f(b) - f(a)}{b-a}$. The proof is done after rearranging.
\end{proof}

\begin{prop}[L'Hopital's]
  If $f, g : [a, b] \to \R$ with $f(a) = g(a) = 0$, $g(x)\neq 0$ on $a < x < b$, $f, g$ differentiable at $x = 0$ with $g'(a) \neq 0$, then $\lim_{x \to a^+} \frac{f(x)}{g(x)}$ exists and is equal to $\frac{f'(a)}{g'(a)}$.
\end{prop}
\begin{remark}
  Other versions exist, but this is certainly one of them.
\end{remark}

\begin{thm}[\textcolor{red}{$\star$} Taylor's]
  Let $f \in C^{n}([a, b])$ such that $f^{(n+1)}(x)$ exists on $(a, b)$. Let $x_0 \in [a, b]$, then, for any $x \in [a, b]$, $\exists c $ between $x, x_0$ such that \[
  f(x) = f(x_0) + f'(x_0)(x - x_0) + \frac{f''(x_0)}{2!}(x-x_0)^2 + \cdots + \frac{f^{(n)}(x_0)}{n!}(x-x_0)^n + \frac{f^{(n+1)}(c)}{(n+1)!}(x-x_0)^{n+1}.
  \]
\end{thm}

\begin{cor}
  Let $x_0 \in [a, b]$. With the same assumptions as Taylor's (but in a neighborhood of $x_0$), with $f'(x_0) = f''(x_0) = \cdots = f^{(n-1)}(x_0) = 0$ and $f^{(n)}(x_0) \neq 0$, then \begin{itemize}
    \item $n$ even; then $f$ has a local minimum at $x_0$ if $f^{(n)}(x_0) > 0$ and a local max if $f^{(n)}(x_0) < 0$.
    \item $n$ odd; neither. 
  \end{itemize}
\end{cor}

\section{Integration}

\textbf{Its all just rectangles.}
\newcommand{\PP}{\mathcal{P}}
\newcommand{\RR}{\mathcal{R}}

\begin{defn}[Riemann Integration]
  Consider an interval $(a, b)$. We call a subdivision $\PP \defeq \{a = x_0, x_1, \dots, x_{n-1}, x_n = b\}$ a partition, and $\dot{\PP}$ a marked partition if in addition we are given a point $t_i \in (x_i, x_{i+1}]$ for each interval in $\dot{\PP}$.

  We put $\diam(\PP) \defeq \max_{i=1}^n \abs{x_i - x_{i-1}}$.

  We define the Riemann sum $S(f, \dot{\PP}) \defeq \sum_{i=1}^n f(t_i) (x_i - x_{i-1})$, and say that $f$ Riemann integrable on $[a, b]$ if $S(f, \dot{\PP}) \to L$ as $\diam(\dot{\PP}) \to 0$ for any choice of tag $t_i$, and write $f \in \RR([a, b])$

  More precisely, if $\forall \epsilon > 0$, $\exists \delta > 0 : \diam(\PP) < \delta$, then for any $t_i \in [x_i, x_{i+1}]$, $\abs{L - S(f, \dot{\PP})} < \epsilon$. We then say the (Riemann) integral of $f$ over $[a, b]$ is $L$ and write $\int_a^b f(x) \dd{x} = L$.
\end{defn}

\begin{prop}
  Riemann integrals are unique, linear in $f(x)$, and respect inequalities (if $f \leq g$ on $[a, b]$, $\int_a^b f(x) \dd{x} \leq \int_a^b g(x) \dd{x}$ if both in $\RR([a, b])$)
\end{prop}

\begin{prop}[\textcolor{red}{$\star$}]
  $f \in \RR[a, b] \implies f$ bounded on $[a, b]$
\end{prop}

\begin{prop}[\important Cauchy Criterion for Integrability]
  $f \in \mathcal{R}[a, b] \iff \forall \epsilon > 0, \exists \delta > 0 : $ if $\dot{P}$ and $\dot{Q}$ are tagged partitions of $[a, b] \st \diam{\dot{P}} < \delta$ and $\diam{\dot{Q}} < \delta$, then $\abs{S(f, \dot{P}) - S(f, \dot{Q})} < \epsilon$
\end{prop}

\begin{remark}
  Ala Cauchy Sequence.
\end{remark}

\begin{thm}[Squeeze Theorem]
  $f \in \RR[a, b] \iff \forall \epsilon > 0, \exists \alpha_\epsilon, \omega_\epsilon \in \RR[a,b] : \alpha_\epsilon \leq f \leq \omega_\epsilon$ and $\int_a^b (\omega_\epsilon - \alpha_\epsilon) < \epsilon$.
\end{thm}

\begin{lemma}
  Let $J \defeq [c, d] \subseteq [a, b]$ and $\varphi_J(x) \defeq \begin{cases}
    1 & x \in J\\
    0 & x \notin J
  \end{cases}$ be the indicator function of $J$. Then $\varphi_J \in \RR[a, b]$ and $\int_a^b \varphi_J = d- c$.
\end{lemma}

\begin{remark}
  Helpful for "approximations"; follows by linearity, induction that step functions (ie sums of indicator functions times constants) are integrable.
\end{remark}

\begin{thm}[\textcolor{red}{$\star$} Continuous]
  $f$ continuous on $[a, b] \implies f \in \RR[a, b]$
\end{thm}

\begin{proof}[Sketch]
  Continuity on a closed interval gives uniform continuity and so a "universal $\delta$"; then, for any partition, take the $x$ such that $f$ attains its minimum and maximum, and define a $\alpha_\epsilon$, $\omega_\epsilon$ as the sum of indicator functions taking the minimum, maximum of $f$ respectively on each partition. Then apply the previous theorem and the squeeze theorem.
\end{proof}

\begin{thm}[Additivity]
  $f \in \RR[a, b] \iff f \in \RR[a, c]$ and $f \in \RR[c, b]$, and $\int_a^b f(x) \dd{x} = \int_a^c f(x) \dd{x} + \int_c^b f(x) \dd{x}$.
\end{thm}

\begin{thm}[\textcolor{red}{$\star$} Fundamental Theorem of Calculus]
  Let $F, f : [a, b] \to \R$ and $E \subseteq [a, b]$ a finite set, such that $F$ continuous on $[a, b], F'(x) = f(x) \forall x \in [a, b] \setminus E$, $f \in \RR[a, b]$. Then $\int_a^b f(x) = F(b) - F(a)$. We call $F$ the "primitive" of $f$.
\end{thm}

\begin{thm}
  For $f \in \RR[a, b]$ and any $z \in [a, b]$, put $F(z) \defeq \int_a^z f(x) \dd{X}$. Then, $F$ continuous on $[a, b]$.
\end{thm}

\begin{thm}[\textcolor{red}{$\star$} Fundamental Theorem of Calculus p2]
  For $f \in \RR[a, b]$ continuous at $c$, then $F(z)$ differentiable at $c$ and $F'(c) = f(c)$.
\end{thm}

% \begin{thm}[Change of Variables]
%   For $[]$
% \end{thm}

\begin{defn}[Lebesgue Measure]
  We say a set $A \subseteq \R$ has Lebesgue measure $0$ iff $\forall \epsilon > 0$, $A$ can be covered by a union of intervals $J_k$ such that $\sum_k \abs{J_k} \leq \epsilon$. We then call $A$ a "null set".

  In particular, any countable set is a null set.
\end{defn}

\begin{thm}[\textcolor{red}{$\star$} Lebesgue Integrability Criterion]
  Let $f : [a, b] \to \R$ be bounded. Then $f \in \RR[a,b] \iff$ the set of discontinuities of $f$ has Lebesgue measure 0.
\end{thm}

\begin{remark}
  In particular, remark that continuity a stronger requirement than integrability.
\end{remark}

\begin{thm}[Composition]
  If $f \in \RR[a, b]$, $\varphi : [c, d] \to \R$ continuous and $f([a, b]) \subseteq [c, d]$, then $\varphi \circ f \in \RR[a, b]$.
\end{thm}

\begin{thm}[Integration by Parts]
  If $F, G$ differentiable $[a, b]$ with $f \defeq F', g \defeq G'$, and $f, g \in \RR[a, b]$, then \[
  \int_a^b f(x) G(x) \dd{x} = F(x) G(x)\eval_{a}^b - \int_a^b F(x) g(x) \dd{x}.
  \]
\end{thm}

\begin{proof}[Sketch]
  Uses additivity and the fundamental theorem of calculus.
\end{proof}

\begin{thm}[Taylor's Theorem, Remainder's Version]
  Suppose $f', f'', \dots, f^{(n)}$ exist on $[a, b]$ and $f^{(n+1)} \in \RR[a, b]$. Then \[
  f(b) = f(a) + \frac{f'(a)}{1!}(b-a) + \frac{f''(a)}{2!}(b-a)^2 + \cdots + \frac{f^{(n)}(a)}{n!}(b-a)^n + R_n,
  \]
  where $R_n \defeq \frac{1}{n!} \int_a^b f^{(n+1)}(t) (b - t)^{n} \dd{t}$.
\end{thm}

\section{Sequences of Functions}

\textbf{A good motivation to keep in mind with the "types" of function-sequence convergence is to answer the question: when can we exchange limits of derivatives of functions and derivatives of limits of functions? What about integrals? What about summations (see next section)? Ie, when does $\lim_{n \to \infty} f_n'(x) = \dv{x}\lim_{n \to \infty} f_n(x)$, etc.}

\begin{defn}[Pointwise, Uniform Convergence]
We say $f_n \to f$ pointwise on $E$ if $\forall x \in E$, $f_n(x) \to f(x)$ as $n \to \infty$.

We say $f_n \to f$ uniformly on $E$ if $\forall \epsilon > 0, \exists N \in \mathbb{N}$ such that $\forall n \geq N, x \in E$, $\abs{f_n(x) - f(x)} < \epsilon$.
\end{defn}

\begin{remark}
Pointwise doesn't care about the "rate of convergence"; as long as each point converges eventually, we're good. Uniform convergence needs all points to converge "at the same rate" (so to speak).

A good example to keep in mind is $f_n \defeq \begin{cases}
  2nx & 0 \leq x \leq \frac{1}{2n}\\
  0 & x > \frac{1}{2n}
\end{cases}$ on $[0, 1]$, which converges pointwise to $0$ but not uniformly.

A good trick for disproving uniform convergence of $f_n \to f$ is by showing $f_n(x_0)$ constant and $\neq f(x_0)$ for all $n$. For instance, $f_n(x) \defeq \sin(\frac{x}{n}) \to 0$ pointwise, but $f_n(\frac{n\pi}{2}) = 1 \forall n$ so the convergence os not uniform.
\end{remark}
\begin{prop}
  Uniform $\implies$ pointwise convergence.
\end{prop}

\begin{thm}
  The metric space of continuous functions $C([a, b])$ complete with respect to $d_\infty(f, g) \defeq \sup_{x \in [a, b]}\abs{f(x) - g(x)}$.
\end{thm}

\begin{thm}[\textcolor{red}{$\star$} Interchange of Limits]
  Let $J \subseteq \R$ be a bounded interval such that $\exists x_0 \in J : f_n(x_0) \to f(x_0)$. Suppose $f_n'(x) \to g(x)$ uniformly on $J$. Then, $\exists f : f_n (x) \to f(x)$ uniformly on $J$, $f(x)$ differentiable on $J$, and moreover $f_n'(x) = g(x) \forall x \in J$.
\end{thm}

\begin{thm}[\textcolor{red}{$\star$} Interchange of Integrals]
  Let  $f_n \in \RR[a, b], f_n \to f$ uniformly on $[a, b]$. Then $f \in \RR[a, b]$ and $\int_a^b f_n(x) \dd{x} \to \int_a^b f(x) \dd{x}$
\end{thm}
\begin{thm}[Bounded Convergence]
  Let $f_n \in \RR[a, b], f_n \to f \in \RR[a, b]$ (not necessarily uniform). Suppose $\exists B > 0 \st \abs{f_n(x)} \leq B \forall x \in [a, b]$ and $\forall n \in \mathbb{N}$, then $\int_a^b f_n \to \int_a^b f$ as $n \to \infty$.
\end{thm}
\begin{remark}
  This provides a weaker condition, but equivalent result as the previous theorem, although remark now that we need the limit function itself to be in $\RR[a, b]$, which was a result, not a necessity, of the previous theorem. In general, uniform continuity very strong, but leads to helpful results.
\end{remark}

\begin{thm}[Dimi's]
  If $f_n \in C([a, b])$, $f_n(x)$ monotone (as a sequence), and $f_n \to f \in C([a, b])$, then $f_n \to f$ uniformly.
\end{thm}

\section{Infinite Series}

\begin{defn}[Covergence of Series]
  Let $\{x_j\} \in X$-normed vector space over $\R$. We say $\sum_{j=1}^\infty x_j$ converges absolutely iff $\sum_{j=1}^\infty \norm{x_j} < + \infty$. In particular, if $X = \R$, then $\norm{\cdot} = \abs{\cdot }$.

  We say $\sum_{j=1}^\infty x_j$ converges conditionally if $\sum_{j=1}^\infty x_j < +\infty$, but $\sum_{j=1}^\infty \norm{x_j} = +\infty$.
\end{defn}

\begin{prop}
  Any rearrangement of an absolutely convergent series gives the same sum. Conversely, the order of summation of a conditionally convergent summation can be rearranged such as to equal any real number.
\end{prop}


\begin{prop}[Absolute Convergence Tests]
  \begin{itemize}
    \item \textbf{Comparison Test:} let $x_n, y_n$ be nonzero real sequences and $r \defeq \lim \abs{\frac{x_n}{y_n}}$. If such a limit exists, then if
    \begin{enumerate}[label=(\alph*)]
      \item $r \neq 0$, $\sum_n x_n$ absolutely convergent $\iff$ $\sum_n y_n$ absolutely convergent.
      \item $r = 0$, $\sum_n {y_n}$ absoltuely convergent $\implies \sum_n x_n$ absolteuly convergent.
    \end{enumerate}
    \item \textbf{Root Test:} if $\exists r < 1 \st \abs{x_n}^{1/n} \leq r \forall n \geq K$-sufficiently large, then $\sum_{n=K}^\infty \abs{x_n}$ converges. Conversely, if$\abs{x_n}^{1/n} \geq 1$ for $n \geq K$-sufficiently large, $\sum_n x_n$ diverges.
    \item \textbf{Ratio Test:} if $x_n \neq 0$ and $\exists 0 < r < 1  \st \abs{\frac{x_{n+1}}{x_n}} \leq r$ for $n \geq K$ sufficiently large, $\sum_n x_n$ absolutely convergent. Conversely, if $\abs{\frac{x_{n+1}}{x_n}} \geq 1$ for $n \geq K$ sufficiently large, then $\sum_n x_n$ diverges.
    \item \textbf{Integral Test:} if $f(x) \geq 0$ non-increasing/non-decreasing function of $x \geq 1$, $\sum_{k=1}^\infty f(k)$ converges iff $\lim_{k \to \infty} \int_1^k f(x) \dd{x}$ finite.
    \item[$\ast$] \textbf{Raube's Test:} let $x_n \neq 0$. \begin{enumerate}[label=(\alph*)]
      \item If $\exists a > 1 \st \abs{\frac{x_{n+1}}{x_n}} \leq 1 - \frac{1}{n} \forall n \geq K$-sufficiently large, then $\sum_{n} x_n$ converges absolutely.
      \item If $\exists a \leq 1 \st \abs{\frac{x_{n+1}}{x_n}} \geq 1 - \frac{1}{n} \forall n \geq K$-sufficiently large, $\sum_{n} x_n$ does not converge absolutely.
    \end{enumerate}
  \end{itemize}
\end{prop}
\begin{remark}
  Proofs of these tests aren't really important (Dima-speaking), but knowing the conditions in which they apply is.
\end{remark}
\begin{prop}[Tests for Non-Absolute Convergence]
  \begin{itemize}
    \item \textbf{Alternating Series:} if $x > 0, x_{n+1} \leq x_n$ such that $\lim_{n \to \infty} x_n = 0$, then $\sum_{n} (-1)^n x_n$ converges.
    \item \textbf{Dirichlet's Test:} if $x_n$ decreasing with limit $0$, and the partial sum $s_n \defeq y_1 + \cdots + y_n$ is bounded, then $\sum_{n} x_n y_n$ converges.
    \item \textbf{Abel's Test:} let $x_n$ convergent and monotone, and suppose $\sum_{n} y_n$ converges. Then $\sum_n x_n y_n$ also converges.
  \end{itemize}
\end{prop}

\begin{defn}[Convergence of Series of Functions]
  We say a series $\sum_n f_n(x)$ converges absolutely to some $g(x)$ on $E$ if $\sum_n \abs{f_n(x)}$ converges for all $x \in E$.

  We say that the convergence is uniform if it is uniform for any $x \in E$, ie $\forall \epsilon > 0 \exists N \in \mathbb{N} \st \forall n \geq N, x \in E, \abs{g(x)- \sum_n f_n(x)} < \epsilon$.
\end{defn}

\begin{prop}[Interchanging Integrals and Summations]
  Suppose for $f_n : [a, b] \to \R$, $\sum_n f_n(x) \to g(x)$ uniformly and $f_n \in \RR[a, b]$. Then $\int_a^b g(x) = \sum_{n=1}^\infty \int_a^b f_n(x) \dd{x}$.
\end{prop}

\begin{prop}[Interchanging Derivatives and Summations]
  Let $f_n : [a, b] \to \R, f_n' \exists$, $\sum_{n}f(x)$ converges for some $[a, b]$ and $\sum_n f_n'(x)$ converges uniformly. Then, there exists some $g: [a, b] \to \R$ such that $\sum_n f_n \to g$ uniformly, $g$ differentiable, and $g'(x) = \sum_n f_n'(x)$, all on $[a, b]$.
\end{prop}

\begin{thm}[\important Cauchy Criterion of Series]
  $f_n(x) : D \to \R$ converges uniformly on $D$ iff $\forall \epsilon > 0, \exists N \st \forall m,n \geq N, \sum_{i=n+1}^m f_i(x) < \epsilon \forall x \in D$.
\end{thm}

\begin{prop}[Weierstrass M-Test]
  If $\abs{f_n(x)} \leq M_n \forall x \in D \subseteq \R$ and $\sum_n M_n < + \infty$, then $\sum_n f_n(x)$ converges uniformly on $D$.
\end{prop}

% \begin{remark}
%   Other than straight application of definitions, this is really our only test
% \end{remark}

\begin{defn}[Power Series]
  A function of the form $f(x) \defeq \sum_{n=0}^\infty a_n (x-c)^n$ is said to be a power series centered at $c$.

  Put $\rho \defeq \limsup_{n \to \infty} \sqrt[n]{\abs{a_n}}$, and put \[
  R \defeq \begin{cases}
    \frac{1}{\rho} & 0 < \rho < + \infty\\
    0 & \rho = +\infty\\
    \infty & \rho = 0
  \end{cases}.
  \]
  We call $R$ the radius of convergence of $f$.
\end{defn}

\begin{thm}[\important Cauchy-Hadamard]
  Let $R$ be the radius of converges of $f$. Then, $f(x)$ converges if $\abs{x - c} < R$, and diverges if $\abs{x -c} > R$.
\end{thm}
\begin{proof}[Sketch]
  Apply the root test to the definition of $R$.
\end{proof}

\begin{remark}
  If $\abs{x - c} = R$, the theorem is inconclusive, and we need to manually check.
\end{remark}
\end{document}