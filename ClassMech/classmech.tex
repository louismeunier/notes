\documentclass[12pt]{article}
\usepackage{amsthm}
\usepackage{libertine}
\usepackage[margin=0.15in]{geometry}
\usepackage{amsmath,amssymb}
\usepackage{multicol}
\usepackage[shortlabels]{enumitem}
\usepackage{siunitx}
\usepackage{setspace}
\usepackage{cancel}
\usepackage{graphicx}
\usepackage{pgfplots}
\usepackage{listings}
\usepackage{tabularx}
\usepackage{titlesec}
\usepackage{thmtools}
\usepackage{thm-restate}
\usepackage[bottom]{footmisc}
\usepackage{xcolor-solarized}
\usepackage{svg}
\usepackage{physics}
\usepackage[colorlinks=true, linkcolor=darkgray]{hyperref}
\usepackage{tikz}
\usepackage{quiver}
\usetikzlibrary{cd}
% \usepackage{cleveref}
\usepackage[]{csquotes}
\usepackage{hyperref}
\usepackage{mdframed}

\usepackage[createShortEnv]{proof-at-the-end}

% makes theorems, definitions, etc. "restatable" as shown
% can add more with same format as you wish
\renewcommand*{\proofname}{}


\declaretheorem[
  % thmbox=S,
  name=Definition,
  refname={Definition, definition}, numberwithin=section
]{definition}

\declaretheorem[
  % thmbox=S,
  name=Axiom,
  refname={Axiom, axiom},
  numberwithin=section
]{axiom}

\declaretheorem[
  % thmbox=S,
  name=Lemma,
  refname={Lemma, lemma},
  numberwithin=section
]{lemma}

\declaretheorem[
  % thmbox=S,
  name=Corollary,
  refname={Corollary, corollary},
  numberwithin=section
]{corollary}

\declaretheorem[
  % thmbox=S,
  name=Remark,
  refname={Remark, remark},
  numberwithin=section
]{remark}

\declaretheorem[
  % name=Exercise, 
  refname={Exercise, exercise},
  numberwithin=section
  % shaded={rulecolor=solarized-cyan, rulewidth=1pt, bgcolor={rgb}{1,1,1}}
]{exercise}

\declaretheorem[
  % thmbox=S,
  name=Theorem,
  refname={Theorem, theorem},
  numberwithin=section
]{theorem}

\declaretheorem[
  % thmbox=M,
  name=Example,
  refname={Example, example},
  numberwithin=section
  % shaded={rulecolor=solarized-cyan, rulewidth=1pt, bgcolor={rgb}{1,1,1}}
]{example}

\newEndThm[no proof here, restate, text proof={Examples}, one big link={\emph{See more}}]{definitionEnd}{definition}
\newEndThm[no proof here, restate, one big link = {\emph{Proof}}]{lemmaEnd}{lemma}
\newEndThm[no proof here, restate, one big link = {\emph{(Solution)}}]{exerciseEnd}{exercise}
\newEndThm[no proof here, restate]{axiomEnd}{axiom}

% \newEndThm[proof here, restate]{theoremEnd}{theorem}

% makes "quoted" text actually look correct
\MakeOuterQuote{"}

% page footer
\newpagestyle{mypage}{%
    \footrule
    \setfoot{\small\textcolor{gray}{§\thesubsection}}{\small\textcolor{gray}{\textit{\sectiontitle: \textbf{\subsectiontitle}}}}{\textcolor{gray}{\small p. \thepage}}
}

% title page settings
\newcommand{\pageauthor}{Louis Meunier}
\newcommand{\pagetitle}{Classical Mechanics}
\newcommand{\pagesubtitle}{PHYS251}

% black square for qed symbol
\renewcommand{\qedsymbol}{$\blacksquare$}

% annoying urls
\urlstyle{same}
\urldef\calc\url{https://notes.louismeunier.net/Calculus%20A%2C%20B/calculus.pdf#page=85 }

\titleformat{\section}
{\centering\normalfont\Large\bfseries}
{\thesection}{1em}{}

\begin{document}
\setstretch{2.25}
\noindent
\begin{center}
    \begin{tabularx}{\textwidth} { 
        >{\raggedright\arraybackslash}X 
        >{\raggedleft\arraybackslash}X}
    \LARGE \pageauthor \\
    \LARGE \textbf{\pagetitle} & \LARGE \textbf{\pagesubtitle}\\
    \end{tabularx}\\
    \rule[2ex]{0.8\textwidth}{1pt}
\end{center}
\begin{mdframed}[backgroundcolor=gray!20]
  \centering
  \underline{Course Summary:} 
  
  \textit{
  Covers classical mechanic topics including applying Newton's Laws, the Work-Energy Theorem, conservation of momentum, oscillations, frames of reference, centres of mass, orbits, Kepler's Laws, Lagrangian mechanics. Large Calculus focus, representing situations functionally with differential equations, graphical analysis of trends.}
\end{mdframed}
  
\setstretch{1.5}
\tableofcontents

% "enables" footer with section+subsection, etc. just comment it out if you don't want it
\pagestyle{mypage}

% makes sections a very dark gray + centered
\titleformat{\section}
{\color{darkgray}\centering\normalfont\Large\bfseries}
{\color{darkgray}\thesection}{1em}{}

% need to change margins and such here for rest of document
% kind of messy but what can you do
\newpage
% modify these as you wish
\newgeometry{margin=0.4in, top=0.4in, bottom=0.5in}
\parskip=0.5em

\section{Introduction \& Notations}
\textit{Content in this section provides a brief overview of the course, as well as basic problem solving, vector algebra, etc. It is safe to skip without loss of continuity.}
\subsection{Vectors}
This course deals mainly with \textit{scalars} (magnitude) and \textit{vectors} (magnitude and direction). We define algebra on vectors, briefly:
\begin{itemize}
  \item $\vb{A} + \vb{B} = \vb{B} + \vb{A}$ \textit{(commutativity of addition)}
  \item $\vb{A} + (\vb{B} + \vb{C}) = (\vb{A} + \vb{B}) + \vb{C}$ \textit{(associativity of addition)}
  \item $c(d\vb{A}) = (cd)\vb{A}$ \textit{(associativity of scalar multiplication)}
  \item $(c+d)\vb{A} = c \vb{A} + d\vb{A}$ \textit{(Distributivity of scalar multiplication)}
  \item $c(\vb{A} + \vb{B}) = c\vb{A} + c\vb{B}$ \textit{(Distributivity of scalar multiplication)}
\end{itemize}
and the operators:
\begin{itemize}
  \item $\vb{A} \times \vb{B} = \vb{C}$ s.t. $|\vb{C}| = |\vb{A}||\vb{B}|\sin \theta$, where $\theta$ is the angle between $\vb{A}$ and $\vb{B}$, and $\vb{C}$ is perpendicular to both $\vb{A}$ and $\vb{B}$. This is equivalent to computing $\det \begin{bmatrix} \vb{i} & \vb{j} & \vb{k} \\ A_x & A_y & A_z \\ B_x & B_y & B_z \end{bmatrix}$.  This \textit{cross product} is \textbf{anti-commutative}, meaning $\vb{A} \times \vb{B} = -\vb{B} \times \vb{A}$. Additionally, note that $\vb{A} \times \vb{A} = 0$.
  \item $\vb{A}\cdot \vb{B} = C = |\vb{A}||\vb{B}|\cos \theta$, where $C$ is a scalar. Note that $C = 0$ when $\theta = \pi/2$, ie $\vb{A}$ and $\vb{B}$ are perpendicular.
\end{itemize}

\subsection{Law of Cosines}

Consider a (planar) triangle constructed of sides $\vb{C} = \vb{A} + \vb{B}.$ We can write \begin{align*}
  \vb{C}^2 &= \vb{C}_x^2 + \vb{C}_y^2\\
  &= (|\vb{A}| - |\vb{B}|\cos \theta)^2 + (|\vb{B}|\sin \theta)^2\\
  &= |\vb{A}|^2 + |\vb{B}|^2 - 2|\vb{A}||\vb{B}|\cos \theta
\end{align*}


\subsection{Perspectives on the Cross Product}

\begin{align*}
  \vec{A} \cross \vec{B} &= (A_x \vb{i} + A_y \vb{j} + A_z \vb{k}) \cross (B_x \vb{i} + B_y \vb{j} + B_z \vb{k})\\
  &= A_x B_y (\vb{i} \cross \vb{j}) + \cdots \\
  &= (A_y B_z - A_z B_y) \vb{i} + \cdots \\
  &\equiv (\vb{A}\times\vb{B})_k = \sum_{i=1}^{3}\sum{j=1}^3\mathcal{E}_{ijk} A_i B_j
\end{align*}
Where $\mathcal{E}_{ijk} = \begin{cases}
  1; & $ijk$ \text{ even permutation of }123\\
  -1; & $ijk$ \text{ odd permutation of }123\\
  0; & \text{otherwise}
\end{cases}$ 

\subsection{Describing a Particle in Space in Polar Coordinates}

Consider a particle moving through space with a constant angular velocity $\dv{\theta}{t} = \omega$. We can describe this movement in terms of planar coordinates as $\vb{r}(t) = r_0 \cos (\omega t) \vb{i} + r_0 \sin(\omega t) \vb{j}$. Differentiating with respect to time, we obtain $\vb{v}(t) = -r_0 \omega (\sin(\omega t)\vb{i} - \cos (\omega t) \vb{j})$. Notice that $\vb{r} \cdot \vb{v} = 0 \forall t$; this should be familiar, as the velocity vector is always perpendicular to the position vector in purely circular motion. Differentiating again, we obtain $\vb{a}(t) = -r_0 \omega^2 (\cos(\omega t)\vb{i} + \sin (\omega t) \vb{j}) = -\omega^2 \vb{r}(t)$. In other words, the acceleration is always opposing the position vector (given the negative sign), and is proportional to the square of the angular velocity.

Assume now, instead, that the particle moves arbitrarily, described by a function $\vb{r}(t)$. In polar coordinates, this position vector is always travelling along the vector $\vectorunit{r}$, with magnitude $r$, and we can write $\vb{r}(t) = r \cdot \vectorunit{r}$. Differentiating:
\begin{align*}
  \vb{v}(t) = \dv{\vb{r}}{t} &= \dv{t}\left(r\cdot\vectorunit{r}\right)\\
  &= \dv{r}{t}\vectorunit{r} + r\dv{\vectorunit{r}}{t}\\
  &= \dot{r}\vectorunit{r} + r \dv{t}\left(\cos \theta \vb{i} + \sin \theta \vb{j}\right)\\
  &= \dot{r}\vectorunit{r} + r \left(-\sin \theta \vb{i} + \cos \theta \vb{j}\right)\dot{\theta}\\
  &= \dot{r}\vectorunit{r} + r \dot{\theta} \vectorunit{\theta}
\end{align*}

Recalling that $\begin{cases}
  \vectorunit{r} = \cos \theta \vb{i} + \sin \theta \vb{j}\\
  \vectorunit{\theta} = -\sin \theta \vb{i} + \cos \theta \vb{j}
\end{cases}$.
Differentiating again:
\begin{align*}
  \vb{a}(t) = \dv{\vb{v}}{t} &= \dv{t}\left(\dot{r}\vectorunit{r} + r \dot{\theta} \vectorunit{\theta}\right)\\
  &= \ddot{r}\vectorunit{r} + \dot{r}\dot{\theta}\vectorunit{\theta} + \dot{r}\dot{\theta}\vectorunit{\theta} + r \ddot{\theta}\vectorunit{\theta} + r \dot{\theta}(-\vectorunit{r})\\
  &= \left(\underbrace{\ddot{r}}_{\text{radial}} - \underbrace{r \dot{\theta}^2}_{\text{centripetal}}\right)\vectorunit{r} + \left(\underbrace{2\dot{r}\dot{\theta}}_{\text{coriolis}} + \underbrace{r \ddot{\theta}}_{\text{tangential}}\right)\vectorunit{\theta}
\end{align*}

we obtain a decomposition of the acceleration vector into radial ($\vectorunit{r}$) and tangential ($\vectorunit{\theta}$) components, labelled accordingly. Note that the centripetal acceleration, labelled, is the same as the acceleration in circular motion mentioned previously.\footnote{None of the other "components" of acceleration are present in the constant angular velocity case because (1) $\ddot{r}=0$, ie no radial acceleration, so the radial and coriolis components are zero, and (2) $\dot{\theta} = \omega \implies \ddot{\theta} = 0$, so the tangential acceleration is zero, leaving only the centripetal acceleration.}\footnotemark

\footnotetext{See \calc for a different perspective on this topic.}

% TODO: update prev
% statics, etc

\subsection{Newton's Laws of Motion}
\begin{enumerate}
  \item[N1:] \textit{"in absence of external force, a body at rest remains at rest, and a body in motion remains in motion, with the same speed \& same direction"}
  
This law defines \textit{intertial frames}, ie ones in which the law holds.

  \item[N2:] $\vec{F} = m \vec{a}$
  \item[N3:] \textit{"if a body \textbf{b} applies a force on body \textbf{a}, then \textbf{a} applies a force on \textbf{b} such that $\vec{F}_a = -\vec{F}_b$ (equal and opposite)"}\footnote{$\implies$ conservation of momentum \dots}
\end{enumerate}

\begin{example}
  Consider two blocks \textbf{A}, \textbf{B} where \textbf{A} lies atop \textbf{B} which all lie upon "the earth".

  \textbf{A} experiences the force $W_A$ due to gravity and the force $F_1$ back from \textbf{B}.

  \textbf{B} experiences the normal force $N$ from the table, the force of gravity $W_B$, and finally the force $F_2$ due to \textbf{A}.

  We can write (from N2) \[\vec{F}_1 + \vec{W}_A = m_A \vec{a}_A\] and similarly
  \[\vec{N} + \vec{W}_B + \vec{F}_2 = m_B \vec{a}_B\]
  Static situation $\implies \vec{a}_A = \vec{a}_B = 0$.
  We can further simplify writing $F_1 = m_A g$ (N2), and finally \[N = W_A + W_B = (m_A + m_B)g.\]
\end{example}

\begin{example}
  Consider a mass $m_1$ laying on a table, connected via a string to a mass $m_2$ hanging off the table, where the string is of small mass and is under a tension $T$. (Looking at a functionally massless stirng, we can consider there to be a constant tension as we can say $T - T' = \delta m a \implies T - T' = 0 \implies T = T'$, where $T, T'$ is the tension on a particular segment of the string in opposing directions).

  On $m_1$, we have the normal force $N$, tension $T$, and weight $m_1 \cdot g$.

  On $m_2$, we have the weight $m_2 \cdot g$, the tension $T$ (equal throughout string as explained above).

  Together, we can write (N2; taking $x$ to represent a motion "left" and $z$ to represent a motion "down") 
  \[- T = m_1 \ddot{x}\] 
  and 
  \[m_2g-T = m_2 \ddot{z}.\]

  Noting that the string must "retain" its length we can write \[l(t) = x(t) + z(t),\] where $x$ and $z$ represent the length of the string in the $x/z$ axes. However, as $l$ must stay constant, we can differentiate twice wrt time to obtain \[0 = \ddot{x} + \ddot{z}.\]
  All together, then we have
  \[
    \ddot{x} = \frac{-m_2 g}{m_1 + m_2}.
  \]
  Integrating twice, we have \[
    x(t) = \frac{-m_2 g}{m_1 + m_2} \cdot \frac{t^2}{2}.
  \]
\end{example}

\begin{example}
  Consider a mass $m_1$ connected to a string which lies along a pulley $P_1$, which then attaches to the center of pulley $P_2$, which has a string whicch, on one end, is attached to a mass $m_2$, and is grounded in the other.

  On $m_1$, we have tension $T$ and the weight $m_1 \cdot g$.
  
  On $m_2$, we have the weight $m_2 \cdot g$ and the tension $T'$ (no reason for $T = T'$; strings aren't connected).

  Consider $P_1$ - it is nailed to the wall at its center, and experiences some $F$ ("up") from the nail, as well as the tension $T_1$ down (\textit{twice}). 

  On $P_2$, we have the tension $T_1$ ("up") and the tension $T_2$ ("down", "twice").

  We can write \begin{align*}
    T_1 - W_1 &= m_1 a_1\\
    T_2 - W_2 &= m_2 a_2\\
    T_1 - 2T_2 &= m_{P_2} \cancel{a_{P_2}}\implies T_1 = 2 T_2
  \end{align*}
  As last time, we can use the fact that the strings cannot stretch. Consider $m_1$ to be at height $y_1$, $P_1$ to be at height $y_{P_1}$, $P_2$ to be at height $y_{P_2}$, and $m_2$ to be at height $y_{2}$.

  We can then take the length $l_1$ of the string about $P_1$ as
  \[
  l_1 = (y_{P_1} - y_1) + (y_{P_2}-y_2) + \pi R_{P_1} 
  \]
  Differentiating twice wrt time, we have
  \[
  0 = a_1 + a_{P_2} \implies a_1 = - a_{P_2}.
  \]
  Analyzing $P_2$ similarly, we will obtain
  \[
    a_2 = 2 a_{P_2}.
  \]

  All together, we have 
  \[
  a_1 = \left(\frac{2m_2 - m_1}{4m_2 + m_1}\right) g.
  \]
\end{example}

Consider a collection of particles. The force on any particular particle, say 1, we can denote \[\vec{F}_{1} = \vec{F}_{12} + \vec{F}_{13} + \dots + \vec{F}_{1n} + \vec{F}^{\text{ext}}_1.\] Notice that any force $\vec{F}_{nm} = - \vec{F}_{mn}$, by N3. Thus, if we add all the forces on all the particles in the bag, we will always have a "pairing" of forces such that each $\vec{F}_{mn}$ is "canceled" by another, leaving behind only the external forces, ie \[\vec{F}^{\text{ext}} = \sum_{i} \vec{F}^{\text{ext}}_{i}.\] Say there are no external forces; then, we have 
\begin{align*}
  \sum_i m_i \vec{a}_i &= 0\\
  \dv{t} \sum_i m_i \vec{v}_i &= 0.\\
\end{align*}
And thus, we have conservation of momentum.

\subsection{Projectile Motion}
Say we have a particle with an initial velocity $v_0$ launched at an angle $\theta$. This particle experiences just one force, $mg$, and by N2 we have \[\cancel{m} \ddot{y} = -\cancel{m}g.\] Integrating, we have \[y(t) = y(t=0) + \underbrace{v_0 \sin \theta}_{\text{vertical component of $v_0$}} t - \frac{g t^2}{2}.\]

In the $x$, we have $m \ddot{x} = 0$, $x(t) = v_0 \cos \theta t$. From here, we can rewrite $x(t)$ as $t$ as a function of $x$ and substitute into $y$ for a function $y = f(x)$. This gives \[
y = \frac{v_0 \sin \theta x}{v_0 \cos \theta} - \frac{g}{2} \frac{x^2}{v_0^2 \cos^2 \theta} (= ax - bx^2)  
\]

\subsection{Forces}

We (generally) have:
\begin{itemize}
  \item 2 "protons" @ 1 fermi (fm = $10^{-15} m$)
  \item Nuclear strong, $2 \times 10^3 N$
  \item E-M, $2 \times 10^2 N$
  \item Weak force, $2 \times 10^{-11} N$
  \item Gravity, $2 \times 10^{-34} N$
\end{itemize}
We have the force between two point masses, $m_a$, $m_b$, at a distance $r_{ab}$:
\[\vec{F} = \frac{- G m_a m_b}{r^2_{ab}},\] noting that the negative sign $\implies$ gravity \textit{attractive}, and where the gravitational constant $G = 6.67 \times 10^{-11} \frac{\text{N} \cdot \text{m}^2}{\text{kg}^2}$.

We consider some "extended object", a spherical shell of mass $M$ a distance $r$ (center) from a "smaller" object $m$. We can subdivide this shell into infinitesimal pieces, then sum (integrate) their respective forces using the formula above to find the total force.

Say the distance between a particular "subsection" of the sphere is $s$ from $m$, and the change in angle in the shell section we have $\dv{\theta}$, with $\theta$ the angle from the shell section to $r$, and $\varphi$ is the angle between $s$ and $r$.

We have \begin{align*}
  F &= \frac{-G m \text{d}M}{s^2} \cdot \cos \varphi.\\
  \dots
\end{align*}

\section{Statics}
We have \textbf{static} motion when the net external force is $0$. Clearly, then,
\[\sum F_\text{ext} = m \cancel{\vec{a}} = 0 \iff \vec{a} = 0.\]

When analyzing a system we thus need to balance:
\begin{itemize}
  \item \textbf{Forces,} include
  \begin{itemize}
    \item Tension;
    \item Normal ($\perp$ to surface, N3);
    \item Friction (opposing direction of motion; $F_k = \mu_k N, F_s \leq \mu_s N$, kinetic vs static frictions resp.);
    \item Gravity ($F_g = \frac{GM m}{R^2} \implies F_g = mg$, on earth);
    \item Spring (\textit{Hooke's Law}, $\vec{F_s} = -k \vec{r}$ )
  \end{itemize}
  \item \textbf{Torques,} $\vec{\tau} = \vec{r} \cross \vec{F}.$ When dealing with rotational motion, sums of torques must be 0 as well to conserve stasis.
\end{itemize}

\begin{example}
  Consider a ladder of length $l$ and mass $m$ leaning against a frictionless wall with another end on a floor with coefficient of friction $\mu$. Assuming the center of gravity is at the geometric center of the ladder ($\frac{l}{2}$), we can analyze the system as follows.\\
  Take $N_1, N_2$ to represent the normal forces on the ladder from the floor and the wall resp, and $F_f$ as the friction force of the floor. We summarize the forces in the $[x]$ and $[y]$, and the torques:
  \begin{align*}
    [y]&\quad 0 = N_1 - mg \implies N_1 = mg \quad \textbf{(1)}\\
    [x]&\quad 0 = N_2 - F_f \implies N_2 = F_f \quad \textbf{(2)}\\
    [\tau] &\quad 0 = \vec{F} \cross \vec{r} \overset{\text{1D}}{=} F \cdot r = \underbrace{mg \cdot \frac{l}{2} \cdot \cos \theta}_{\text{rotation about c.o.m.}} - \underbrace{N_2 \cdot l \sin \theta}_{\text{about "floor point"}} \implies N_2 = \frac{mg}{2 \tan \theta} \quad \textbf{(3)}
  \end{align*}
  From \textbf{(1),(2),(3)}, we can find that the angle required for the ladder to not slip (ie, net forces and net torques sum to zero; will not slip nor rotate and fall), we require that \[\tan \theta \geq \frac{1}{2\mu}.\]
\end{example}

\section{Non-Zero Forces}


\subsection{Trends of Common Forces}
% TODO: with plots
% TODO: forces on a big sphere.
% TODO: conical pendulum
% TODO friction

\subsection{Generalizing Forces as Functions}

Consider the general form of Newton's Second Law: \[\vec{F} = m \vec{a}.\]
This $\vec{F}$ can take a number of forms, depending on its dependence on different variables. Consider;
\begin{itemize}
  \item \textbf{Time}: $F(t) = m \dv[2]{x}{t} = m \dv{v}{t}$. This is separable ODE:
  \begin{align*}
    \int \dd{v} = v(t) &= \int \dd{t} \frac{F(t)}{m}\\
    \implies x(t) &= \int \dd{t} v = \iint \dd{t} \frac{F(t)}{m}
  \end{align*}
  \item \textbf{Position}: $F(x) = m \dv{v}{t}$. We can rewrite our acceleration by the chain rule as $\dv{v}{t} = \dv{v}{x}\cdot \dv{x}{t} = \dv{v}{x}v$ (NB: this "trick" is very helpful in evaluating physical ODE problems such as this, where the reliance on a particular function of time is unknown, but can be obtained  from physical properties...). We can then write:
  \begin{align*}
    m \dv{v}{x}v &= F(x)\\
    m \int \dd{v} v &= \int F(x) \dd{x}\\
    \eval{\frac{1}{2}mv^2}_{D} &= \int_{D} \dd{x} F(x) \qquad \textit{NB: Work-Energy Thm}
  \end{align*}
  \item \textbf{Velocity}: $m \dv{v}{t} = F(v) \implies m \int \dd{v} \frac{1}{F(v)} = \int \dd{t}$; we will see this more later.
\end{itemize}
\subsection{Viscous Resistance}

Consider some force that is dependent on velocity, $F(v)$ (friction, air resistance for instance, is often approximately modeled this way). Let's approximate this force function at small velocities by writing its Taylor Series expansion about $v= 0$ (note that higher order terms $O(v^n)$ can be consider negligible assuming $0 \leq v < 1$, a typical approach to this type of small-perturbation analysis):
$$F(v) = F(0) + v \eval{\pdv{F}{v}}_{v=0} + \frac{v^2}{2!}\cdot \eval{\pdv[2]{F}{v}}_{v=0} + \cdots + O(v^n)$$

Note that $v = 0 \implies a = 0 \implies F(0) = 0$, and thus the first term can be disregarded. Under a further assumption that $v\ll 1$ and remains so, we can eliminate $O(n^2)$ terms, and would be left with \[F(v) = - v \eval{\pdv{F}{v}}_{v=0} = -cv,\] with $c:=\eval{\pdv{F}{v}}_{v=0}$, a constant. This can be rewritten and solved as a simple ODE;
\begin{align*}
  F(v) = m \dv{v}{t} &= -cv\\
  \frac{m}{v} \dd{v} &=-c \dd{t}\\
  m \ln v &= -ct + k\\
  v(0) = v0 &\implies k = m \ln v_0\\
  &\implies v(t) = v_0 e^{-\frac{ct}{m}}\\
  \tau := \frac{m}{c},\quad& v(t) = v_0 e^{-\frac{t}{\tau}}\\
  &\implies x(t) = x_0 + v_0 \tau(1-e^{-\frac{t}{\tau}})
\end{align*}
% TODO: add plot?

\begin{example}[Air resistance $\propto v$]
  Consider a sky diver falling, with some initial velocity $v_0$, under the influence of gravity $F_g$ downwards (positive) and air resistance $F_v$ upwards (negative). Assuming that air resistance is proportional and opposing to velocity, we can write $F_v = - cv$ where $c$ a constant. From Newton's Second:
  \begin{align*}
    \sum F = ma = m \dv{v}{t} = F_g + F_v &= mg - cv\\
  \int_{v_0}^v \dd{v^*} \frac{m}{mg - ct} &= \int_0^t \dd{t^*}\\
  -\frac{m}{c}\ln\left(\frac{mg - cv}{mg - cv_0}\right)  &= t\\
  v(t) &= g \tau (1-e^{-\frac{t}{\tau}}) + v_0e^{-\frac{t}{\tau}}, \quad \tau := \frac{m}{c}
  \end{align*}
  A common follow-up to this analysis is to compute the terminal velocity $v_t$. We will consider two methods to compute it in this situation.\\
  First, consider the equation we found for $v(t)$. Taking the limit of $t \to \infty$, we have
  \begin{align*}
    v_t = \lim_{t \to \infty} v(t) &= \lim_{t \to \infty} \left[g \tau (1 -\cancelto{0}{e^{- \frac{t}{\tau}}}) + \cancelto{0}{v_0 e^{-\frac{t}{\tau}}}\right]\\
    &= g\tau = \frac{gm}{c}
  \end{align*}
  Alteratively, consider our original formula,\[ma = mg - cv.\] Terminal velocity is reached when $a = 0$; more physically, when the forces in opposing directions completely cancel each other. Thus, we can solve for $v$ directly;
  \[m\cdot 0 = mg - cv \implies mg = c v_t \implies v_t = \frac{gm}{c},\] equivalent to our previous computation.
\end{example}

\begin{example}[Air resistance $\propto v^2$]
  Consider the previous example with $F_v = -c v^2$. We can write \[\sum F = m \dv{v}{t} = mg -cv^2 \implies v_t = \sqrt{\frac{mg}{c}}.\] A closed form of the position of the trajectory is possible, but difficult.
\end{example}

% TODO: move projectile motion, etc here.
\section{Oscillations}
 
Many natural systems experience \textit{oscillations,} characterized by some type of force that leads to repetitive motion.
\subsection{Simple Springs}

\textbf{Hooke's Law} states that $F_s = -kx$, where $k$ is a constant spring coefficient (typically $0 < k < 1$), and $x$ is the displacement \text{from equilibrium}. In general, we can write \begin{align*}
  F_s = m \dv{v}{t} = m \dv{v}{x}v &= - kx\\
  m\int \dd{v} v &= -k \int \dd{x} x\\
  m \frac{v^2}{2} &= -k \frac{x^2}{2}\\
  mv &= -kx \implies v = -\frac{k}{m}x
\end{align*}
Often, we write $\omega^2 := \frac{k}{m},$ the phase of the oscillator.\\
\subsubsection{Series vs Parallel}
Consider a spring of coefficient $k_1$ connected end-to-end to a spring of $k_2$, and let $x_1$, $x_2$ represent the equilibrium points of the spring system resp (where $x_2 = x_1 + $ equilibrium of spring $k_2$). Consider the springs as massless (or of being of infinitesimal, negligible mass), so the only forces in question are due to the spring force(s).\\
At the endpoint of spring 1, we have \[F_1 : k_1 x_1 = (x_2 - x_1)k_2 \implies x_2 = \left(\frac{k_1+k_2}{k_2}\right)x_1.\] If we consider a mass on the end point of the system, we can write \begin{align*}
  F = -k_2 \underbrace{(x_2 - x_1)}_{\text{displacement}} &= - \left(\frac{k_1 k_2}{k_1 +k_2}\right) x_2\\
  &= - k_{\text{eff}} x_2
\end{align*}
Note that $\frac{1}{k_\text{eff}} = \frac{1}{k_1}+\frac{1}{k_2}$; this is the relationship between any number of springs connected in \textbf{series}.

Next, consider two springs of $k_1$, $k_2$ each connecting a mass to a wall. We can write the force on the mass, taking $x$ to represent the displacement of the mass from equilibrium, as\[F = -k_1 x - k_2 x = -(k_1+k_2)x = - k_\text{eff} x,\]
where $k_\text{eff} = k_1 + k_2$; these springs are in \textbf{parallel}.

(Naturally, all of these analyses are under the assumption of no torsion, no gravity, etc (no other external forces))

\subsection{Simple Harmonic Motion}

Consider again a spring ($k$). From N2, we can write:
\begin{align*}
  m \dv[2]{v}{t} = -kx &\implies m \ddot{x} + kx = 0\\
  &\implies \ddot{x} + \omega_0^2 x = 0
\end{align*}

This is a (fairly straightforward) second order linear homogenous equation (we will see non-homogenous versions of this later, when we introduce external forcing, etc.). The typical means of solving these is to assume $x(t)$ is of the form $A e^{\alpha t}$, plug in, and solve for the relevant constants (noting that $\alpha \in \mathbb{C}$):
\begin{align*}
  (A e^{\alpha t})'' + \omega^2 Ae^{\alpha t} &= 0\\
  \cancel{A} \alpha^2 \cancel{e^\alpha t} + \omega^2 \cancel{Ae^{\alpha t}} &=0\\
  &\implies \alpha^2 = -\omega^2\implies \alpha = \pm i \omega, i = \sqrt{-1}
\end{align*}
Note that all dependence on time cancelled, as desired, as well as $A$; we are working with a second-order ODE, so we must have two "free" constants dependent on initial conditions, represented by this $A$.\\
Now, we have solutions of the form $x(t) = Ae^{\pm i \omega t}$. This form doesn't help us much physically, but we can simplify using Euler's formula, or, since we are dealing with purely a sign change in our parameters, the definition of $\cos (ix) = \cosh x  = \frac{e^{x}+e^{-x}}{2} \implies \frac{e^{i \omega t} + e^{-i \omega t}}{2} =\cos (\omega t)$. Generally, then, we can write \[x(t) = A \cos (\omega t + \varphi),\] where $A, \varphi$ are constants, typically defined by IVs.\\Note that, equivalently, we could have written \[x(t) = B \sin \omega t + C \cos \omega t,\] but chose to rewrite in terms of $\varphi$ for a more intuitive, physical meaning. Specifically, we have \[A \cos \varphi = C, A \sin \varphi = -B, \tan \varphi = - \frac{B}{C}.\] Graphically, this means that spring motion is a constant-amplitude sinusoidal wave that will oscillate forever. Realistically, forces such as friction, gravity, etc would cause a damping force over time and eventually slow the mass to a halt.
\subsection{Damping Forces}
Consider a mass $m$ connected to a spring $k$, but now add some form of resistance (typically described as a piston), where force opposes the motion of the mass and is proportional to velocity. We write $F_r = - b v$ where $b$ a constant. Writing the equation of motion:
\begin{align*}
  m \ddot{x} = - kx - bv\\
  \ddot{x} + \frac{b}{m}\dot{x} + \frac{k}{m}x = 0\\
  \equiv \ddot{x} + \gamma \dot{x} + \omega^2 x=0
\end{align*}
We now have a second order linear homogenous ode, but with an added $\dot x$. We can try the same approach, with a trial solution $x(t) = A e^{\alpha t}$:

\begin{align*}
  (Ae^{\alpha t})'' + \gamma (A e^{\alpha t})' + \omega^2Ae^{\alpha t} &= 0\\
  \underbrace{\alpha^2 + \alpha \gamma + \omega^2}_{\text{characteristic polynomial}} &= 0\\
  &\implies \alpha = -\frac{\gamma}{2} \pm \sqrt{\frac{\gamma^2}{4} - \omega^2}
\end{align*}
We thus have a number of cases based on the discriminant. Let $\tilde{\gamma} = \frac{\gamma}{2}$.
\begin{itemize}
  \item \textbf{underdamped; }$\tilde{\gamma}^2 < \omega^2 \implies $ discriminant $\in \mathbb{C}$, thus $\alpha = - \frac{\gamma}{2} \pm i \sqrt{\omega^2 - \frac{\gamma^2}{4}}.$ We then have a solution
  \begin{align*}
    x(t) &= e^{-\tilde{\gamma} t} = A_1 e^{i \tilde{\omega} t} + A_2 e^{-i \tilde{\omega} t}\\
    &\implies \boxed{x(t) = \underbrace{A e^{-\tilde{\gamma}t}}_{\text{damping, } \to 0} \underbrace{\cos(\tilde{\omega}t + \varphi)}_{\text{oscillating}}}
  \end{align*}
   As $t\to \infty$, $x(t)\to 0$ because of the damping term. The inverse exponential goes to zero, functionally diminishing the amplitude of the cosine with time.
  \item \textbf{overdamped; }$\tilde{\gamma}^2 > \omega^2 \implies$ discriminant $\in \mathbb{R}$, thus $\alpha = - \tilde{\gamma} \pm \sqrt{\tilde{\gamma}^2 - \omega^2} \implies \alpha \leq 0$. This yields a purely exponential solution,
  \[\boxed{x(t) = Ae^{-\alpha_-t} + B e^{-\alpha_+ t}}\] since the oscillating term comes from an imaginary number in the exponential. Thus, there is no oscillation, and the solution tends to zero.
  \item \textbf{critically damped; }$\tilde{\gamma}^2 = \omega^2\implies$ discriminant $=0$, we have only one root, and we thus are missing a solution case. One can use a new trial solution $x(t) = B t e^{-\tilde{\gamma}t}$, and after some computation, we have
  \[\boxed{x(t) = Ae^{-\tilde{\gamma}t} +Bte^{-\tilde{\gamma}t}}\] Note the linear $t$ term in one of the exponentials. This will persist when looking at $x'(t)$, and will thus dominate the inverse exponential as $t \to 0$, hence the term "critically damped", as this type of damping will bring $x(t)$ to zero the fastest given the same initial conditions and spring constant.
\end{itemize}
% TODO: plots
\subsection{Driven Oscillators and Resonance}

% TODO: drawing

Consider a spring ($k$) connected to a mass $m$ on one end and to a wall which is moving as a function of time, where $x_w(t)$ represents the location of the wall from equilibrium. We write \[x_w(t) = x_0 \cos \omega t,\] ie the wall oscillates. From N2:
\begin{align*}
  m \ddot{x} = -kx + kx_w &= -kx + \underbrace{k x_0}_{:= F_0} \cos \omega t\\
  \ddot{x} + \omega_0^2x &= \frac{F_0}{m} \cos \omega t
\end{align*}
Solving for $x(t)$:
\[x(t) = B \cos (\omega_0t + \varphi) + \frac{F_0}{m} \left(\frac{1}{w_0^2 - \omega^2}\right)\cos (\omega t).\]
This is the general equation for a \textbf{driven oscillator}.\\
Note too that \[\lim_{\omega \to \omega_0^-} = \infty; \quad \lim_{\omega \to \omega_0^+} = - \infty,\] an effect we call \textbf{resonance}. In reality, of course, some sort of damping (eg friction) makes these not go to infinity, but grow large/small regardless.\\
\subsubsection{With Damping}

Consider \[\ddot{x} + \gamma \dot x + \omega_0^2 x = \frac{F_0}{m} \cos \omega t,\] an equation for a damped, driven oscillator. This can be solved similarly to the previous, and also gives different damped cases. Consider the underdamped case:
\[Ae^{-\tilde{\gamma}t}\cos(\tilde \omega t + \varphi_1) + \frac{F_0}{m}\cdot \frac{\cos(\omega t + \varphi_2)}{\left[(\omega_0^2-\omega^2)^2+\omega^2\gamma^2\right]^\frac{1}{2}}.\] When we are near resonance, $\omega \approx \omega_0 \implies \omega_0^2 - \omega^2 = (\omega_0 - \omega)(\omega_0+\omega) \approx 2 \omega_0,$ so we can approximate the denominator of the driven part as \[\left[\cdots \right]^\frac{1}{2}\approx \left[(\omega_0-\omega)^2 + \left(\frac{\gamma}{2}\right)^2\right].\]
This $\left(\frac{\gamma}{2}\right)^2$ term serves as a "regulator" to the resonance effect.
% TODO: add plots

\subsection{Coupled Oscillators}
% TODO: picture
Consider a series of three springs and two equal masses of $m$ between a wall. Let $x_1$ be the displacement of the leftmost mass, and $x_2$ the displacement of the rightmost. We can write
\[\textbf{(i)}\quad m\ddot{x_1} = -kx_1 + k(x_2 -x_1);\qquad \textbf{(ii)}\quad m\ddot{x_2} = -k(x_2-x_1)-kx_2\]
Adding and subtracting \textbf{(i),(ii)}:
\[\ddot{x_1}+\ddot{x_2}+\omega_0^2(x_1+x_2)=0; \qquad (\ddot{x_1} - \ddot{x_2})+3 \omega_0^2 (x_1 - x_2) = 0.\] We can let $y_1(t) = x_1 + x_2$ and $y_2(t) = x_1 - x_2$, and solve these separately as two homogenous second order ODEs, then add them back together. These $y_1, y_2$ are called the \textbf{normal modes}. This will yields
\begin{align*}
  x_1(t) = B_+ \cos(\omega_0t + \varphi_+) + B_- \cos(\sqrt{3}\omega_0t + \varphi_-)\\
  x_2(t) = B_+ \cos(\omega_0t+\varphi_+) - B_- \cos(\sqrt{3} \omega_0 t + \varphi_-)
\end{align*}

We dub $\omega_0, \sqrt{3}\omega_0$ the \textbf{normal frequencies}. If both masses are given the same initial displacements, they will both oscillate at $\omega_0$; if they are given the same initial displacement but with opposite signs, they will oscillate at $\sqrt{3}\cdot\omega_0$.

Note that, while this is being applied in the context of springs, similar ideas can be used in other forces proportional to displacement.

\begin{example}
  Take a "coupled" system of pendulums attached to a ceiling, with point masses at their ends and a spring connecting the two masses. Find their equations of motion and their normal frequencies.
  % TODO
\end{example}


\section{Momentum}

\subsection{Some Derivations}

We can write \[\vec{F}^{\text{ext}} = \sum_i m_i \ddot{r}_i = M \ddot{\vec{R}},\] where $\vec{R}$ is the "center of mass coordinate"., ie \[\vec{R} - \frac{1}{M} \sum_i m_i \vec{r_i}.\]


\subsection{Center of Mass}

\begin{example}[Linear density]
  Consider a rod with linear density (the density at a given point is proportional to the distance along the rod), $\lambda$;
  \[\lambda(s) = \lambda_0 \left(\frac{S}{L}\right) = \dv{m}{x},\] where $S$ is the point along the length. The total mass:
  \[M = \int_\text{rod} \dd{m} = \int_0^L \dd{m} = \int_0^L \lambda \dd{x} = \frac{x^2}{2}\cdot\eval{\frac{\lambda_0}{L}}_0^L = \frac{L \lambda_0}{2}.\]
  We can now compute the center of mass $\mathsf{X}$:
  \[\mathsf{X} = \frac{1}{M}\sum_i x_i m_i \overset{\text{continuous}}{=} \frac{1}{M}\int_0^L \dd m \,x = \frac{2}{L\lambda_0} \int_0^L \lambda \dd{x} = \frac{2}{L\lambda_0} \int_0^L \lambda_0 \frac{x^2}{L} \dd{x} = \frac{2}{L \lambda_0} \cdot \frac{\lambda_0 L^3}{3L} = \frac{2}{3}L.\] Note that there is no dependence on $\lambda$!
\end{example}

\begin{example}[Triangular sheet]
  Consider a triangular sheet of height $h$ and base $b$, of uniform density $\rho$. We write \[\vec{R} = \frac{1}{M} \cdot \int \underbrace{\rho \dd{V}}_{\dd{M}} \vec{r}.\]
  We have that $\dd{V} = \dd x \dd y \dd z$. We will assume that it is negligibly thin, so let $\dd z = 0$. This means we can write mass as $M = \rho \dd{x}\dd{y} := \rho A$, where $A$ is the area of the sheet. We can now compute its center of mass:
  \begin{align*}
    X_c &= \frac{1}{A}\iint \dd x \dd y x = \frac{hb}{2}\int \dd x \cdot x \cdot \int_0^{\frac{h}{b}x} = \frac{2}{3}b\\
    Y_c &= \cdots = \frac{h}{3}
  \end{align*}
  Thus, the center of mass is $(X_c,Y_c)=(\frac{2}{3}b,\frac{h}{3})$.
\end{example}

\subsection{Variable Mass Problems}

We define \textbf{momentum} as $P(t) = M\cdot v$, and $\dv{t} P(t) = F$. A common technique to approaching problems involving variable mass is to approach it first via infinitesimal changes in different variables over some $\Delta t$, then bring $\Delta t \to 0$ to find $F$. Generally, though, we have that \[\dv{t}P = \dv{t}(Mv) = \dv{M}{t} v + M \dv{v}{t}M = \dot{M}v +M \dot{v} = F.\]


\begin{example}
  Consider a cart of mass $M$, collecting rain at a rate $\dv{m}{t} = \sigma$, and traveling at a constant velocity $v$. What force must we apply for this constant velocity to persist, if any? Consider its momentum at some time $t$, and swiftly later at $t + \Delta t$:
  \begin{align*}
    P(t) &= Mv\\
    P(t+\Delta t) &= (M+ \Delta)v \\
    & \lim_{\Delta t} \frac{\Delta P}{\Delta t} = \lim \frac{\Delta m}{\Delta t} v \implies \dv{P}{t} = \dv{m}{t}v = \boxed{\sigma v = F}
  \end{align*}
  Say, now, we let $F = 0$. We would then have, setting $M_c$ as the mass of the cart alone, \begin{align*}
    P(t) &= (M_c + \sigma t)v\\
    P(t+\Delta t) & = (M_c + \sigma t + \sigma \Delta t)(v + \Delta v)\\
    &\implies \Delta P = \sigma v \Delta t + (M_c + \sigma t) \Delta v = 0\\
    &\implies\int \dd{t} \frac{\sigma}{M_c + \sigma t} = -\int \dd{v} \frac{1}{v}\\
    &\implies v = v_0 \frac{M_c}{M_c + \sigma t}
  \end{align*}
\end{example}

\subsubsection{Rocket Motion}

Consider a rocket of mass $m$ moving at velocity $\vec{v}$ and expelling fuel of mass $\Delta m$, which leaves the rocket at some velocity $\Delta u$. We can write:
\[\vec{P}(t) = m \vec{v}; \qquad \vec{P}(t+\Delta t) = (m-\Delta m)(\vec{v} + \Delta v)+ \Delta m (\vec{u} + \vec{v} + \Delta v)\] Working out the differentials, this yields the famous \textbf{rocket equation} \begin{align*}
  \boxed{\vec{F} = m \dv{\vec{v}}{t} - \dv{m}{t}\vec{u}}
\end{align*}

\begin{example}[Rocket Eqn: \underline{Free space}]
  $\vec{F} = 0 \implies m \dv{\vec{v}}{t} = \dv{m}{t} \vec{u} \implies \dd \vec{v} = \vec{u} \frac{\dd{m}}{m} \implies \vec{v_f} = - \vec{u} \ln \left(\frac{m_0}{m_f}\right).$
\end{example}
\begin{example}[Rocket Eqn: \underline{Force due to gravity}]
  $F = mg = m \dv{\vec{v}}{t} - \vec{u}\dv{m}{t} \implies \vec{v_f} = gt - \vec{u} \ln \left(\frac{m_0}{m_f}\right)$
\end{example}

\begin{example}[Rope falling on a scale]
  % TODO: picture?
  Consider\footnotemark a rope of mass $m$ that is held above a scale such that part of it is resting on the scale. What does the scale read when the rope is left to fall?\\
  % TODO: finish
\end{example}
\footnotetext{\href{https://www.youtube.com/watch?v=OufI0NBIx4w}{See this video}}

\section{Work \& Energy}

\subsection{Introduction: 1 Dimension}

Consider force as a function of position, $\vec{F}(\vec{r}) = m \dv{\vec{v}}{t}$. We can simplify, in one direction, \begin{align*}
  m \dv{v}{t} = m \dv{v}{x}\cdot\cancelto{v}{\dv{x}{t}} &= F(x)\\
  \implies \int_{v_0}^v \dd{v^*} v^* &= \int_{x_0}^x \dd{x^*} F(x^*)\\
  \underbrace{\frac{1}{2}mv^2 - \frac{1}{2}mv_0^2}_{\Delta \text{KE}} &= \underbrace{\int_{x_0}^x \dd{x^*} F(x^*)}_{:=\text{Work}}
\end{align*}

This derives the \textbf{Work-Energy Theorem}, ie \[\Delta \text{KE} = W.\]

\begin{example}[Work due to gravity]
  Consider $F = \vec{g}$. We can write \[\frac{1}{2}m v_1^2 - \frac{1}{2}m v_0^2 = \int_{z_0}^{z_1} - mg \dd{z} = -mg(z_1- z_2).\] Assume we end at zero velocity $v_1 = 0$ and define our "start" as zero displacement $z_0 = 0$, then we have \[z_1 = \frac{v_0^2}{2g},\] ie, the final position has no dependence on mass.
\end{example}

\subsection{Extension to Higher Dimensions}

We wrote previously $\vec{F}(\vec{r}) = m \dv{\vec{v}}{t}$, a vector valued function with vector valued arguments. We can manipulate this into a nicer form akin to the 1-dim case as follows;
\begin{align*}
  \vec{F} \cdot \dd{\vec{r}} = m \dv{\vec{v}}{t} \cdot \dd{\vec{r}}\\
  \star \dv{\vec{r}}{t} = \vec{v} \implies \dd{\vec{r}} = \dd{t} \vec{v}\star\\
  \oint_R \vec{F} \cdot \dd{\vec{r}} = m \int_R\dv{\vec{v}}{\cancel{t}}\cdot \vec{v} \cancel{\dd{t}}
\end{align*}
Note that, we are "canceling" the differentials rather informally, but with an equivalent result as properly taking a double integral over time as well as distance. Note that, as $\vec{r}$ is some arbitrary curve/trajectory in space, we are "formally" taking a line integral, hence the $\oint$. We call the trajectory we are integrating over $R$; say it has endpoints $a,b$. We can then simplify\[
  \oint_R \vec{F} \cdot \dd{\vec{r}} = m \int_a^b \dd{\vec{v}} \vec{v} = \frac{1}{2}m \left(v_a^2 - v_b^2\right),\] where our RHS is the familiar KE. Thus, 
  \[\boxed{\oint_R \vec{F} \cdot \dd{\vec{r}} = \Delta \text{KE}= W.} \]

Generally, the LHS can't necessarily be integrated but we can make some simplifying observations. First $\dd{\vec{r}}$ is always tangent to the trajectory $\vec{R}$, as should be familiar from the definition of a derivative. We can break down the force components along a given trajectory as the parallel and perpendicular components to the trajectory at a given point; \[\vec{F} = F_{\parallel} + F_\perp \implies \vec{F} \cdot \dd{r} = (F_{\parallel} + F_\perp)\cdot \dd{r}.\] However, by definition, the dot product of a vector with a perpendicular is equal to zero, and thus $\dd{r} \cdot F_\perp = 0$, and we thus need only be concerned with the parallel components of force.

\begin{example}[Inverse Square Field]
  We have $F(r) - \frac{GmM}{r^2}$. We can write the work done on a particle to escape the field as \begin{align*}
    W &= \lim_{r \to \infty}\int_{R_E}^{r} \dd{r^*}\left(\frac{-GmM}{r{^*}^2}\right)\\
    &= \eval{\frac{GmM}{r}}_{R_E}^{\infty} = -\frac{GmM}{R_E}
  \end{align*}
  We can use this to find the escape velocity $v_e$, ie the initial velocity needed for a particle to \emph{just} escape such a field;
  \begin{align*}
    W = \Delta \text{KE} &= \cancelto{0}{\frac{1}{2}mv_f^2}-\frac{1}{2}mv_e^2\\
    &= \frac{GmM}{R_E} \implies v_e = \sqrt{2 \frac{GM}{R_E}}=\sqrt{2 g R_E}\\
  \end{align*} Recalling that $g:= \frac{GM}{R_E^2}$.
\end{example}

\begin{example}[Inverted Pendulum]
  A string of length $l$ is attached to a vertical wall with a mass $m$ at its end. Given an initial angular displacement of $\varphi_0$, we can write:
  \begin{align*}
    \vec{F} \cdot \dd{\vec{r}} &= \underbrace{\cancelto{0}{\vec{T}\cdot \dd{\vec{r}}}}_{\perp \text{ motion}} + m \vec{g} \cdot \dd{\vec{r}}\\
  \end{align*}
  The $\parallel$ component of $m\vec{g}$, the only relevant component on the RHS, is simply $m g \sin \varphi$. We can also rewrite $\dd{\vec{r}}$ to have a solvable integral by noting the relationship between arc length and radius, $s = l \varphi \implies \dd{\vec{r}} = l \dd{\varphi}$. Substituting this in:
  \begin{align*}
    \oint \dd{\vec{r}}\cdot \vec{F} &= \int_{\varphi_0}^\varphi \dd{\varphi'} m g l \sin \varphi'\\
    &= mgl \eval{(-\cos \varphi')}_{\varphi_0}^{\varphi}\\
    &= mgl(\cos \varphi_0 - \cos \varphi)
  \end{align*}
\end{example}

\begin{example}[Constant Uniform Force]
  \begin{align*}
    \vec{F} = F_0 \hat{m} \implies \oint \vec{F} \cdot \dd{\vec{r}} &= F_0 \oint \dd{\vec{r}}\cdot \hat{m}\\
    &= F_0 \cdot \hat{m} \cdot \oint_a^b \left[\hat{i} \dd{x} + \hat{j} \dd{y} + \hat{k} \dd{z}\right]\\
    &= F_0 \hat{m}\left[(x_b-x_a)\hat{i} + (y_b-y_a)\hat{j} + (z_b-z_a)\hat{k}\right]\\
    &= F_0 \hat{m}(\vb{r_b} - \vb{r_a})
  \end{align*}
\end{example}

% \begin{example}[Central Force]
%   $\vec{F} = f(r) \hat{r}$
% \end{example}

\subsection{Force Fields}

In many of the examples of the previous section, the work done over an area depending solely on the force at the starting and ending points; this begs the question, in what situations is the work done \textit{independent} of the path taken? This is equivalent to asking how can we tell if a force is \textit{conservative}?\\
Consider some path $\vec{r}$ from a point $\vec{r}_a$ to $\vec{r}_b$. If we consider $a$ and $b$ to be the same point (ie, we are in a loop), then we can write \[\oint_a^b \vec{F} \cdot \dd{\vec{r}} = f(\vec{r}_b) - f(\vec{r}_a) = 0.\] Consider the following theorem:
\begin{theorem}[Stoke's Theorem]
  \[\int_C\vec{F}\cdot \dd{\vec{r}} = \iint (\curl \vec{F}) \dd{\vec{s}}\]
  (read: the line integral of a vector field over a loop is equal to the integral of the perpendicular components of the curl of that field over the surface enclosed by the loop in question.)
\end{theorem}

We define $\text{curl} \vec{F} :=\curl \vec{F}$. From Stoke's Thm, then, if we require the LHS to be zero, then $\curl \vec{F} = 0$.

\section{Collisions \& Conservation Laws}

\subsection{2D}
% example: general
% todo: add picture
Consider two balls of mass $m_1$, $m_2$ initially traveling at velocities of $v_1$, $v_2$ respectively, in the $x$-direction solely. \textit{Conservation of momentum} states that momentum, that is, $\vec{P} = m \vec{v}$ must be conserved following a collision. Note the vector notation: this is not accidental. Momentum must be conserved vectorially, that is, direction-wise. In our standard example, we would write
\begin{align*}
[x]& \quad m_1 v_1 = m_1 v_1' \cos \theta + m_2 v_2' \cos \phi  \\
[y]& \quad 0 = m_1 v_1' \sin \theta - m_2 v_2' \sin \phi
\end{align*}
Where $\theta, \phi$ are the angles of reflection from the horizontal the balls experience respectively. Note that we have four unknowns in this case, $v_1', v_2', \theta, \phi$, but only 2 equations relating them. Under certain assumptions of lack of friction, etc, we can also assume that kinetic energy is conserved; this would then be called a \textbf{elastic collision} (think that the balls "bounce"; they don't stick together, as this would require some other force to keep them together). This allows us to derive another equation:
\[
\frac{v_1'}{v_1} = \frac{m_1}{m_1 + m_2}\left(\cos \theta \pm \sqrt{\cos^2 \theta - \left(\frac{m_1^2 - m_2^2}{m_1^2}\right)}\right)  
\]
which follows directly form the equation for kinetic energy applied to the before/after collision velocities. The square root in this equation gives us a few different cases:
\begin{enumerate}
  \item $m_1 > m_2 \implies \cos^2 \theta \geq 1 - \frac{m_2^2}{m_1^2},$ indicating that $\theta$ must be restricted to relatively small angles.
  \item $m_1 < m_2 \implies \sqrt{\cdots} \in \mathbb{R},$ that is, there is no restriction on $\theta$.
  \item $m_1 = m_2 \implies v_1' = v_1 \cos \theta = v_1' + v_2 ' \cos (\theta + \phi) \implies v_2' \cos (\theta + \phi) = 0 \implies \theta + \phi = \frac{\pi}{2}$.
\end{enumerate}

\subsection{Centre of Mass Frame}
% example: angles and collisions, etc
Often, it is inconvenient to do the analysis above, even with only two bodies, since we have to worry about constantly tracking several measurements over time. Instead, it is often helpful to consider the movement of the \textbf{centre of mass} of a system. The center of mass coordinate, generally, of $n$ bodies is calculated
\[
\vb{R}  = \frac{\sum_{i}^{n} m_i \vb{r}_i} {\sum_i^n m_i}
\]
where $\vb{P}, M$ are measurements of the entire system now, rather than individual quantities. From here, we can consider the velocity of the center of mass by computing \[
\vb{P} = M \vb{V} \implies \vb{V} = \frac{\vb{P}}{M}.
\]

Consider this the velocity of the center of mass; the other particles move at a given \textbf{lab} velocity relative to the so-called lab frame, and have another velocity relative to the moving center of mass, that is,
\[
\vb{v}_i = \vb{v}_i^{c}  + \vb{V}
\]

where we use $^c$ to denote quantities in the center of mass frame. Note, here, that masses initially at rest will appear to move at $- \vb{V}$ in the c.o.m. We can also consider other quantities in c.o.m.:
\[
m_i \vb{v}_i^c \equiv \vb{P}_1^c = m_i(\vb{v}_i - \vb{V}) = m_i (\vb{v}_i - \frac{\vb{P}}{M}) = \frac{m_1m_2}{m_1 + m_2} (\vb{v}_1 - \vb{v}_2) \equiv \underbrace{\mu}_{\textbf{reduced mass}} (\vb{v}_1 - \vb{v}_2) 
\]

\subsubsection{Angles}
Recall that in the lab frame, we had two angles to worry about, $\theta, \phi$. Now, we have only 1, and thus reduced cases to worry about; the angle that the velocity of the center of mass $\vb{V}$ travels post-collision, $\theta_c$, can be derived \[  
  \tan \theta  = \frac{\sin \theta_c}{\cos \theta_c + \frac{m_1}{m_2}};
\]
giving us the cases 
\begin{enumerate}
  \item $m_1 < m_2$: $\frac{V}{v_1^{c'}} = \frac{V}{\frac{m_2 v_1}{m_1+m_2}} = \frac{m_1}{m_2} < 1 \implies V < v_1^{c'}$, hence no restriction on $\theta$.
  \item $m_1 > m_2$: $\frac{m_1}{m_2} > 1 \implies V > v_1^{c'}$, hence we have a restriction on $\theta$ % TODO: what is it?
\end{enumerate}
% TODO: add pictures

\subsubsection{Conservation of Energy in CM}
Consider a two-body elastic collision. In the CM frame:
\begin{alignat*}{2}
  \frac{1}{2}m_1 (v_1^{c})^2 + \frac{1}{2}m_2 (v_2^c)^2 = \frac{1}{2}m_2(v_1^{c'})^2 + \frac{1}{2}m_2(v_2^{c'})^2 && \qquad \text{(Energy Conservation)}\\
  |m_1 \vb{v}_1^c| = |m_2 \vb{v}_2^c| \qquad && \text{(Magnitudes, pre-collision)}\\
  |m_1 \vb{v}_1^{c'}| = |m_2 \vb{v}_2^{c'}| \qquad && \text{(Magnitudes, post-collisions)}
\end{alignat*}

Now, we can rewrite this in terms of the original velocities given the conservation of energy in the lab frame:
\begin{align*}
  K^{\text{lab}} &= \frac{1}{2}m_1 v_1^2 + \frac{1}{2}m_2 v_2^2\\
  &= \underbrace{\frac{1}{2}m_1 (v_1^c)^2 + \frac{1}{2} (v_2^c)^2}_{K^{\text{cm}}} + \underbrace{\frac{1}{2}(m_1 + m_2)(V)^2}_{\text{"Energy" of CM moving}} + \cancelto{\vb{P}^c = 0}{\frac{1}{2}(m_1 v_1^c + m_2 v_2^c)V}\\
  \implies K^\text{lab} = K^\text{cm} + \frac{1}{2}\bf{M} \vb{V}^2
\end{align*}

That is, we can consider that there is always some energy, $\frac{1}{2}\bf{M}\vb{V}^2$, which we note is constant, that is \textit{locked} in the motion of the center of mass.

% TODO: examples

\section{Rotational Motion}

\subsection{Angular Momentum}

\begin{definition}[Angular Momentum]
  The \emph{angular momentum} of a point mass is defined by \[
    \vb{L} = \vb{r} \times \vb{p},  
  \]
  where $\vb{r}$ is the position of the vector to a given origin and $\vb{p}$ is the momentum of the mass. Equivalently, \[
  |\vb{L}|  = |\vb{r}||\vb{p}|\sin \theta,
  \]
  where $\theta$ is the angle between the $\vb{r}$ and $\vb{p}$ vectors.
\end{definition}

% example: conical pendulum

\subsection{Torque}

\begin{definition}[Torque]
  \emph{Torque} is defined on a point mass \[
    \vb{\tau} = \vb{r} \times \vb{F},
    \]
    that is, a force $\vb{F}$ acts on a particle at a distance $\vb{r}$.
\end{definition}


Make note of the following:
\begin{align*}
  \vb{L} &= \vb{r} \times \vb{p}\\
  \implies \dv{\vb{L}}{t} &= \dv{t}\left(\vb{r} \times \vb{p}\right)\\
  &= \dv{\vb{r}}{t} \times \vb{p} + \vb{r} \times \dv{\vb{p}}{t}\\
  &= \cancel{\vb{v} \times m \vb{v}} + \vb{r} \times \vb{F}\\
  &= \vb{r} \times \vb{F} \equiv \vb{\tau}
\end{align*}
That is, the time derivative of angular momentum is torque.

% TODO torque vs forces diagram
% TODO conical pendulum again?
\subsubsection{Central Forces}
% TODO (graD) example from hw
% TODO example from notes with projectile from rocket

\begin{definition}[Central Force]
  A \emph{central force} is one of the form \[
  \vb{F} = f(r) \hat{r},  
  \]
  that is, a function of the distance from a particular radius, which acts radially.
\end{definition}

Consider the torque done by an arbitrary central force:
\[
  \vb{\tau} = \vb{r} \times \vb{F} = \vb{r} \times f(r) \hat{r} = 0.  
\]
This implies, then, that $\dv{\vb{L}}{t} = 0$ as well, that is, angular momentum is a constant and thus we are restricted to purely planar, 2D movement.

\subsubsection{Torque on Extended Objects}
% TODO: diagram and stuff
Consider an arbitrary rigid body and consider an infinitesimal portion of its mass. We can write \begin{align*}
  \vec{\tau} = \sum_i \vec{\tau_i} = \sum_i \vec{r_i} \times m_i \vec{g}\\
  \equiv \vec{R} \times M\vec{g},
\end{align*}
where $\vec{R} = \frac{1}{M} \sum_{i} m_i \vec{r_i}$ and $M = \sum_i m_i$.

\subsection{Moments of Inertia}

Now suppose said body rotates about the $z$-axis at a constant $\omega$. We can write the angular momentum of the object as \begin{align*}
  L_z^i = \vec{r_i} m_i \times \vec{v} = r_i m_i v \cos \phi = \rho_i m_i v = \rho_i^2 m_i \omega\\
  \implies L_z = \left(\sum_i m_i \rho_i^2\right)\omega \equiv I_z \omega,
\end{align*}
where we define, more precisely, the moment of inertia of the body about the $z$ axis as \[
I_z = \sum_i m_i \rho_i^2 = \int \eta \dd{V} (x^2 + y^2),  
\]
noting that the final equal defines $\eta$ as the density of the object, $\eta = \frac{\dd{m}}{\dd{V}}$. Note too that this final equation was rewritten explicitly in cartesian coordinates which may not always be the most convenient coordinate system. 

Abstractly, the moment of inertia of a body can be thought of as a rotational analog of mass, or its ability to "resist" rotations by an external torque.

\begin{example}[Moment of Inertia of a Thin Hoop]
  As with most of these computations, we must rewrite $\dd{V}$ in the integral in appropriate coordinates; cylindrical work best here, and we have \[\dd{V} = \rho \dd{\rho} \dd{\varphi} \dd{z}.\]  
  Taking the density to be some constant $\eta$, the radius of the hoop to be $R$, the "radial width" to be $\Delta R$, and the thickness to be $t$, we can write \begin{align*}
    I_z = \int \eta \dd{V} (x^2 + y^2) = \eta \int_R^{R + \Delta R} \dd{\rho} \rho^3 \int_{0}^t \dd{z} \int_{0}^{2\pi} \dd{\varphi}\\
    = \eta \left[\frac{(R + \Delta R)^4 - R^4}{4}\right]\cdot t \cdot 2 \pi
  \end{align*}
 Substituting in $\eta = \frac{M}{V}$, and taking $\Delta R$ and $t$ to be negligible in relation to $R$, this simplifies to \[
 \boxed{I_z = MR^2}.
 \]
\end{example}
\begin{remark}
  This general method of find moments of inertia is widely applicable; it can be generalized:
  \begin{enumerate}
    \item Write $\dd{V}$ in context-appropriate coordinates.
    \item Write $(x^2 + y^2)$ (or whatever your notion of "distance" is in your context) in your new coordinates.
    \item Integrate; if $\eta$ constant, you can substitute and simplify afterwards using $\eta = \frac{M}{V}$. If $\eta$ not-constant, you will have to make sure you also convert it to your chose coordinate system before evaluating.
  \end{enumerate}
\end{remark}
% some examples of calculations, general approachj
\begin{theorem}[Parallel Axis Theorem]
  Given the moment of inertia of a body about its center $I_c$ and about, say, the $z$-axis, the moment of inertia of the body about an axis $z'$, such that $z'$ parallel to $z$ and $z'$ is a distance $h$ from $z$, is given by \[
    I' = I_c + Mh^2.
  \]
\end{theorem}

\begin{theorem}[Perpendicular Axis Theorem]
  
\end{theorem}

% add example of how to do both

\subsection{Conservation of Angular Momentum}

% lots of examples

\subsection{Rotational \& Translational Motion}

\subsection{Slipping \& Rolling}
% examples

\section{Lagrangian Mechanics}

The previous sections have all taken a \textbf{Newtonian} approach to mechanics, centered on laws relating forces in a system. This section will "derive" and review a different perspective on mechanics, called \textbf{Lagrangian Mechanics}, centered on the concept of minimization of energies; forces "no longer exist".
\subsection{Introduction to Calculus of Variations}


\section{More on Angular Motion - Generalizations}

\section{Central Forces}

\subsection{Conic Sections}
\subsection{Kepler's Laws}
\end{document}