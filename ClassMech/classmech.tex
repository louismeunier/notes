\documentclass[12pt]{article}
\usepackage{amsthm}
\usepackage{libertine}
\usepackage[margin=0.15in]{geometry}
\usepackage{amsmath,amssymb}
\usepackage{multicol}
\usepackage[shortlabels]{enumitem}
\usepackage{siunitx}
\usepackage{setspace}
\usepackage{cancel}
\usepackage{graphicx}
\usepackage{pgfplots}
\usepackage{listings}
\usepackage{tabularx}
\usepackage{titlesec}
\usepackage{thmtools}
\usepackage{thm-restate}
\usepackage[bottom]{footmisc}
\usepackage{xcolor-solarized}
\usepackage{svg}
\usepackage{physics}
\usepackage[colorlinks=true, linkcolor=darkgray]{hyperref}
\usepackage{tikz}
\usepackage{quiver}
\usetikzlibrary{cd}
% \usepackage{cleveref}
\usepackage[]{csquotes}
\usepackage{hyperref}

\usepackage[createShortEnv]{proof-at-the-end}

% makes theorems, definitions, etc. "restatable" as shown
% can add more with same format as you wish
\renewcommand*{\proofname}{}


\declaretheorem[
  thmbox=S,
  name=Definition,
  refname={Definition, definition}, numberwithin=section
]{definition}

\declaretheorem[
  thmbox=S,
  name=Axiom,
  refname={Axiom, axiom},
  numberwithin=section
]{axiom}

\declaretheorem[
  thmbox=S,
  name=Lemma,
  refname={Lemma, lemma},
  numberwithin=section
]{lemma}

\declaretheorem[
  thmbox=S,
  name=Corollary,
  refname={Corollary, corollary},
  numberwithin=section
]{corollary}

\declaretheorem[
  name=Exercise, 
  refname={Exercise, exercise},
  numberwithin=section,
  shaded={rulecolor=solarized-cyan, rulewidth=1pt, bgcolor={rgb}{1,1,1}}
]{exercise}

\declaretheorem[
  thmbox=S,
  name=Theorem,
  refname={Theorem, theorem},
  numberwithin=section
]{theorem}

\declaretheorem[
  % thmbox=M,
  name=Example,
  refname={Example, example},
  numberwithin=section,
  shaded={rulecolor=solarized-cyan, rulewidth=1pt, bgcolor={rgb}{1,1,1}}
]{example}

\newEndThm[no proof here, restate, text proof={Examples}, one big link={\emph{See more}}]{definitionEnd}{definition}
\newEndThm[no proof here, restate, one big link = {\emph{Proof}}]{lemmaEnd}{lemma}
\newEndThm[no proof here, restate, one big link = {\emph{(Solution)}}]{exerciseEnd}{exercise}
\newEndThm[no proof here, restate]{axiomEnd}{axiom}

% \newEndThm[proof here, restate]{theoremEnd}{theorem}

% makes "quoted" text actually look correct
\MakeOuterQuote{"}

% page footer
\newpagestyle{mypage}{%
    \footrule
    \setfoot{\small\textcolor{gray}{§\thesubsection}}{\small\textcolor{gray}{\textit{\sectiontitle: \textbf{\subsectiontitle}}}}{\textcolor{gray}{\small p. \thepage}}
}

% title page settings
\newcommand{\pageauthor}{Louis Meunier}
\newcommand{\pagetitle}{Classical Mechanics}
\newcommand{\pagesubtitle}{PHYS251}

% black square for qed symbol
\renewcommand{\qedsymbol}{$\blacksquare$}

% annoying urls
\urlstyle{same}
\urldef\calc\url{https://notes.louismeunier.net/Calculus%20A%2C%20B/calculus.pdf#page=85 }

\titleformat{\section}
{\centering\normalfont\Large\bfseries}
{\thesection}{1em}{}

\begin{document}
\setstretch{2.25}
\noindent
\begin{center}
    \begin{tabularx}{\textwidth} { 
        >{\raggedright\arraybackslash}X 
        >{\raggedleft\arraybackslash}X}
    \LARGE \pageauthor \\
    \LARGE \textbf{\pagetitle} & \LARGE \textbf{\pagesubtitle}\\
    \end{tabularx}\\
    \rule[2ex]{0.8\textwidth}{1pt}
\end{center}

\setstretch{1.5}
\tableofcontents

% "enables" footer with section+subsection, etc. just comment it out if you don't want it
\pagestyle{mypage}

% makes sections a very dark gray + centered
\titleformat{\section}
{\color{darkgray}\centering\normalfont\Large\bfseries}
{\color{darkgray}\thesection}{1em}{}

% need to change margins and such here for rest of document
% kind of messy but what can you do
\newpage
% modify these as you wish
\newgeometry{margin=0.5in, top=0.4in, bottom=0.75in}
\parskip=0.5em

\section{Introduction \& Notations}
\subsection{Vectors}
This course deals mainly with \textit{scalars} (magnitude) and \textit{vectors} (magnitude and direction). We define algebra on vectors, briefly:
\begin{itemize}
  \item $\vb{A} + \vb{B} = \vb{B} + \vb{A}$ \textit{(commutativity of addition)}
  \item $\vb{A} + (\vb{B} + \vb{C}) = (\vb{A} + \vb{B}) + \vb{C}$ \textit{(associativity of addition)}
  \item $c(d\vb{A}) = (cd)\vb{A}$ \textit{(associativity of scalar multiplication)}
  \item $(c+d)\vb{A} = c \vb{A} + d\vb{A}$ \textit{(Distributivity of scalar multiplication)}
  \item $c(\vb{A} + \vb{B}) = c\vb{A} + c\vb{B}$ \textit{(Distributivity of scalar multiplication)}
\end{itemize}
and the operators:
\begin{itemize}
  \item $\vb{A} \times \vb{B} = \vb{C}$ s.t. $|\vb{C}| = |\vb{A}||\vb{B}|\sin \theta$, where $\theta$ is the angle between $\vb{A}$ and $\vb{B}$, and $\vb{C}$ is perpendicular to both $\vb{A}$ and $\vb{B}$. This is equivalent to computing $\det \begin{bmatrix} \vb{i} & \vb{j} & \vb{k} \\ A_x & A_y & A_z \\ B_x & B_y & B_z \end{bmatrix}$.  This \textit{cross product} is \textbf{anti-commutative}, meaning $\vb{A} \times \vb{B} = -\vb{B} \times \vb{A}$. Additionally, note that $\vb{A} \times \vb{A} = 0$.
  \item $\vb{A}\cdot \vb{B} = C = |\vb{A}||\vb{B}|\cos \theta$, where $C$ is a scalar. Note that $C = 0$ when $\theta = \pi/2$, ie $\vb{A}$ and $\vb{B}$ are perpendicular.
\end{itemize}

\subsection{Law of Cosines}

Consider a (planar) triangle constructed of sides $\vb{C} = \vb{A} + \vb{B}.$ We can write \begin{align*}
  \vb{C}^2 &= \vb{C}_x^2 + \vb{C}_y^2\\
  &= (|\vb{A}| - |\vb{B}|\cos \theta)^2 + (|\vb{B}|\sin \theta)^2\\
  &= |\vb{A}|^2 + |\vb{B}|^2 - 2|\vb{A}||\vb{B}|\cos \theta
\end{align*}


\subsection{Perspectives on the Cross Product}

\begin{align*}
  \vec{A} \cross \vec{B} &= (A_x \vb{i} + A_y \vb{j} + A_z \vb{k}) \cross (B_x \vb{i} + B_y \vb{j} + B_z \vb{k})\\
  &= A_x B_y (\vb{i} \cross \vb{j}) + \cdots \\
  &= (A_y B_z - A_z B_y) \vb{i} + \cdots \\
  &\equiv (\vb{A}\times\vb{B})_k = \sum_{i=1}^{3}\sum{j=1}^3\mathcal{E}_{ijk} A_i B_j
\end{align*}
Where $\mathcal{E}_{ijk} = \begin{cases}
  1; & $ijk$ \text{ even permutation of }123\\
  -1; & $ijk$ \text{ odd permutation of }123\\
  0; & \text{otherwise}
\end{cases}$ 

\subsection{Describing a Particle in Space in Polar Coordinates}

Consider a particle moving through space with a constant angular velocity $\dv{\theta}{t} = \omega$. We can describe this movement in terms of planar coordinates as $\vb{r}(t) = r_0 \cos (\omega t) \vb{i} + r_0 \sin(\omega t) \vb{j}$. Differentiating with respect to time, we obtain $\vb{v}(t) = -r_0 \omega (\sin(\omega t)\vb{i} - \cos (\omega t) \vb{j})$. Notice that $\vb{r} \cdot \vb{v} = 0 \forall t$; this should be familiar, as the velocity vector is always perpendicular to the position vector in purely circular motion. Differentiating again, we obtain $\vb{a}(t) = -r_0 \omega^2 (\cos(\omega t)\vb{i} + \sin (\omega t) \vb{j}) = -\omega^2 \vb{r}(t)$. In other words, the acceleration is always opposing the position vector (given the negative sign), and is proportional to the square of the angular velocity.

Assume now, instead, that the particle moves arbitrarily, described by a function $\vb{r}(t)$. In polar coordinates, this position vector is always travelling along the vector $\vectorunit{r}$, with magnitude $r$, and we can write $\vb{r}(t) = r \cdot \vectorunit{r}$. Differentiating:
\begin{align*}
  \vb{v}(t) = \dv{\vb{r}}{t} &= \dv{t}\left(r\cdot\vectorunit{r}\right)\\
  &= \dv{r}{t}\vectorunit{r} + r\dv{\vectorunit{r}}{t}\\
  &= \dot{r}\vectorunit{r} + r \dv{t}\left(\cos \theta \vb{i} + \sin \theta \vb{j}\right)\\
  &= \dot{r}\vectorunit{r} + r \left(-\sin \theta \vb{i} + \cos \theta \vb{j}\right)\dot{\theta}\\
  &= \dot{r}\vectorunit{r} + r \dot{\theta} \vectorunit{\theta}
\end{align*}

Recalling that $\begin{cases}
  \vectorunit{r} = \cos \theta \vb{i} + \sin \theta \vb{j}\\
  \vectorunit{\theta} = -\sin \theta \vb{i} + \cos \theta \vb{j}
\end{cases}$.
Differentiating again:
\begin{align*}
  \vb{a}(t) = \dv{\vb{v}}{t} &= \dv{t}\left(\dot{r}\vectorunit{r} + r \dot{\theta} \vectorunit{\theta}\right)\\
  &= \ddot{r}\vectorunit{r} + \dot{r}\dot{\theta}\vectorunit{\theta} + \dot{r}\dot{\theta}\vectorunit{\theta} + r \ddot{\theta}\vectorunit{\theta} + r \dot{\theta}(-\vectorunit{r})\\
  &= \left(\underbrace{\ddot{r}}_{\text{radial}} - \underbrace{r \dot{\theta}^2}_{\text{centripetal}}\right)\vectorunit{r} + \left(\underbrace{2\dot{r}\dot{\theta}}_{\text{coriolis}} + \underbrace{r \ddot{\theta}}_{\text{tangential}}\right)\vectorunit{\theta}
\end{align*}

we obtain a decomposition of the acceleration vector into radial ($\vectorunit{r}$) and tangential ($\vectorunit{\theta}$) components, labelled accordingly. Note that the centripetal acceleration, labelled, is the same as the acceleration in circular motion mentioned previously.\footnote{None of the other "components" of acceleration are present in the constant angular velocity case because (1) $\ddot{r}=0$, ie no radial acceleration, so the radial and coriolis components are zero, and (2) $\dot{\theta} = \omega \implies \ddot{\theta} = 0$, so the tangential acceleration is zero, leaving only the centripetal acceleration.}\footnotemark

\footnotetext{See \calc for a different perspective on this topic.}

% TODO: update prev
% statics, etc

\subsection{Newton's Laws of Motion}
\begin{enumerate}
  \item[N1:] \textit{"in absence of external force, a body at rest remains at rest, and a body in motion remains in motion, with the same speed \& same direction"}
  
This law defines \textit{intertial frames}, ie ones in which the law holds.

  \item[N2:] $\vec{F} = m \vec{a}$
  \item[N3:] \textit{"if a body \textbf{b} applies a force on body \textbf{a}, then \textbf{a} applies a force on \textbf{b} such that $\vec{F}_a = -\vec{F}_b$ (equal and opposite)"}\footnote{$\implies$ conservation of momentum \dots}
\end{enumerate}

\begin{example}
  Consider two blocks \textbf{A}, \textbf{B} where \textbf{A} lies atop \textbf{B} which all lie upon "the earth".

  \textbf{A} experiences the force $W_A$ due to gravity and the force $F_1$ back from \textbf{B}.

  \textbf{B} experiences the normal force $N$ from the table, the force of gravity $W_B$, and finally the force $F_2$ due to \textbf{A}.

  We can write (from N2) \[\vec{F}_1 + \vec{W}_A = m_A \vec{a}_A\] and similarly
  \[\vec{N} + \vec{W}_B + \vec{F}_2 = m_B \vec{a}_B\]
  Static situation $\implies \vec{a}_A = \vec{a}_B = 0$.
  We can further simplify writing $F_1 = m_A g$ (N2), and finally \[N = W_A + W_B = (m_A + m_B)g.\]
\end{example}

\begin{example}
  Consider a mass $m_1$ laying on a table, connected via a string to a mass $m_2$ hanging off the table, where the string is of small mass and is under a tension $T$. (Looking at a functionally massless stirng, we can consider there to be a constant tension as we can say $T - T' = \delta m a \implies T - T' = 0 \implies T = T'$, where $T, T'$ is the tension on a particular segment of the string in opposing directions).

  On $m_1$, we have the normal force $N$, tension $T$, and weight $m_1 \cdot g$.

  On $m_2$, we have the weight $m_2 \cdot g$, the tension $T$ (equal throughout string as explained above).

  Together, we can write (N2; taking $x$ to represent a motion "left" and $z$ to represent a motion "down") 
  \[- T = m_1 \ddot{x}\] 
  and 
  \[m_2g-T = m_2 \ddot{z}.\]

  Noting that the string must "retain" its length we can write \[l(t) = x(t) + z(t),\] where $x$ and $z$ represent the length of the string in the $x/z$ axes. However, as $l$ must stay constant, we can differentiate twice wrt time to obtain \[0 = \ddot{x} + \ddot{z}.\]
  All together, then we have
  \[
    \ddot{x} = \frac{-m_2 g}{m_1 + m_2}.
  \]
  Integrating twice, we have \[
    x(t) = \frac{-m_2 g}{m_1 + m_2} \cdot \frac{t^2}{2}.
  \]
\end{example}

\begin{example}
  Consider a mass $m_1$ connected to a string which lies along a pulley $P_1$, which then attaches to the center of pulley $P_2$, which has a string whicch, on one end, is attached to a mass $m_2$, and is grounded in the other.

  On $m_1$, we have tension $T$ and the weight $m_1 \cdot g$.
  
  On $m_2$, we have the weight $m_2 \cdot g$ and the tension $T'$ (no reason for $T = T'$; strings aren't connected).

  Consider $P_1$ - it is nailed to the wall at its center, and experiences some $F$ ("up") from the nail, as well as the tension $T_1$ down (\textit{twice}). 

  On $P_2$, we have the tension $T_1$ ("up") and the tension $T_2$ ("down", "twice").

  We can write \begin{align*}
    T_1 - W_1 &= m_1 a_1\\
    T_2 - W_2 &= m_2 a_2\\
    T_1 - 2T_2 &= m_{P_2} \cancel{a_{P_2}}\implies T_1 = 2 T_2
  \end{align*}
  As last time, we can use the fact that the strings cannot stretch. Consider $m_1$ to be at height $y_1$, $P_1$ to be at height $y_{P_1}$, $P_2$ to be at height $y_{P_2}$, and $m_2$ to be at height $y_{2}$.

  We can then take the length $l_1$ of the string about $P_1$ as
  \[
  l_1 = (y_{P_1} - y_1) + (y_{P_2}-y_2) + \pi R_{P_1} 
  \]
  Differentiating twice wrt time, we have
  \[
  0 = a_1 + a_{P_2} \implies a_1 = - a_{P_2}.
  \]
  Analyzing $P_2$ similarly, we will obtain
  \[
    a_2 = 2 a_{P_2}.
  \]

  All together, we have 
  \[
  a_1 = \left(\frac{2m_2 - m_1}{4m_2 + m_1}\right) g.
  \]
\end{example}

Consider a collection of particles. The force on any particular particle, say 1, we can denote \[\vec{F}_{1} = \vec{F}_{12} + \vec{F}_{13} + \dots + \vec{F}_{1n} + \vec{F}^{\text{ext}}_1.\] Notice that any force $\vec{F}_{nm} = - \vec{F}_{mn}$, by N3. Thus, if we add all the forces on all the particles in the bag, we will always have a "pairing" of forces such that each $\vec{F}_{mn}$ is "canceled" by another, leaving behind only the external forces, ie \[\vec{F}^{\text{ext}} = \sum_{i} \vec{F}^{\text{ext}}_{i}.\] Say there are no external forces; then, we have 
\begin{align*}
  \sum_i m_i \vec{a}_i &= 0\\
  \dv{t} \sum_i m_i \vec{v}_i &= 0.\\
\end{align*}
And thus, we have conservation of momentum.

\subsection{Projectile Motion}
Say we have a particle with an initial velocity $v_0$ launched at an angle $\theta$. This particle experiences just one force, $mg$, and by N2 we have \[\cancel{m} \ddot{y} = -\cancel{m}g.\] Integrating, we have \[y(t) = y(t=0) + \underbrace{v_0 \sin \theta}_{\text{vertical component of $v_0$}} t - \frac{g t^2}{2}.\]

In the $x$, we have $m \ddot{x} = 0$, $x(t) = v_0 \cos \theta t$. From here, we can rewrite $x(t)$ as $t$ as a function of $x$ and substitute into $y$ for a function $y = f(x)$. This gives \[
y = \frac{v_0 \sin \theta x}{v_0 \cos \theta} - \frac{g}{2} \frac{x^2}{v_0^2 \cos^2 \theta} (= ax - bx^2)  
\]
\end{document}