

\prAtEndRestateii*

\prAtEndRestateiii*

\makeatletter\Hy@SaveLastskip\label{proofsection:prAtEndiii}\ifdefined\pratend@current@sectionlike@label\immediate\write\@auxout{\string\gdef\string\pratend@section@for@proofiii{\pratend@current@sectionlike@label}}\fi\Hy@RestoreLastskip\makeatother\begin{proof}[Proof of \pratendRef{thm:prAtEndiii}]\phantomsection\label{proof:prAtEndiii}Follows a similar structure to the previous example. Let $S$ be the subse4t of $\mathbb {N}$ for which the statement holds. $1 \in S$ by inspection (\textit {(a)} holds), and we prove \textit {(b)} by assuming $n \in S$ and showing $n+1 \in S$ (algebraically). Thus, by AI.i, $S = \mathbb {N}$ and the statement holds $\forall n \in \mathbb {N}$. \qed \end{proof}

\prAtEndRestateiv*

\prAtEndRestatev*

\makeatletter\Hy@SaveLastskip\label{proofsection:prAtEndv}\ifdefined\pratend@current@sectionlike@label\immediate\write\@auxout{\string\gdef\string\pratend@section@for@proofv{\pratend@current@sectionlike@label}}\fi\Hy@RestoreLastskip\makeatother\begin{proof}[Proof of \pratendRef{thm:prAtEndv}]\phantomsection\label{proof:prAtEndv}Again, very similar to the previous induction examples. Take $S$ to be the subset of $\mathbb {N}$ for which the statement holds. \textit {(a)} of AI.ii holds by inspection (where $m = 2$), and \textit {(b)} holds by assuming $n \in S$ and showing that $n+1 \in S$. Thus, $S = \{2, 3, 4, \dots \}$, and the statement holds $\forall n \geq 2$.\end{proof}

\prAtEndRestatevi*

\prAtEndRestatevii*

\prAtEndRestateviii*
