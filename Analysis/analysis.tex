\documentclass[12pt]{article}
\usepackage{amsthm}
\usepackage{libertine}
\usepackage[margin=0.15in]{geometry}
\usepackage{amsmath,amssymb}
\usepackage{multicol}
\usepackage[shortlabels]{enumitem}
\usepackage{siunitx}
\usepackage{setspace}
\usepackage{cancel}
\usepackage{graphicx}
\usepackage{pgfplots}
\usepackage{listings}
\usepackage{tabularx}
\usepackage{titlesec}
\usepackage{thmtools}
\usepackage{thm-restate}
\usepackage[side]{footmisc}
\usepackage[colorlinks=true, linkcolor=darkgray]{hyperref}
\usepackage{cleveref}
\usepackage[]{csquotes}
\usepackage{xcolor-solarized}
\usepackage{shortcuts}
\usepackage{mdframed}
% \usepackage[createShortEnv]{proof-at-the-end}

\renewcommand*{\proofname}{}


\declaretheorem[
  % thmbox=S,
  name=Definition,
  refname={Definition, definition}, numberwithin=section,
  shaded={rulecolor=solarized-blue, rulewidth=2pt}
]{definition}

\declaretheorem[
  % thmbox=S,
  name=Axiom,
  refname={Axiom, axiom},
  numberwithin=section,
  shaded={rulecolor=solarized-orange, rulewidth=2pt}
]{axiom}

\declaretheorem[
  % thmbox=S,
  name=Lemma,
  refname={Lemma, lemma},
  numberwithin=section,
  shaded={rulecolor=solarized-orange, rulewidth=1pt, bgcolor={rgb}{1,1,1}}
]{lemma}

\declaretheorem[
  % thmbox=S,
  name=Corollary,
  refname={Corollary, corollary},
  numberwithin=section,
  shaded={rulecolor=solarized-orange, rulewidth=1pt, bgcolor={rgb}{1,1,1}}
]{corollary}

\declaretheorem[
  % thmbox=S,
  name=Remark,
  refname={Remark, remark},
  numberwithin=section
]{remark}

\declaretheorem[
  % thmbox=S,
  name=Theorem,
  refname={Theorem, theorem},
  numberwithin=section,
  shaded={rulecolor=solarized-red, rulewidth=1pt}
]{theorem}

\declaretheorem[
  % thmbox=M,
  name=Example,
  refname={Example, example},
  numberwithin=section,
  shaded={rulecolor=solarized-cyan, rulewidth=1pt, bgcolor={rgb}{1,1,1}}
]{example}

\declaretheorem[
  % thmbox=S,
  name=Proposition,
  refname={Proposition, proposition},
  numberwithin=section,
  shaded={rulecolor=solarized-magenta, rulewidth=1pt, bgcolor={rgb}{1,1,1}}
]{proposition}



% makes "quoted" text actually look correct
\MakeOuterQuote{"}

% page footer
\newpagestyle{mypage}{%
    \footrule
    \setfoot{\small\textcolor{gray}{§\thesubsection}}{\small\textcolor{gray}{\textit{\sectiontitle: \textbf{\subsectiontitle}}}}{\textcolor{gray}{\small p. \thepage}}
}

% title page settings
\newcommand{\pageauthor}{Louis Meunier}
\newcommand{\pagetitle}{Analysis I, II}
\newcommand{\pagesubtitle}{MATH254}

% black square for qed symbol
\renewcommand{\qedsymbol}{$\blacksquare$}

\titleformat{\section}
{\centering\normalfont\Large\bfseries}
{\thesection}{1em}{}
\pgfplotsset{compat=1.18}
\begin{document}
\setstretch{2.25}
\noindent
\begin{center}
    \begin{tabularx}{\textwidth} { 
        >{\raggedright\arraybackslash}X 
        >{\raggedleft\arraybackslash}X}
    \LARGE \pageauthor \\
    \LARGE \textbf{\pagetitle} & \LARGE \textbf{\pagesubtitle}\\
    \end{tabularx}\\
    \rule[2ex]{0.8\textwidth}{1pt}
\end{center}

\setstretch{1.5}
\begin{mdframed}[backgroundcolor=gray!20]
  \underline{Course Outline:}\\\textit{
    Fundamentals of set theory. Properties of the reals. Limits, limsup, liminf. Continuity. Functions. Differentiation.
  }
\end{mdframed}

\tableofcontents

% "enables" footer with section+subsection, etc. just comment it out if you don't want it
\pagestyle{mypage}

% makes sections a very dark gray + centered
\titleformat{\section}
{\color{darkgray}\centering\normalfont\Large\bfseries}
{\color{darkgray}\thesection}{1em}{}

% \color{solarized-red}{Theorems},\color{solarized-orange}{ Axioms, Lemmas, Corollaries},\color{solarized-blue}{ Definitions}, \color{solarized-cyan}{ Examples}, \color{solarized-magenta}{ Propositions},\color{darkgray}{ Remarks}

% need to change margins and such here for rest of document
% kind of messy but what can you do
\newpage
% modify these as you wish
\newgeometry{margin=0.25in, top=0.4in, bottom=0.5in, marginparwidth=1.4in, marginparsep=0.3in, outer=0.2in, includemp}
\parskip=0.6em


\section{Logic, Sets, and Functions}
\subsection{Mathematical Induction \& The Naturals}

The \textbf{natural numbers}, $\mathbb{N} = \{1, 2, 3, \dots \}$, are specified by the 5 \textbf{Peano Axioms}:

\begin{enumerate}[label=(\arabic*)]
  \item 1 $\in$ $\mathbb{N}$
  \footnote{using 0 instead of 1 is also valid, but we will use 1 here.}
  \item every natural number has a successor in $\mathbb{N}$
  \item 1 is not the successor of any natural number
  \item if the successor of $x$ is equal to the successor of $y$, then $x$ is equal to $y$
  \footnote{axioms (2)-(4) can be equivalently stated in terms of a successor function $s(n)$ more rigorously, but won't here}
  \item \textbf{the axiom of induction}
\end{enumerate}

The \textbf{Axiom of Induction} (AI), can be stated in a number of ways. 

\begin{axiom}[AI.i]
  Let $S \subseteq \mathbb{N}$ with the properties:
  \begin{enumerate}[label=(\alph*)]
    \item $1 \in S$ 
    \item if $n \in S$, then $n+1 \in S$\footnotemark
  \end{enumerate}
  then $S = \mathbb{N}$.
\end{axiom}
\footnotetext{\textit{(a)} is called the \textbf{inductive base}; \textit{(b)} the \textbf{inductive step}. All AI restatements are equivalent in having both of these, and only differentiate on their specific values.}

\begin{example}\label{example:ai.i}
  Prove that, for every $n \in \mathbb{N}$, $1 + 2 + \cdots + n = \frac{n(n+1)}{2} (\equiv (1))$
  \begin{proof}[Proof (via AI.i)]
    Let $S$ be the subset of $\mathbb{N}$ for which $(1)$ holds; thus, our goal is to show $S = \mathbb{N}$, and we must prove \textit{(a)} and \text{(b)} of AI.i.
    \begin{itemize}
      \item by inspection, $1 \in S$ since $1 = \frac{1(1+1)}{2} = 1$, proving \textit{(a)}
      \item assume $n \in S$; then, $1 + 2 + \cdots + n = \frac{n(n+1)}{2}$ by definition of $S$. Adding $n+1$ to both sides yields:
      \begin{align}
        1 + 2 + \cdots + n + (n+1) &= \frac{n(n+1)}{2} + (n+1) \\
        &= (n+1)(\frac{n}{2}+1) \\
        &= \frac{(n+1)(n+2)}{2} \\
        &= \frac{(n+1)((n+1)+1)}{2}
      \end{align}
    \end{itemize}
    Line (4) is equivalent to statement (1) (substituting $n$ for $n+1$), and thus if $n \in S$, then $n+1 \in S$ and \textit{(b)} holds. Thus, by AI.i, $S = \mathbb{N}$ and $1 + 2 + \cdots + n = \frac{n(n+1)}{2}$ holds $\forall n \in \mathbb{N}$.
  \end{proof}
\end{example}

\begin{example}
  Prove (by induction), that for every $n \in \mathbb{N}$, $1^3 + 2^3 + \cdots + n^3 = \left[\frac{n(n+1)}{2}\right]^2$.
  \begin{proof}[Proof]
    Follows a similar structure to the previous example. Let $S$ be the subset of $\mathbb{N}$ for which the statement holds. $1 \in S$ by inspection (\textit{(a)} holds), and we prove \textit{(b)} by assuming $n \in S$ and showing $n+1 \in S$ (algebraically). Thus, by AI.i, $S = \mathbb{N}$ and the statement holds $\forall n \in \mathbb{N}$.
  \end{proof}
  This can also be proven directly (Gauss' method).
\begin{proof}[Proof (Gauss' method)]
  Let $A(n) = 1 + 2 + 3 + \cdots + n$. We can write $2 \cdot A(n) = 1 + 2 + 3 + \cdots + n + 1 + 2 + 3 + \cdots + n$. Rearranging terms ($1$ with $n$, $2$ with $n - 1$, etc.), we can say $2\cdot A(n) = (n+1)+(n+1)+\cdots$, where $(n+1)$ is repeated $n$ times; thus, $2\cdot A(n) = n(n+1)$, and $A(n) = \frac{n(n+1)}{2}$.
\end{proof}
\end{example}



\begin{axiom}[AI.ii]
  Let $S \subseteq \mathbb{N}$ s.t.
  \begin{enumerate}[label=(\alph*)]
    \item $m \in S$
    \item $n \in S \implies n+1 \in S$
  \end{enumerate}
  then $\{m, m+1, m+2, \dots\} \subseteq S$.
\end{axiom}

\begin{example}
  Using AI.ii, prove that for $n \geq 2$, $n^2 > n+1$
  \begin{proof}[Proof]
    Again, very similar to the previous induction examples. Take $S$ to be the subset of $\mathbb{N}$ for which the statement holds. \textit{(a)} of AI.ii holds by inspection (where $m = 2$), and \textit{(b)} holds by assuming $n \in S$ and showing that $n+1 \in S$. Thus, $S = \{2, 3, 4, \dots \}$, and the statement holds $\forall n \geq 2$.
  \end{proof}
\end{example}

\begin{axiom}[Principle of Complete Induction, AI.iii]
  Let $S \subseteq \mathbb{N}$ s.t.
  \begin{enumerate}[label=(\alph*)]
    \item $1 \in S$
    \item if $1, 2, \dots, n -1 \in S$, then $n \in S$
  \end{enumerate}
  then $S = \mathbb{N}$.
\end{axiom}

Finally, combing AI.ii and AI.iii;
\begin{axiom}[AI.iv]
  Let $S \subseteq \mathbb{N}$ s.t.:
  \begin{enumerate}[label=(\alph*)]
    \item $m \in S$
    \item if $m, m + 1, \dots, m + n \in S$, then $m + n + 1 \in S$
  \end{enumerate}
  then $\{m, m+1, m+2, \dots\} \subseteq S$.
\end{axiom}

% idk why it won't restate it
\begin{theorem}[Fundamental Theorem of Arithmetic]\label{thm:fta}
  Every natural number $n$ can be written as a product of one or more primes. \footnotemark
\end{theorem}
\footnotetext{1 is not a prime number}
\begin{proof}[Proof of \cref{thm:fta}]
  Let $S$ be the set of all natural numbers that can be written as a product of one or more primes. We will use AI.iv to show $S = \{2, 3, \dots \}$.
  \begin{itemize}
    \item \textit{(a)} holds; 2 is prime and thus $2 \in S$
    \item suppose that $2, 3, \dots, 2 + n \in S$. Consider $2 + (n+1)$:
    \begin{itemize}
      \item if $2 + (n+1)$ is \textit{prime}, then $2 + (n+1) \in S$, as all primes are products of $1$ and themselves and are thus in $S$ by definition.
      \item if $2 + (n+1)$ is \textit{not prime}, then it can be written as $2 + (n+1) = a \cdot b$ where $a,b \in \mathbb{N}$, and $ 1 < a < 2 + (n+1)$ and $1 < b < 2 + (n+1)$. By the definition of $S$, $a,b \in S$, and can thus be written as the product of primes. Let $a = p_1 \cdot \cdots \cdot p_l$ and $b = q_1 \cdot \cdots \cdot q_j$, where the $p$'s and $q$'s are prime and $l, j \geq 1$. Then, $a\cdot b$ is a product of primes, and thus so is $2 + (n+1)$. Thus, $2 + (n+1) \in S$, and by AI.iv, $S = \{2, 3, 4, \dots\}$
    \end{itemize}
  \end{itemize}
\end{proof}

\subsection{Extensions: Integers, Rationals, Reals}

Consider the set of naturals $\mathbb{N} = \{1, 2, 3, \dots\}$. Adding $0$ to $\mathbb{N}$ defines $\mathbb{N}_0 = \{0, 1, 2, \dots\}$. We define the \textbf{integers} as the set $\mathbb{Z} = \{\dots, -3, -2, -1, 0, 1, 2, 3, \dots\}$, or the set of all positive and negative whole numbers.

Within $\mathbb{Z}$, we can define multiplication, addition and subtraction, with the neturals of 1 and 0, respectively. However, we cannot define division, as we are not guaranteed a quotient in $\mathbb{Z}$. This necessitates the \textbf{rationals}, $\mathbb{Q}$. We define \[\mathbb{Q} = \{\frac{p}{q}, p \in \mathbb{Z}, q \in \mathbb{Z}, q \neq 0\}.\]
On $\mathbb{Q}$, we have the familiar operations of multiplication, addition, subtraction and properties of associativity, distributivity, etc. We can also define division, as $\frac{\frac{p}{q}}{\frac{p'}{q'}}= \frac{pq'}{qp'}$.

We can also define a relation $<$ between fractions, such that 
\begin{itemize}
  \item $x < y$ and $y < z \implies x < z$
  \item $x < y \implies x + z < y + z$
\end{itemize}

$\mathbb{Q}$, together with its operations and relations above, is called an \textbf{ordered field}.

\subsubsection{The Insufficiency of the Rationals}

We can consider historical reasoning for the extension of $\mathbb{Q}$ to $\mathbb{R}$. Consider a right triangle of legs $a$, $b$ and hypotenuse $c$. By the Pythagorean Theorem, $a^2 + b^2 = c^2$. Consider further the case there $a = b = 1$, and thus $c^2 = 2$. Does $c$ exist in $\mathbb{Q}$?
\begin{proposition}\label{prop:irr2}
  $c^2 = 2$, $c \notin \mathbb{Q}$.
\end{proposition} 
\begin{proof}[Proof of \cref{prop:irr2}]
  Suppose $c \in \mathbb{Q}$. We can thus write $c = \frac{p}{q}$, where\footnotemark  $p, q \in \mathbb{N}$, and $p, q$ share no common divisors, ie they are in "simplest form". Notably, $p$ and $q$ cannot \emph{both} be even (under our initial assumption), as they would then share a divisor of $2$. We write \begin{align*}
    c &= \frac{p}{q}\\
    c^2 = 2 & = \frac{p^2}{q^2}\\
    2q^2 &= p^2
  \end{align*}
  $p \in \mathbb{N} \implies p^2 \in \mathbb{N}$, and thus $p^2$, and therefore\footnotemark $p$, must be divisible by 2 ($\implies p \text{ even}$). Therefore, we can write $p = 2p_1, p_1 \in \mathbb{N}$, and thus $2q^2 = (2p_1^2)^2 \implies q^2 = 2p_1^2$. By the same reasoning, $q$ must now be even as well, contradicting our initial assumption that $p$ and $q$ share no common divisors. Thus, $c \notin \mathbb{Q}$.
\end{proof}

\footnotetext[5]{Note that in the definition of $\mathbb{Q}$, $p, q$ are defined to be in $\mathbb{Z}$; however, as we are using a geometric argument, we can assume $c>0 \implies \text{Sign}(p) = \text{Sign}(q)$, and we can just take $p, q \in \mathbb{N}$ for convenience and wlog.}

\footnotetext[6]{$\sqrt{\text{even}} = \text{even}$}

\subsection{Sets \& Set Operations}
% ! just look at algebra notes
\begin{itemize}
  \item $A \cup B = \{x : x \in A \text{ or } x \in B\}$
  \item $A \cap B = \{x : x \in A \text{ and } x \in B\}$
  \item $\bigcup_{i=1}^{\infty} A_n = \bigcup_{n \in \mathbb{N}} A_n = \{x : x \in A_n \text{ for some } n \in \mathbb{N}\}$
  \item $\bigcap_{i=1}^{\infty} A_n = \bigcap_{n \in \mathbb{N}} A_n = \{x : x \in A_n \forall n \in \mathbb{N}\}$
  \item $A^C = \{x : x\in X \text{ and } x \notin A\}$\footnote{$X$ is often omitted if it is clear from context.}
\end{itemize}

\begin{theorem}[De Morgan's Theorem(s)]\label{thm:demorgan}
  Let $A, B$ be sets. Then,
  \[(a)\hspace{1cm}(A \cap B)^C = A^C \cup B^C\]and
  \[(b)\hspace{1cm}(A \cup B)^C = A^C \cap B^C.\]
\end{theorem}
\begin{proof}[Proof of \cref{thm:demorgan}]

\begin{itemize}
  \item[(b)] \emph{(A similar argument follows\dots)}
\end{itemize}
\end{proof}
\begin{proposition}\label{prop:demorgangen}
  \begin{align*}
    (a)\, \left(\bigcap_{n=1}^\infty A_n\right)^C = \bigcup_{n=1}^\infty A_n^C \\
    (b)\, \left(\bigcup_{n=1}^\infty A_n\right)^C = \bigcap_{n=1}^\infty A_n^C 
  \end{align*}
\end{proposition}

\begin{proof}[Proof of \cref{prop:demorgangen}]
  Consider Proposition (b). Working from the left-hand side, we have
  \begin{align*}
    \left(\bigcup_{n=1}^\infty A_n\right)^C &= \{x : x \notin \bigcup A_n\}\\
    &= \{x : x\notin A_n \forall\, n \in \mathbb{N}\}\\
    &= \bigcap \{x : x \notin A_n\}\\
    &= \bigcap A_n^C
  \end{align*}
  (a) can be logically deduced from this result. Consider the RHS, $\bigcup A_n^C$. Taking the complement:
  \begin{align*}
    \left(\bigcup A_n^C\right)^C &\overset{\text{via (b)}}{=} \bigcap A_n^{C^C}\\
    &= \bigcap A_n
  \end{align*}
  Taking the complement of both sides, we have $\bigcup A_n^C = \left(\bigcap A_n\right)^C$, proving (a). 
\end{proof}

\subsection{Functions}
\begin{definition}
  Let $A, B$ be sets. A \emph{function} $f$ is a rule assigned to each $x \in A$ a corresponding unique element $f(x) \in B$. We denote \[f: A \to B.\]
\end{definition}
\begin{definition}
  The \emph{domain} of a function $f: A \to B$, denoted $\text{Dom}(f) = A$. The \emph{range} of $f$, denoted $\text{Ran}(f) = \{f(x) : x \in A\}$. Clearly, $\text{Ran}(f) \subseteq B$, though equality is not necessary.
\end{definition}
\begin{example}
  The function $f(x) = \sin x$, $f: \mathbb{R} \to [-1, 1]$. Here, $\text{Dom}(f) = \mathbb{R}$, and $\text{Ran}(f) = [-1, 1]$.
\end{example}
\begin{example}[Dirichlet Function]\footnotemark
  $f:\mathbb{R} \to \mathbb{R}$, $f(x) = \begin{cases}
    1, x \in \mathbb{Q}\\
    0, x \notin \mathbb{Q}
  \end{cases}$. Despite not having a true "explicit" formula, so to speak, this is still a valid function (under modern definitions).
\end{example}
\footnotetext{Look up a \href{https://en.wikipedia.org/wiki/Dirichlet_function}{graph} of this function. Its beautiful. It's also interesting to note that its integral is simply 0.}

\subsubsection{Properties of Functions}

\begin{proposition}
  Let $f: A \to B$, $C \subseteq A$, $f(C) = \{f(x) : x \in C\}$. We claim $f(C_1 \cup C_2) = f(C_1) \cup f(C_2)$.
\end{proposition}
\begin{proof}[Proof]
  We will prove this by showing (1) $\subseteq$ and (2) $\supseteq$.

  \begin{itemize}
    \item[(1)] $y \in f(C_1 \cup C_2) \implies \text{ for some } x \in C_1 \cup C_2, y = f(x)$. This means that either for some $x \in C_1, y = f(x)$, or for some $x \in C_2, y = f(x)$. This implies that either $y \in f(C_1)$, or $y \in f(C_2)$, and thus $y$ \textit{must} be in their union, ie $y \in C_1 \cup C_2$.
    \item[(2)] $y \in f(C_1) \cup f(C_2) \implies y \in f(C_1)$ or $y \in f(C_2)$. This means that for some $x \in C_1, y = f(x)$, or for some $x \in C_2, y = f(x)$. Thus, $x$ \textit{must} be in $C_1 \cup C_2$, and for some $x \in C_1 \cup C_2, y = f(x) \implies y \in f(C_1 \cup C_2)$.
  \end{itemize}
  (1) and (2) together imply that $f(C_1 \cup C_2) = f(C_1)\cup f(C_2)$.
\end{proof}

\begin{example}\label{example:union}
  Let $A_n = 1, 2, \dots$ be a sequence of sets. Prove that $f(\bigcup_{n=1}^{\infty} A_n) = \bigcup_{n=1}^{\infty}f(A_n)$.
  \begin{proof}[Proof]
    Let $y \in f(\bigcup_{n=1}^\infty A_n)$. This implies that $\exists x \in \bigcup_{n=1}^\infty A_n$ s.t. $f(x) = y$. This implies that $x \in A_n$ for some $n$, and $y \in f(A_n)$ for that same "some" $n$, and thus $y$ must be in the union of all possible $f(A_n)$, ie $y \in \bigcup f(A_n)$. This shows $\subseteq$, use similar logic for the reverse.
  \end{proof}
\end{example}

\begin{proposition}\label{prop:intersection}
  $f(C_1 \cap C_2) \subseteq f(C_1) \cap f(C_2)$
  \footnotemark
\end{proposition}
\footnotetext{NB: the reverse is not always true, ie these sets are not always equal; "lack" of equality is more "common" than not.}

\begin{proof}[Proof]
  $y \in f(C_1 \cap C_2) \implies$ for some $x \in C_1 \cap C_2, y = f(x)$. This implies that for some $x \in C_1, y = f(x)$ \textbf{and} for some $x \in C_2, y = f(x)$. Note that this does \textit{not} imply that these $x$'s are the same, ie this reasoning is not reversible as in the previous union case. This implies that $y \in f(C_1)$ and $y \in f(C_2) \implies y \in f(C_1) \cap f(C_2)$.
\end{proof}

\begin{example}
  Prove that if $A_n, n = 1, 2, \dots$, $f(\bigcap_{n=1}^{\infty} A_n) \subseteq \bigcap_{n=1}^{\infty}f(A_n)$.
  \begin{proof}[Proof (Sketch)]
    Use the same idea as in \cref{example:union}, but, naturally, with intersections.
  \end{proof}
\end{example}

\begin{example}
  Take $f(x) = \sin x$, $A = \mathbb{R}, B = \mathbb{R}$, and take $C_1 = [0, 2 \pi], C_2 = [2 \pi, 4 \pi]$. Then, $f(C_1) = [-1,1]$, and $f(C_2) = [-1,1]$. But $C_1 \cap C_2 = \{2 \pi\}$; $f(\{2 \pi\}) = \{\sin 2 \pi\} = \{0\}$, and thus $f(C_1 \cap C_2) = \{0\}$, while $f(C_1) \cap f(C_2) = [-1,1]$, as shown in \cref{prop:intersection}.
\end{example}

\begin{definition}[Inverse Image of a Set]
  Let $f: A \to B$ and $D \subseteq B$. The \emph{inverse image} of $D$ by $F$ is denoted $f^{-1}(D)$\footnotemark and is defined as \[f^{-1}(D) = \{x \in A : f(x) \in D\}.\]
\end{definition}
\footnotetext{Note that this is \textbf{not} equivalent to the typical definition of an inverse \textit{function}; $f^{-1}$ may not exist}
\begin{example}
  $A = [0, 2 \pi], B = \mathbb{R}, f(x) = \sin x, D = [0,1]$. 
  
  $f^{-1}(D) = \{x \in A: f(x) \in D\} = \{x \in [0, 2 \pi] : \sin(x) \in [0,1]\} = [0,\pi]$.
\end{example}

\begin{proposition}\label{prop:intersectionsimple}
  Given function $f$ and sets $D_1, D_2$,
  \begin{itemize}
    \item[(a)] $f^{-1}(D_1 \cup D_2) = f^{-1}(D_1) \cup f^{-1}(D_2)$
    \item[(b)] $f^{-1}(D_1 \cap D_2) = f^{-1}(D_1) \cap f^{-1}(D_2)\footnotemark$ 
  \end{itemize}
\end{proposition}
\footnotetext{Just see next proposition; if you really need convincing, just use $2$ rather than $\infty$ as the upper limit of the unions/intersections and use the same proof.}

\begin{proposition}
  Let $A_n, n = 1,2,3 \dots$. Then, 
  \begin{itemize}
    \item[(a)] $f^{-1}(\bigcup_{n=1}^{\infty}A_n) = \bigcup_{n=1}^{\infty}f(A_n)$
    \item[(b)] $f^{-1}(\bigcap_{n=1}^{\infty}A_n) = \bigcap_{n=1}^{\infty}f(A_n)$
  \end{itemize}
\end{proposition}

\begin{proof}[Proof]\footnotemark
  \begin{itemize}
    \item[(a)]
  \begin{align*}
    x \in f^{-1}(\bigcup_{n=1}^{\infty}A_n) &\iff f(x) \in \bigcup_{n=1}^\infty A_n\\
    & \iff f(x) \in A_n \text{ for some } n \in \mathbb{N}\\
    & \iff x \in f^{-1}(A_n) \text{ for some } n \in \mathbb{N}\\
    & \iff x \in \bigcup_{n=1}^\infty f^{-1}(A_n)
  \end{align*}
  \item[(b)]
  \begin{align*}
    x \in f^{-1}(\bigcap_{n=1}^{\infty}A_n) &\iff f(x) \in \bigcap_{n=1}^\infty A_n\\
    &\iff f(x) \in A_n \text{ for all } n \in \mathbb{N}\\
    &\iff x \in f^{-1}(A_n) \text{ for all } n \in \mathbb{N}\\
    &\iff x \in \bigcap_{n=1}^\infty f^{-1}(A_n)\footnotemark
  \end{align*}
  \end{itemize}
\end{proof}
\footnotetext{This is a "proof by definitions" as I like to call it.}
\footnotetext{Similar proof can be used to prove \cref{prop:intersectionsimple}, less generally.}

\begin{remark}
  $f: A \to B$, $A_1 \subseteq A$. Given $f(A_1^C)$ and $f(A_1)^C$, there is \textbf{no general relation} between the two.

  For instance, take $A = [0, 6 \pi], B = [-1, 2], C = [0, 2 \pi]$, and $f(x) = \sin x$. Then, $f(C) = [-1,1],$ and $f(C^C) = f([-1,0)) = [-1,1]$, but $f(C)^C = [-1,1]^C = (1,2]$, and $f(C^C) \neq f(C)^C$; in fact, these sets are disjoint.
  % VERIFY
\end{remark}

\begin{proposition}
  Let $f: A\to B$ and let $D \subseteq B$. Then $f^{-1} (D^C) = [f^{-1}(D)]^C$.
\end{proposition}
\begin{proof}[Proof]
  \begin{align*}
    f^{-1}(D^C) &= \{x: f(x) \in D^C\} = \{x : f(x) \notin D\}\\
    [f^{-1}(D)]^C &= [\{x:f(x) \in D\}]^C = \{x : x \notin f^{-1}(D)\} = \{x : f(x) \notin D\}
  \end{align*}
\end{proof}

\subsection{Reals}
\begin{axiom}[Of Completeness]\label{axiom:ac}
  Any non-empty subset of $\mathbb{R}$ that is bound from above has at least one upper bound (also called the supremum).

  In other words; let $A \subseteq \mathbb{R}$ and suppose $A$ is bounded from above ($A$ has at a least upper bound). Then $\sup(A)$ exists.
\end{axiom}

Real numbers, algebraically have the same properties as the rationals; we have addition, multiplication, inverse of non-zero real numbers, and we have the relation $<$. All together, $\mathbb{R}$ is an ordered field. 

\begin{definition}
  Let $A \subseteq \mathbb{R}$. A number $b \in \mathbb{R}$ is called an \textbf{upper bound} for $A$ if for any $x \in A$, $x \leq B$.

  A number $l \in \mathbb{R}$ is called a \textbf{lower bound} for $A$ if for any $x \in A$, $x \geq l$.
\end{definition}

\begin{definition}[The Least Upper Bound]
  Let $A \subseteq \mathbb{R}$. A real number $s$ is called the \textbf{least upper bound} for $A$ if the following holds:
  \begin{itemize}
    \item[(a)] $s$ is an upper bound for $A$
    \item[(b)] if $b$ is any other upper bound for $A$, then $s \leq b$.
  \end{itemize}

  The least upper bound of a set $A$ is \emph{unique}, if it exists; if $s$ and $s'$ are two least upper bounds, then by (a), $s$ and $s'$ are upper bound for $A$, and by (b), $s \leq s'$ and $s' \leq s$, and thus $s = s'$.

  This least upper bound is called the \emph{supremum} of $A$, denoted $\sup(A)$.
\end{definition}

\begin{definition}[The Greatest Lower Bound]
  Let $A \subset \mathbb{R}$. A number $i \in \mathbb{R}$ is called the \textbf{greatest lower bound} for $A$ if the following holds:
  \begin{itemize}
    \item[(a)] $i$ is a lower bound for $A$
    \item[(b)] if $l$ is any other lower bound for $A$, then $i \geq l$.
  \end{itemize}
  If $i$ exists, it is called the \emph{infimum} of $A$ and is denoted $i = \inf(A)$, and is unique by the same argument used for $\sup(A)$.
\end{definition}

\begin{proposition}
  Let $A \subseteq \mathbb{R}$ and let $s$ be an upper bound for $A$. Then $s = \sup(A)$ iff for any $\varepsilon>0$, there exists $x \in A$ s.t. $s- \varepsilon < x$.
\end{proposition}
\begin{proof}[Proof]
  We have two statements:
  \begin{enumerate}
    \item[I.] $s = \sup(A)$;
    \item[II.] For any $\epsilon > 0$, $\exists x \in A$  s.t. $s - \epsilon < x$;
  \end{enumerate}
  and we desire to show that I $\iff$ II.
  \begin{itemize}
    \item I $\implies$ II: Let $\epsilon > 0$. Then, since $s = \sup(A)$, $s - \epsilon$ \textit{cannot} be an upper bound for $A$ (as $s$ is the least upper bound, and thus $s - \epsilon < s$ cannot be an upper bound at all). Thus, there exists $x \in A$ such that $s - \epsilon < x$, and thus if I holds, II must hold.
    \item II $\implies$ I: suppose that this does not hold, ie II holds for an upper bound $s$ for A, but $s \ne \sup(A)$. Then, there exists some upper bound $b$ of $A$ s.t. $b < s$. Take $\epsilon = s - b$. $\epsilon > 0$, and since II holds, there exists $x \in A$ such that $s - \epsilon < x$. But since $s - \epsilon = b$ and thus $b<x$, then $b$ cannot be an upper bound for $A$, contradicting our initial condition. So, if II $\implies$ I does \textit{not} hold, we have a "impossibility", ie a value $b$ which is an upper bound for $A$ which cannot be an upper bound, and thus II $\implies$ I.
  \end{itemize}
\end{proof}

\begin{proposition}
  Let $A \subseteq \mathbb{R}$ and let $i$ be a lower bound for $A$. Then $i = \inf(A) \iff$ for every $\epsilon > 0$ there exists $x \in A$ s.t. $x < i + \epsilon$.\footnotemark
\end{proposition}
\footnotetext{Use similar argument to proof of previous proposition.}

\begin{remark}
\cref{axiom:ac} can also be expressed in terms of infimum. Define $-A = \{-x : x \in A\}$. Then, if $b$ is an upper bound for $A$, then $b \geq x \forall x \in A$, then $-b \leq -x \forall x \in A$, ie -b is a lower bound of $-A$. Similarly, if $l$ is a lower bound for $A$, $-l$ is an upper bound for $-A$.

Thus, if $A$ is bounded from above, then \[-\sup(A) = \inf(-A),\] and if $A$ is bounded from below, \[-\inf(A) = \sup(-A).\]
\end{remark}

\begin{axiom}[AC (infimum)]
  Let $A \subseteq \mathbb{R}$; if $A$ bounded from below, $\inf(A)$ exists.
\end{axiom}

\begin{definition}[$\max$, $\min$]
  Let $A \subseteq \mathbb{R}$. An $M \in A$ is called a \emph{maximum} of $A$ if for any $x \in A$, $x \leq M$. $M$ is an upper bound for $A$, \textbf{but also} $M \in A$.

  If $M$ exists, then $M = \sup(A)$; $M$ is an upper bound, and if $b$ any other upper bound, then $b \geq M$, because $M \in A$, and thus $M = \sup(A)$.

  NB: $M = \max(A)$ \textbf{need not} exist, while $\sup(A)$ must exist. Consider $A = [0,1)$; $\sup(A) = 1$, but there exists no $\max(A)$.

  The same logic exists for the existence of minimum vs infimum (consider $(0,1)$, with no maximum nor minimum).
\end{definition}

\begin{theorem}[Nested interval property of $\mathbb{R}$]\label{thm:nestedinterval}
  Let $I_n = [a_n, b_n] = \{x : a_n \leq x \leq b_n\}, n  = 1,2,3 \dots$ be an infinite sequence of bounded, closed intervals s.t. \[I_1 \supseteq I_2 \supseteq I_3 \supseteq \dots I_n \supseteq I_{n+1} \supseteq \dots \]
  Then, $\bigcap_{n=1}^\infty I_n \neq \varnothing$ (note that this does \emph{not} hold in $\mathbb{Q}$).
\end{theorem}
\begin{proof}[Proof]\footnote{Sketch: show that the left-end points are increasing and the right-end points are decreasing. Show either that all the left-end points are bounded from above or that all the right-end points are bounded from below. As a result, there exists a sup/inf (depending on which end you choose) of the set of all the right/left points. For the sup case, all upper bounds must be $\geq$ sup, and thus the sup is in all $I_n$, and thus in their intersect, and thus the intersect is not empty.}
  We have $I_n = [a_n, b_n], I_{n+1} = [a_{n+1}, b_{n+1}], \dots$. And the inclusion $I_n \supseteq I_{n+1}$. $a_n \leq a_{n+1} \leq b_{n+1} \leq b_{n}, \forall n \geq 1$. So, the sequence $a_n$ (left-end) is increasing, and the sequence $b_n$ (right-end) is decreasing. 
  
  We also have that for any $n, k \geq 1$, $a_n \leq b_k$. We see this by considering two cases:
  \begin{itemize}
    \item Case 1: $n \leq k$, then $a_n \leq a_k$ (as $a_n$ is increasing), and thus $a_n \leq a_k \leq b_k$.
    \item Case 2: $n > k$, then $a_n \leq b_n \leq b_k$ (again, as $b_n$ is decreasing). 
  \end{itemize}
  Let $A = \{a_n : n \in \mathbb{N}\}$. Then, $A$ is bounded from above by \emph{any} $b_k$ (as in our inequality we showed above). Let $x = \sup(A)$, which must exist by \cref{axiom:ac}.

  Note that as a result, \(x \geq a_n\) for all $n$, and for all $k$, \(x \leq b_k,\) as $x$ is the lowest upper bound and must be $\leq$ all other upper bounds, and so for all $n \geq 1$, $a_n \leq x \leq b_n$, ie $x \in I_n \forall n \geq 1$, and thus $x \in \bigcap_{n=1}^{\infty} I_n$ and so $\bigcap_{n=1}^{\infty} \neq \varnothing$.
\end{proof}

\begin{remark}
  The proof above emphasized the left-end points; it can equivalently be proven via the right-end points, and using $y=\inf(\{b_n : n \in \mathbb{N}\}) = \inf(B)$, rather than $\sup(A)$, and showing that $y \in \bigcap I_n$.
  % TODO
\end{remark}

\begin{remark}
  Note too that, if $x = \sup(A)$ and $y = \inf(B)$, then $x, y \in \bigcap_{n=1}^\infty I_n$; in fact, $\bigcap_{n=1}^\infty I_n = [x,y]$.
  % TODO: part 2
\end{remark}

\begin{remark}
  The intervals $I_n$ \emph{must} be closed; if not, eg $I_n = (0, \frac{1}{n})$, then $\bigcap_{n=1}^\infty I_n = \varnothing$.
\end{remark}

\subsection{Density of Rationals in Reals}

\begin{proposition}[Archimedian Property]\label{prop:archimedian}
  \begin{itemize}
    \item[(a)] For any $x \in \mathbb{R}$, there exists a natural number $n$ s.t. $n > x$.
    \item[(b)] For any $y \in \mathbb{R}$ satisfying $y > 0$, $\exists n \in \mathbb{N}$ such that $\frac{1}{n} < y$.
  \end{itemize}
\end{proposition}
\begin{remark}
  (a) states that $\mathbb{N}$ is not a bounded subset of $\mathbb{R}$.

\end{remark}
  
\begin{remark}
  (b) follows from (a) by taking $x = \frac{1}{y}$ in (a), then $\exists n\in \mathbb{N}$ s.t. $n > \frac{1}{y} \implies \frac{1}{n} < y$, and thus we need only prove (a).
\end{remark}

\begin{remark}
  Recall that $\mathbb{Q}$ is an ordered field (operations $+, \cdot$ and a relation $<$). $\mathbb{Q}$ can be extended to a larger ordered field with extended definitions of these operations/relations, such that it contains elements that are larger than any natural numbers (ie, not bounded above). This is impossible in $\mathbb{R}$ due to AC.
\end{remark}

\begin{proof}[Proof]
   Suppose (a) not true in $\mathbb{R}$, ie $\mathbb{N}$ is bounded from above in $\mathbb{R}$. Let $\alpha = \sup \mathbb{N}$, which exists by AC.

   Consider $\alpha - 1$; since $\alpha - 1 < \alpha$, $\alpha - 1$ is not an upper bound of $\mathbb{N}$. So, there exists some $n \in \mathbb{N}$ s.t. $\alpha - 1 < n$; then, $\alpha < n+ 1$ where $n+1 \in \mathbb{N}$, and thus $\alpha$ is also not an upper bound, as there exists a natural number that is greater than $\alpha$. This contradicts the assumption that $\alpha = \sup \mathbb{N}$, so (a) must be true.
\end{proof}

\begin{theorem}[Density]
  Let $a,b \in \mathbb{R}$ s.t. $a < b$. Then, $\exists x \in \mathbb{Q}$ s.t. $a < x < b$.
\end{theorem}

\begin{remark}
  If you take $a \in \mathbb{R}$ and $\epsilon > 0$, then by the theorem, $\exists x \in \mathbb{Q}$ where $x \in (a - \epsilon, a + \epsilon)$. So any real number can be approximated arbitrarily closely (via choose of $\epsilon$) by a rational number.
\end{remark}

\begin{proof}[Proof]
  Since $b - a > 0$, by (b) of \cref{prop:archimedian}, $\exists n \in \mathbb{N}$ s.t. $\frac{1}{n} < b -a$, ie $na + 1 < nb$.

  Let $m \in \mathbb{Z}$ s.t. $m - 1 \leq na < m$. Such an integer must exists since $\bigcup_{m \in \mathbb{Z}} [m-1, m) = \mathbb{R}$, the family $[m-1,m), m \in \mathbb{Z}$ makes partitions of $\mathbb{R}$. Then, $na < m$ gives that $a < \frac{m}{n}$. On the other hand, $m -1 \leq na$ gives $m \leq na+1 < nb$. So $\frac{m}{n} < b$ and it follows that $\frac{m}{n}$ satisfies $a < \frac{m}{n} < b$.
\end{proof}

In the proof, we used the claim:

\begin{proposition}
  If $z \in \mathbb{R}$, then there exists $m \in \mathbb{Z}$ s.t. $m - 1 \leq z < m$.
\end{proposition}

\begin{proof}[Proof]
  Let $S$ be a non-empty subset of $\mathbb{N}$. Then $S$ has the least element; $\exists m \in S$ s.t. $m \leq n, \forall n \in S$.

  We can assume $z \geq 0$; if $0 \leq z < 1$, then we are done (take $m = 1$), and assume that $z \geq 1$. Let now $S = \{n \in \mathbb{N} : z < n\}$, $\neq \varnothing$ by \cref{prop:archimedian}, (a). Let $m$ be the least element of $S$. It exists by Well-Ordering Property; then, since $m \in S$, $z < m$. But, we also have $m - 1 \leq z$, otherwise, if $z < m-1$ then $m - 1 \in S$ and then $m$ is not the least element of $S$. Thus, we have $m -1 \leq z < m$, as required.
\end{proof}

\begin{theorem}
  The set $J$ of irrationals is also dense in $\mathbb{R}$. That is, if $a,b \in \mathbb{R}, a < b$, $\exists$ irrational $y$ s.t. $a < y < b$ (noting that $J = \mathbb{R} \setminus \mathbb{Q}$).
\end{theorem}

\begin{proof}[Proof]
  Fix $y_0 \in \mathbb{J}$. Consider $a - y_0$, $b - y_0$. $a - y_0 < b - y_0$, and by density of rationals, $\exists x \in \mathbb{Q}$ s.t. $a - y_0 < x < b - y_0$. Then, $a < y_0 + x < b$; let $y = x+y_0$, and we have $a < y < b$.

  Note that $y$ cannot be rational; if $y \in \mathbb{Q}$, $y = x + y_0 \implies y - x = y_0$, and since $x \in \mathbb{Q}$, $y - x \in \mathbb{Q} \implies y_0 \in \mathbb{Q}$, contradicting the original choice of $y_0 \notin \mathbb{Q}$. Thus, $y \in J$.
\end{proof}

\begin{theorem}
  $\exists$ a unique positive real number $\alpha$ s.t. $\alpha^2 = 2$.
\end{theorem}

\begin{proof}[Proof]

  \begin{itemize}
    We show both uniqueness, existence:\footnotemark
    \item Uniqueness: if $\alpha^2 = 2$ and $\beta^2 = 2$, $\alpha \geq 0, \beta \geq 0$, then $0= \alpha^2 - \beta^2 = (\alpha - \beta)(\alpha + \beta) > 0$, and so $\alpha - \beta = 0 \implies \alpha = \beta$.
    \item Existence: consider the set $A = \{x \in \mathbb{R} : x \geq 0 \text{ and } x^2 < 2\}$. $A$ is not empty as $1 \in A$. The set of $A$ is bounded above by 2, since if $x \geq 2$, then $x^2 \geq 4 > 2$, so $x \notin A$. So, by AC, $\sup A$ exists; let $\alpha = \sup A$. We will show that $\alpha^2 = 2$, by showing that both $\alpha^2 < 2$ and $\alpha^2 > 2$ are contradictions.
    
    \begin{itemize}[label=$\bullet$]
      \item $\alpha^2 < 2$
    
      For any $n \in \mathbb{N}$ we expand \[\left(\alpha + \frac{1}{n}\right)^2 = \alpha^2 + \frac{2\alpha}{n} + \frac{1}{n^2} \leq \alpha^2 + \frac{2 \alpha + 1}{n},\] noting that $\frac{1}{n^2} \leq \frac{1}{n}$ for $n \geq 1$.

      Let $y = \frac{2 - \alpha^2}{2\alpha + 1}$, which is strictly positive. By \cref{prop:archimedian}, $\exists n_0 \in \mathbb{N}$ s.t. \[\frac{1}{n_0} < \frac{2-\alpha^2}{2\alpha + 1} \text{ or } \frac{2\alpha + 1}{n_0} < 2 - \alpha^2.\] Substituting this $n_0$ into our inequality, we have \[\left(\alpha + \frac{1}{n_0}\right)^2 \leq \alpha^2 + \frac{2 \alpha + 1}{n_0} < \alpha^2 + 2 - \alpha^2 = 2.\] Since $\alpha + \frac{1}{n_0}$ is positive, $\alpha + \frac{1}{n_0} \in A$. But, since $\alpha = \sup A$, $\alpha + \frac{1}{n_0} \leq \alpha$, which is impossible, so $\alpha^2 < 2$ cannot be true.
      \item $\alpha^2 > 2$
      
      Take $n \in \mathbb{N}$; \[\left(\alpha - \frac{1}{n}\right)^2 = \alpha^2 - \frac{2\alpha}{n} + \frac{1}{n^2} > \alpha^2 - \frac{2\alpha}{n}.\] Now, let $y = \frac{\alpha^2 - 2}{2\alpha}$; $y > 0$, and by \cref{prop:archimedian}, $\exists n_0 \in \mathbb{N}$ s.t. \[\frac{1}{n_0} < \frac{\alpha^2 - 2}{2 \alpha}, \text{ or } \frac{2\alpha}{n_0} < \alpha^2 - 2.\] Substituting this $n_0$, we have \[\left(\alpha - \frac{1}{n_0}\right)^2 > \alpha^2 - \frac{2 \alpha}{n_0} > \alpha^2 + 2 - \alpha^2 = 2.\] So for any $x \in A$, we have $\left(\alpha- \frac{1}{n_0}\right)^2 > 2 > x^2$. $\alpha - \frac{1}{n_0}> 0$, and $x > 0$, since $x \in A$. Then, $\left(\alpha - \frac{1}{n_0}\right)^2 > x^2$ gives that $\alpha - \frac{1}{n_0}>x$.

      So, $\alpha - \frac{1}{n_0} > x$ for all $x \in A$. So $\alpha - \frac{1}{n_0}$ is an upper bound for $A$, but since $\alpha = \sup A$, $\alpha - \frac{1}{n_0} \geq \alpha$ ie $\alpha \geq \alpha + \frac{1}{n_0}$, which is impossible. So $\alpha^2 > 2$ cannot be true.
    \end{itemize}

    Thus, $\alpha^2 = 2$.
  \end{itemize}
\end{proof}

\footnotetext{Proof sketch: uniqueness is clear. Existence follows from showing that $\alpha^2$ cannot be either $<$ or $> 2$. This is done by contradiction, taking some number slightly larger/smaller than $\alpha$ for the $</>$ resp., then showing that this number cannot be greater/less than $\alpha$. In the $<$ case, we show that $\alpha + \frac{1}{n_0}$ for a particular $n_0$ must be in $A$, and so $\alpha$ cannot be $\sup A$ and thus a contradiction is reached. For the $>$ case, we need slightly different logic (really, more algebra), and get to another contradiction, this time by showing that $\alpha - \frac{1}{n_0}$ is an upper bound for $A$ by our assumption, contradicting.}

\begin{remark}
  A similar argument gives that for any $x \in \mathbb{R}$, $x \geq 0$, $\exists! \alpha \in \mathbb{R}$, $\alpha \geq 0$ such that $\alpha^2 = x$. This $x$ is called the \emph{square root} of $x$, denoted $\alpha = \sqrt{x}$.
\end{remark}

\begin{remark}
  For any natural number $m \geq 2$ and $x \geq 0$, $\exists ! \alpha \in \mathbb{R}, \alpha \geq 0$ s.t. $\alpha^m = x$. The proof is similar, and we call $\alpha$ the $m$-th root of $x$.
\end{remark}

\begin{remark}
  Our last proof also gives that $\mathbb{Q}$ cannot satisfy AC. Suppose it does, ie any set in $\mathbb{Q}$ bounded from above has a supremum $\in \mathbb{Q}$. Then, consider $B = \{x \in \mathbb{Q} : x \geq 0 \text{ and } x^2 < 2\}$; set $\alpha = \sup B$. The exact same proof can be used, but we will not be able to find an upper bound in $\mathbb{Q}$.
\end{remark}


\subsection{Cardinality}

\begin{definition}
  Let $f: A \to B$.
  \begin{enumerate}
    \item $f$ injective (one-to-one) if $a_1 \neq a_2 \implies f(a_1) \neq f(a_2)$
    \item $f$ surjective (onto) if for any $b \in B \exists a \in A $ s.t. $f(a) = b$.
    \item $f$ bijective if both.
  \end{enumerate}
\end{definition}

\begin{definition}[Composition]
  If $f: A \to B, g: B \to C$, the \emph{composite map} $h = g \circ f$ is define by $h(x) = g(f(x))$. Note that $h: A \to C$.
\end{definition}

\begin{example}
  Consider functions $f,g$.
  \begin{enumerate}
    \item If $f, g$ injective, so is $h = g \circ f$
    \item If $f, g$ bijective, then so is $h$
    \item If $\exists E \subseteq C$, then $h^{-1}(E)= f^{-1}(g^{-1}(E))$
  \end{enumerate}  
\end{example}

\begin{definition}
  The inverse function\footnotemark is defined only for bijective map $f: A \to B$. $y \in B$, $f^{-1}(y) = x$ where $x \in A$ s.t. $f(x) = y$.
\end{definition}

\footnotetext{Not the same as the inverse \textit{image} of a set by a function, which is defined for any function.}

\begin{example}
  \begin{enumerate}
    \item $A = \mathbb{R}, B = (0, \infty),f(x) = e^x$. $f$ is a bijection, and $f^{-1}(y) = \ln y, y \in (0, \infty).$
    \item $A = (-\frac{\pi}{2}, \frac{\pi}{2}, B = \mathbb{R})$. $f(x) = \tan x$, $f^{-1}(y) = \arctan y$
  \end{enumerate}
\end{example}

\begin{definition}[Equal Cardinalities]
  Let $A,B$ be two sets. We say $A, B$ have the same cardinality, denote $A \sim B$ if there exists a function $f: A \to B$.
\end{definition}

\begin{example}
  Let $E = \{2, 4, 6, \dots\}$ (even natural numbers). Define $f : \mathbb{N} \to E$ by $f(n) = 2n$. Thus, $f$ is a bijection, and $\mathbb{N} \sim E$.\footnotemark
\end{example}

\footnotetext{See \href{https://notes.louismeunier.net/Algebra/algebra.pdf}{these independent notes} for more.}

\begin{theorem}
  The relation $\sim$ is a relation of equivalence.
  \begin{enumerate}
    \item $A \sim A$
    \item if $A \sim B$, then $B \sim A$
    \item if $A \sim B$ and $B \sim C$, then $A \sim C$
  \end{enumerate}
\end{theorem}

\begin{definition}[Countable]
  A set $A$ is \emph{countable} if $\mathbb{N} \sim A$.
\end{definition}

\begin{remark}
  According to this, finite sets are not countable; this is just a convention. Sometimes, we say a set is countable if it is finite \textit{or} to above definition holds, where we say that a set is \emph{countably infinite} if it is infinite and countable.

  Other times, finite sets are treated separately than countable sets.
\end{remark}

\begin{theorem}\label{thm:basicfactI}
  Suppose that $A \subseteq B$.
  \begin{enumerate}
    \item If $B$ is finite or countable, then so is $A$
    \item If $A$ is infinite and uncountable, then so is $B$
  \end{enumerate}
\end{theorem}

\begin{definition}[Cartesian Product]
  If $A, B$ sets, $A \times B = \{(a,b) :a,b \in A,B\}$.
\end{definition}

\begin{proposition}\label{prop:basicfactII}
  $\mathbb{N}\times \mathbb{N} \sim \mathbb{N}$; there exists a bijection $f: \mathbb{N} \times \mathbb{N} \to \mathbb{N}$.
\end{proposition}

\begin{proposition}\label{prop:equivcountable}
  Let $A$ be a set. The following are equivalent statements:
  \begin{enumerate}[label=(\alph*)]
    \item $A$ is finite or a countable set;
    \item there exists a surjection from $\mathbb{N}$ onto $A$;
    \item there exists a injection from $A$ into $\mathbb{N}$.
  \end{enumerate}
\end{proposition}

\begin{proof}[Proof]
  We proceed by proving that each statement implies the next (and thus are equivalent).
  \begin{itemize}
    \item  (a)$ \implies$ (b): Suppose $A$ is finite and has $\mathbb{N}$ elements. Then there exists a bijection $h :\{1,2,\dots n\} \to A$. We now define a map $f: \mathbb{N} \to A$, by setting $$f(m) = \begin{cases}
      h(m) & \text{if } m \leq n\\
      h(n) & \text{if } m > n
    \end{cases}.$$
    $f$ is surjective, and thus (b) holds.
    If (a) countable, $\exists$ bijection $f : \mathbb{N} \to A$, and any bijection is a surjection, so (b) also holds.
    \item (b) $\implies$ (c): Let $h: \mathbb{N} \to A$ be a surjection, whose existence is guaranteed by (b). Then, for any $a \in A$, the set $$h^{-1}(\{a\}) = \{m \in \mathbb{N}: h(m) = n\} \neq \varnothing,$$ since $h$ is a surjection. Then, by the well-ordering property of $\mathbb{N}$, the set $h^{-1}(\{a\})$ has a least element.\\
    If $n$ is the least element of $h^{-1}(\{a\})$, we set $f(a) = $. This defines a function \[f: A \to \mathbb{N},\] and we aim to show that $f$ is injective, ie that $f(a_1) = f(a_2) \implies a_1 = a_2$.\\
    Suppose $f(a_1) = f(a_2) = n$. Then, $n$ is the least element of $h^{-1}(\{a_1\})$ and of $h^{-1}(\{a_2\})$, and in particular, $h(n) = a_1$ and $h(n) = a_2$, and thus $a_1 = a_2$ and so $f$ is indeed injective.

    \item (c) $\implies$ (a): Let $f: A \to \mathbb{N}$ be an injection, whose existence is guaranteed by (c). Consider the range of $f$, ie \[f(A) = \{f(a) : a \in A\}.\] Since $f$ an injection, $f$ is a bijection between $A$ and $f(A)$.\\
    Otoh, $f(A) \subseteq \mathbb{N}$, and so by \cref{thm:basicfactI}, $f(A)$ is either finite or countable, and there exists a bijection between $A$ and some set that is either fininte or countable. Thus, $A$ must also be finite or countable, and so (a) holds.
  \end{itemize}
\end{proof}

\begin{theorem}
  Let $A_n, n = 1,2,\dots$ be a sequence of sets such that each $A_n$ is either finite or countable. Then, their union \[A = \bigcup_{n=1}^\infty A_n\] is also either finite or countable.
\end{theorem}

\begin{proof}[Proof]
  We will use (a) $\iff$ (b) from \cref{prop:equivcountable} to prove this.

  Since each $A_n$ finite or countable, by (a) $\implies$ (b), there exists a surjection $$\varphi_n: \mathbb{N} \to A_n.$$ Now, let $h: \mathbb{N} \times \mathbb{N} \to A,$ (the union) by setting \[h(n,m) = \varphi_n(m).\] We aim to show that $h$ is also surjective.\\ If $a \in \bigcup_{n=1}^\infty A_n$, then $a \in A_n$ for some $n \in \mathbb{N}$. Since $\varphi_n : \mathbb{N} \to A_n$ is a surjection, there exists an $m \in \mathbb{N}$ s.t. $\varphi_n(m) = a$. By definition of $h$, we have \[h(n,m) = a,\] and thus $h$ is a surjection.

  By \cref{prop:basicfactII}, there exists a bijection $f: \mathbb{N} \times \mathbb{N} \to \mathbb{N}$, and we can define the composite map \[h \circ f : \mathbb{N} \to A\,(= \cup_{n=1}^\infty A_n),\] which is a surjection as both $h,f$ are surjections. So, there exists a surjection from $\mathbb{N} \to A$, and by \cref{prop:equivcountable}, (b) $\implies$ (a), and thus $A = \bigcup_{n=1}^\infty A_n$ is also finite our countable.

\end{proof}

\begin{remark}
  If $A = \bigcup_{n=1}^\infty A_n$, where each $A_n$ is either finite or countable, and at least one $A_n$ is countable, then $A$ is countable.
\end{remark}

\begin{remark}
  If $A_1, \dots,  A_n$ are finitely many finite or countable sets then their union $A_1 \cup \cdots \cup A_n$ is also finite or countable (essentially just previous proof where we use $n$ instead of $\infty$ for the upper limit of the union...). 
\end{remark}

\begin{theorem}\label{thm:countablerationals}
  The set $\mathbb{Q}$ of rational numbers is countable.
\end{theorem}

\begin{proof}[Proof]
  We write $$\mathbb{Q}= A_0 \cup A_1 \cup A_2,$$ where $A_0 = \{0\}, A_1 = \{\frac{m}{n} : m,n \in \mathbb{N}\}$, and $A_2 = \{- \frac{m}{n}: m,n \in \mathbb{N}\}$.\\
  Let us show that $A_1$ is countable; define $$h: \mathbb{N} \times \mathbb{N} \to A, f(m,n) = \frac{m}{n}.$$ $h$ is clearly a surjection; if $f : \mathbb{N} \to \mathbb{N} \times \mathbb{N}$ is a bijection, then by \cref{prop:basicfactII}, $h \circ f : \mathbb{N} \to A_1$ is a surjection. By \cref{prop:equivcountable}, $A_1$ is countable.\\
  We prove that $A_2$ countable in essentially the same way.\\
  Then, $A_0 \cup A_1 \cup A_2$ is also countable, as it is the union of countable sets, and thus $\mathbb{Q}$ is also countable.
\end{proof}
\newpage
\begin{theorem}
  The set $\mathbb{R}$ of real numbers is uncountable.\footnotemark
\end{theorem}
\footnotetext{Proof sketch: by contradiction. Assume that a bijection exists, and show that it cannot be a surjection by the previous props/thms. Specifically, carefully construct nested intervals $I_n$, for which $x_i \notin I_i$, and then show that the intersection of all these intervals is empty, contradicting the nested interval property of the real line.}
\begin{proof}[Proof]
  We will argue by contradiction; suppose $\mathbb{R}$ is countable, then show that the nested interval property (\cref{thm:nestedinterval}) of the real line fails.\\
  Let $f: \mathbb{N} \to \mathbb{R}$ be a bijection, setting $f(1) = x_1, f(2) = x_2, \dots, f(n) = x_n, \dots$; we can then list the elements of $\mathbb{R}$ as $\mathbb{R} = \{x_1, x_2, x_3, \dots, x_n, \dots\}$.\\
  We can now construct a sequence $I_n, n \in \mathbb{N}$ of bounded, closed intervals, such that $I_1$ does not contain $x_1$.\\ If $x_2 \notin I_1$, then $I_2 = I_1$. If $x_2 \in I_1$, then divide $I_1$ into four equal closed intervals.\\
  Call the leftmost/rightmost of these intervals $I_1'$ and $I_1''$ respectively. We know that $x_2 \in I_1$, so we must have that either $x_2 \notin I_1'$ or $x_2 \notin I_1''$ If $x_2 \notin I_1'$, then $I_2 = I_1'$. If $x_2 \notin I_1''$, then $I_2 = I_1''$.\\
  Thus, we have constructed $I_1, I_2$ s.t.\[I_1 \supseteq I_2 \text{ and } x_1 \notin I_1,x_2 \notin I_2.\] Consider $x_3$; if $x_3 \notin I_2$, then $I_3 = I_2$. If $x_3 \in I_2$, we repeat the "dividing" process as before. Since $x_3 \in I_2$, either $x_3 \notin I_2'$ or $x_3 \notin I_2''$. If $x_3 \notin I_2'$, $I_3 = I_2'$. Else, if $x_3 \notin I_2''$, $I_3 = I_2''$.\\
  We have now that \[I_1 \supseteq I_2 \supseteq I_3 \text{ and } x_1 \notin I_1, x_2 \notin I_2, x_3 \notin I_3,\] and we can continue this construction to obtain an infinite sequence of bounded, closed intervals $I_n$ s.t. $$I_1 \supseteq I_2 \supseteq \cdots \supseteq I_n \supseteq I_{n+1} \supseteq \cdots,$$ and for each $n$, $x_n \notin I_n$.\\
  Consider the intersection of all these $I_n$'s,$$\bigcap_{n=1}^\infty I_n.$$ For every $m$, $x_m \notin I_m$, so for every $m \in \mathbb{N}, x_m \notin \bigcap_{n=1}^\infty I_n$, and so $\mathbb{R} = \{x_1, x_2, \dots x_m, \dots\}$ has an empty intersection with this intersection, ie
  $$\mathbb{R} \cap \left( \bigcap_{n=1}^\infty I_n \right) = \varnothing.$$ Otoh, $\bigcap_{n=1}^\infty I_n \subseteq \mathbb{R}$, so we must have that $\bigcap_{n=1}^\infty I_n = \varnothing$ contradicting the nested interval property of the real line which states that this intersection must not be empty. We thus have a contradiction, and our assumption that $\mathbb{R}$ countable fails. \footnotemark
\end{proof}
\footnotetext{Note that \cref{thm:nestedinterval} is built upon the Axiom of Completeness, a "fact" of $\mathbb{R}$ (what makes it "distinct" from $\mathbb{Q}, \mathbb{N}$, etc). Thus, we are really just using AC, with some abstractions sts.}
% TODO: add proof sketch
\begin{proposition}
  The set $J$ of all irrational numbers in $\mathbb{R}$ is uncountable.
\end{proposition}

\begin{proof}[Proof]
  We have that $\mathbb{R} = \mathbb{Q} \cup J$. If $J$ countable, then $\mathbb{R}$ would also be countable as the union of two countable sets (as we showed $\mathbb{Q}$ countable in \cref{thm:countablerationals}). $\mathbb{R}$ uncountable, so $J$ is also uncountable.
\end{proof}

\begin{proposition}
  The set $(-1,1) \subseteq \mathbb{R}$ is uncountable.
\end{proposition}

\begin{proof}[Proof]
  We can write $\mathbb{R} = \bigcup_{n=1}^\infty (-n,n)$. If each $(-n,n)$ is countable, then $\mathbb{R}$ would also be countable, as a countable union of countable sets. Thus, there must exist some $n_0 \in \mathbb{N} \text{ s.t. } (-n_0, n_0)$ is not countable. The map\[f: (-n_0, n_0) \to (-1,1), f(x) = \frac{x}{n_0}\] is a bijection, and so $(-1,1)$ is uncountable.
\end{proof}

\begin{example}
  Show that the map \[f(x) = \frac{x}{1-x^2}\] is a bijection between $(-1,1)$ and $\mathbb{R}$ ie $(-1,1) \sim \mathbb{R}$.
\end{example}

\begin{proof}[Proof]
  % TODO
\end{proof}



\begin{proposition}
  Any bounded non-empty open interval $(a,b) \in \mathbb{R}$ is uncountable.
\end{proposition}

\begin{proof}[Proof]
  We will construct a bijection $f: (a,b) \to \mathbb{R}$ so that $(a,b) \sim \mathbb{R}$. Since $\mathbb{R}$ is  uncountable, so must $(a,b)$. 
  
  The map \[f(x)=\frac{2(x-a)}{b-a} - 1\] is a bijection between $(a,b)$ and $(-1,1)$, and we have shown that $(-1,1) \sim \mathbb{R}$, so $(a,b) \sim \mathbb{R}$, and thus any open interval has the same cardinality as $\mathbb{R}$.
\end{proof}

\begin{example}
  Prove that $\exists$ bijection between $[0,1)$ and $(0,1)$, and conclude that $[0,1) \sim (0,1) \sim \mathbb{R}$. Then conclude for any $a < b$, $[a,b) \sim \mathbb{R}$.

  \begin{proof}[Proof]
    % TODO
  \end{proof}

\end{example}

\subsubsection{Power Sets}
\begin{definition}[Power Set]
  Let $A$ be a set. The \emph{power set} of $A$m denoted $\mathcal{P}(A)$ is the collection of all subsets of $A$.

  Generally, if $A$ finite of size $n$, $\mathcal{P}(A)$ has $2^n$ elements.
\end{definition}

\begin{theorem}[Cantor Power Set Theorem]
  Let $A$ be any set. Then there exists no surjection from $A$ onto $\mathcal{P}(A)$. \footnotemark
\end{theorem}

\footnotetext{Certified Classic}

\begin{proof}[Proof]
  Suppose that there exists a surjection, \[f: A \to \pset{A}.\] Let $D \subseteq A$ defined as \[D = \{a \in A: a \notin f(a)\}.\] Since $D \subseteq \pset{A}$, and $f$ is surjective, there must exist some $a_0 \in A \st f(a_0) = D$.\\
  We have two cases:
  \begin{enumerate}
    \item $a_0 \in D.$ But then, by definition of $D$, $a_0 \notin f(a_0) = D$, so $a_0 \in D$ is not possible as it implies $a_0 \notin D$.
    \item $a_0 \notin D.$ But then, since $D = f(a_0)$, $a_0 \notin f(a_0)$, and so by definition of $D$, $a_0 \in D$, which is again not possible.
  \end{enumerate}
  So, the assumption of a surjection existing has led to $a_0 \in A$ such that neither $a_0 \in D$ nor $a_0 \notin D$, which is impossible. Thus there can be no surjective $f$.\\
  Notice, though, that there exists an injection $A \to \pset{A}, a \mapsto \{a\}$, and thus there is an injection but no bijection.\\
  Thus, we can say that $\pset{A}$ is strictly bigger than $A$.\\
\end{proof}


\section{Sequences}
\subsection{Definitions}
\begin{definition}
  Let $A$ be a set. An $A$-valued sequence indexed by $\mathbb{R}$ is a map \[x: \mathbb{N} \to A.\] The value $x(n)$ is called the $n$-th element of the sequence. One writes $x(n) = x_n$, or lists its elements \[\{x_1, x_2, x_3, \dots\} \equiv \{x_n\}_{n \in \mathbb{N}} \equiv (x_n)_{n \in \mathbb{N}} \equiv \{x_n\}.\]
\end{definition}

\begin{definition}[Convergence]
  We say that a sequence $(x_n)$ converges to a real number $x$ if for every $\epsilon > 0$, $\exists N \in \mathbb{N} \st$ for all $n \geq N$ we have \[|x_n - x| < \epsilon.\] If sequence $(x_n)$ converges to $x$, we write $\lim_{n \to \infty} x_n = x$.
\end{definition}

\begin{example}
  Let $(x_n)$ be a sequence defined by $x_n = \frac{1}{n}, n \in \mathbb{N}$, then $\lim_{n \to \infty} x_n = 0$.
  \begin{proof}[Proof]
    Let $\epsilon > 0$. Let $N \in \mathbb{N} \st N > \frac{1}{\epsilon}$. Then for $n \geq N$, we have that $$0 < \frac{1}{n}\leq \frac{1}{N} < \epsilon.$$ So, for $n \geq N, |x_n - 0| < \epsilon$, and so the limit is $0$.
  \end{proof}
\end{example}



\begin{definition}[Limit Redefinition]
  The limit can be written in terms of quantifiers.
  \[\lim_{n \to \infty} x_n = x\] means that \[(\forall \epsilon > 0)(\exists N \in \mathbb{N})(\forall n \geq N)(|x_n - x| < \epsilon).\]
\end{definition}

% TODO: add "strategy"
\begin{example}
  Prove that $$\lim_{n\to \infty} \frac{n^2+1}{n^2} = 1.$$

\begin{proof}[Proof]
  Let $\epsilon > 0$. Let $N$ be a natural number such that $N > \frac{1}{\sqrt{\epsilon}}$. Then, for $n \geq N$,
  \begin{align*}
    |\frac{n^2+1}{n^2}-1| = |\frac{n^2+1-n^2}{n^2}| = \frac{1}{n^2} \leq \frac{1}{N^2} < \epsilon.
  \end{align*}
\end{proof}
\end{example}

\begin{definition}[Divergent Sequences]
  If a sequence $(x_n)$ does not converge to any real number $x$, we say that the sequence is divergent. For instance, consider \[x_n = (-1)^n, n \geq 1.\] The sequence alternates between $1$ and $-1$ and so intuitively does not converge. How do we prove it?
\end{definition}

\begin{proof}[Proof]
  By contradiction; suppose that $x_n  = (-1)^n$ be a converging sequence. Let $x = \lim_{n\to \infty}x_n$. Take $\epsilon = 1$, then $\exists N \in \mathbb{N} \st$ for all $n \geq N$ we have that $|x-x_n| < \epsilon = 1$.\\
  Consider indices $n = N, n = N+1$. We have \[|x_{N+1} - x_N| = |x_{n+1} - x + x - x_N| \leq \underbrace{|x_{N+1} -x| + |x - x_N|}_{\text{triangle inequality}} < 1 + 1 = 2.\] But we also have that $$|(-1)^{N+1}-(-1)^N| = |(-1)^{N+1}+(-1)^{N+1}| = 2,$$ We thus have that $2 < 2$, which is a contradiction. Thus, $x_n$ is not convergent.
\end{proof}

\subsection{Properties of Limits}
\begin{lemma}[Triangle Inequality]
  For $x,y,z \in \mathbb{R},$
  \[(i)\quad|x + y|  \leq |x| + |y|; \qquad (ii) \quad|x-y| \leq |x-z| + |z-y|\footnotemark\]
\end{lemma}

\footnotetext{Generally, proofs involving limits will consist of 1) picking/defining an $\epsilon$ based on given limit/series definitions, and then 2) using triangle inequality/related techniques to reach the desired conclusion.}


\begin{proof}[Sketch proof]
  $(i)$:
  $|x+y| = \begin{cases}
    x + y & x + y \geq 0\\
    -(x+y) & x + y \leq 0
  \end{cases}$.
  So if $x + y \geq 0$, $|x+y| = x + y \leq |x| + |y|$. \\
  If $x+y >0, |x+y| = -(x+y) = (-x) + (-y) \leq |x| + |y$.\\\\
  $(ii)$: $|x-y| = |x-z + z - y| \leq |x-z| + |z-y|$ (using $(i)$).
\end{proof}

\begin{definition}[Metric Space]
  A pair $(X,d)$ where $X$ is a set and $d: X \times X \to [0, \infty)$ having the following properties:
  \begin{enumerate}
    \item $d(x,y) = 0 \iff x = y$;
    \item $d(x,y) = d(y,x)$;
    \item $\forall x,y,z \in X$, the triangle inequality holds;
    \[d(x,y) \leq d(x,z) + d(z,y)\]
  \end{enumerate}
\end{definition}

\begin{example}
  $X = \mathbb{R}$  , $d(x,y) = |x-y|$. Clearly, 1., 2., 3. all hold.
\end{example}


\begin{theorem}
  A limit of a sequence is unique. In other words, if the sequence is converging, then its limit is unique. The sequence cannot converge to two distinct numbers $x$ and $y$.\footnotemark
\end{theorem}

\footnotetext{
  Proof sketch: contradiction, assume two distinct limits, and take $\epsilon$ as their midpoint. Arrive at a contradiction by using triangle inequalities to show that $|x-y| < |x-y|$, and thus the limits cannot be distinct.
}
\begin{proof}[Proof]
  By contradiction; suppose $\exists (x_n) \st \lim_{n \to \infty} x_n = x$ and $\lim_{n \to \infty} x_n = y$, and that $x \neq 0$.\\
  Take $\epsilon = \frac{|x-y|}{2}$. Since $x \neq y$, we have that $\epsilon > 0$. Since $\lim_{n \to \infty} x_n = x$, $\exists N_1 \in \mathbb{N}$ s.t. for $n \geq N_1$, $|x_n - x| < \epsilon$. \\
  Similarly,  since $\lim x_n = y$, $\exists N_2 \in \mathbb{N}$ s.t for $g \geq N_2, |x_n - y| < \epsilon$. \\
  Take some $n \geq \max(N_1, N_2)$; then \begin{align*}
    |x-y| = |x-x_n + x_n - y| &\leq |x-x_n|+|x_n-y|\\
    &< \epsilon + \epsilon = |x-y|\\
    &\implies |x-y| < |x-y|, \bot
  \end{align*}
\end{proof}

\begin{theorem}\label{thm:convbound}
  Any converging sequence is bounded.\footnotemark\\In other words, if $(x_n)$ is a converging sequence, $$\exists M > 0 \st |x_n| \leq M \forall n \geq 1.$$
\end{theorem}

\footnotetext{
  Take $\epsilon = 1$, which is greater than $|x_n -x |$ by limit definition for $n \geq N$ for some $N$. We then use this to show that $|x_n| < 1 + |x|$, then construct a summation $M$ such that it bounds $|x_n|$; it is equal to $|x_1| + |x_2| + \cdots$ up to $|x_{N-1}|$, then plus $1+|x|$. We have finished.
}


\begin{proof}[Proof]
  Let $(x_n)$ be a converging sequence, and $x = \lim_{n\to\infty} x_n$. Take $\epsilon = 1$ in the definition of the limit; then, $\exists N \in \mathbb{N} \st \forall n \geq N$, $|x_n - x| < 1$.\\
  This gives that for $n \geq N$, $|x_n| = |x_n - x + x| \leq |x_n - x| + |x| < 1 + |x|$.\\
  Let now $M = |x_1| + |x_2| + \cdots + |x_{N-1}| + (1 + |x|)$. Then, for any $n \geq 1$, $|x_n| \leq M$;\\If $n \leq N-1$, then $|x_n|$ is a summand in $M$, and thus $|x_n| \leq M$.\\ If $n \geq N$, then we have by the choice of $N$ that $|x_n| < 1 + |x| \leq M$.\\ Thus, for all $n \geq 1$, $|x_n| \leq M$, and is thus bounded given $(x_n)$ converges.
\end{proof}

\begin{proposition}[Algebraic Properties of Limits]\label{prop:apl}
  Let $(x_n), (y_n)$ be sequences such that\footnotemark \[\lim x_n = x, \quad \lim y_n = y.\]
  Then:
  \begin{enumerate}
    \item For any constant $c$, \(\lim c \cdot x_n = c \cdot \lim x_n = c\cdot x\)
    \item $\lim (x_n + y_n) = \lim x_n + \lim y_n = x+y$
    \item $\lim x_n \cdot y_n = (\lim x_n)(\lim y_n) = x\cdot y$
    \item Suppose $y \neq 0$, $y_n \neq 0 \forall n \geq 1$. Then, $\lim \frac{x_n}{y_n} = \frac{\lim x_n}{\lim y_n} = \frac{x}{y}$
  \end{enumerate}
\end{proposition}

\footnotetext{Note that the contrary of these statements need not hold; ie, if $\lim (x_n \cdot y_n)$ exists, this does not imply the existence of $\lim x_n$ and $\lim y_n$. Consider \cref{example:counterlimitalgebra}}
\newpage
\begin{remark}
  Let $X$ be the collection of all sequences of real numbers, $X = \{(x_n): x_n \text{ is a sequence}\}.$ If $(x_n) \in X$ and $c \in \mathbb{R}$, we can define $c \cdot(x_n) = (c\cdot x_n)$\footnotemark; this defines \emph{scalar multiplication} on $X$.\\ We can also define \emph{addition}; if $(x_n)$ and $(y_n)$ are two sequences in $X$, then $(x_n) + (y_n) = (x_n + y_n)$. Then, with these two operations $X$ is a \emph{vector space}.
\end{remark}

\footnotetext{NB: this denotes $c$ multiplying to each $n$th element in $x_n$, ie $c \cdot x_1$, $c \cdot x_2$, etc}

\begin{example}\label{example:counterlimitalgebra}
  Take $x_n = (-1)^n, y_n = (-1)^{n+1}$, $n \geq 1$.\\
  $(x_n) + (y_n) = 0, x_n \cdot y_n = -1$, and so $\lim x_n + y_n = 0, \lim x_n \cdot y_n = -1$, while neither $\lim x_n$ nor $\lim y_n$ exist.
\end{example}

\begin{proof}[Proof (part 3. of \cref{prop:apl})]
  Take\footnotemark $\lim x_n = x, \lim y_n = y$. Since $(x_n)$ is converging, it is bound by \cref{thm:convbound}, and there exists $M > 0 \st \forall n \geq 1, |x_n| \leq M$.\\
  Now, \begin{align*}
    |x_n y_n - xy| &= |x_n y_n - x_n y + x_n y - xy|\\
    & \leq |x_n y_n - x_n y| + |x_n y - xy|\\
    &= |x_n| \cdot |y_n - y| + |y|\cdot |x_n - x|\\
    &\leq M \cdot |y_n - y| + |y|\cdot |x_n -x| \quad (i)
  \end{align*}
  Let $\epsilon > 0$; since $\lim y_n = y$, there exists $N_1 \in \mathbb{N} \st n \geq N_1, |y_n - y| < \frac{\epsilon}{2M}$. Sim, since $\lim x_n = x, \exists N_2 \in \mathbb{N}$ s.t. $|x_n - x|  <\frac{\epsilon}{2(|y|+1)}$\\
  Let $N = \max (N_1, N_2)$, $n \geq N$. Then, we have, with $(i)$,
  \begin{align*}
    (i) \quad |x_n y_n - xy| &\leq M\cdot |y_n - y| + |y| \cdot |x_n| - x\\
    &< M\cdot \frac{\epsilon}{2M} + |y|\cdot \frac{\epsilon}{2 (|y|+1)}\\
    &\leq \frac{\epsilon}{2} + \frac{\epsilon}{2}.
  \end{align*}
  Thus, for $n \geq N$, $|x_n y_n - xy| < \epsilon$, and by definition of the limit, $\lim x_n y_n = xy$.
\end{proof}
\footnotetext{
Proof sketch: take an upper bound of $x_n$. Then, show that $|x_n y_n - xy| < \epsilon$, by using triangle inequalities to show inequality to a combination of $M$, arbitrarily small values (by def of limits of $x_n, y_n$ resp,), and $|y|$.}

\begin{theorem}[Order Properties of Limits]
  Let $(x_n), (y_n)$ be two sequences such that \[\lim x_n = x, \quad \lim y_n = y.\]
  \begin{enumerate}
    \item $x_n \geq 0 \forall n \implies x \geq 0$.
    \item $x_n \geq y_n \forall n \implies x \geq y$.
    \item $c$ is constant since $c \leq x_n \forall n \geq 1 \implies c \leq x$. $x_n \leq c \forall n \geq 1 \implies x_n \leq c$.
  \end{enumerate}
\end{theorem}

\begin{remark}
  2., 3. follow from 1. Set $z_n = x_n - y_n \forall n \geq 1$. Then, $z_n \geq 0 \forall b \geq 1$, $\lim z_n = \lim (x_n-y_n) = \lim x_n - \lim y_n$ (as these limits exist) $= x-y$. By 1., $\lim z_n \geq 0$, and so either $x - y \geq 0$ or $x \geq y$.
\end{remark}

\begin{proof}[Proof of 1.]
   Suppose 1. does not hold; suppose $\exists (x_n) \st \lim x_n = x$, $x_n \geq 0 \forall \geq,$ but $x <0$.\\
   Let $\epsilon > 0 \st x < -2 \epsilon < 0$. With this $\epsilon, \lim x_n = x$ gives that $\exists N \in \mathbb N \st \forall n \geq N, |x_n - x| < \epsilon$, or particularly, $x_n - x < \epsilon$.\\
   Then, $x_n < \epsilon + x$, and since $x < -2 \epsilon$, we have $\forall n \geq N$, $x_n < -\epsilon$, and thus $\forall n \geq N$, $x_n < 0$, a contradiction.
\end{proof}

\begin{theorem}[The Squeeze Theorem]
  Let $(x_n), (y_n), (z_n)$ be sequences such that \(x_n \leq y_n \leq z_n, \,\forall n \geq 1,\) and \(\lim_{n \to \infty} x_n = \lim_{n \to \infty} z_n = \ell,\) then \(\lim_{n \to \infty}y_n = \ell.\footnotemark\)
\end{theorem}
\footnotetext{Sketch: This follows a similar technique to many that follow. Use the definitions of the limits of $x_n, z_n$ to take an arbitrary $\epsilon$, and an $N$ for each respective limit. Take the max of these $N$'s, and show that for all $n \geq \max N_i$, you can show that f $y_n - l$ is less than $\epsilon$ and greater than $-\epsilon$. Really, this is just a proof of applying definitions correctly.}

\begin{proof}[Proof]
  Let $\epsilon > 0$. Since $\lim x_n = \ell$, there $\exists N_1 \in \mathbb{N} \st \forall n \geq N_1, |x_n - \ell| < \epsilon$.\\
  Since $\lim z_n = \ell$, there $\exists N_2 \in \mathbb{N} \st \forall n \geq N_2, |z_n - \ell| < \epsilon$.\\
  Take $N = \max \{N_1,N_2\}$ and take $n \geq N$. Then, \[y_n \leq z_n \implies y_n - \ell \leq z_n - \ell \leq |z_n - \ell| < \epsilon,\] since $n \geq \max\{N_1, N_2\} \implies n \geq N_2.$\\
  Now, we have that \[y_n \geq x_n \implies y_n - \ell \geq x_n - \ell > - \epsilon, \] since $|x_n - \ell| < \epsilon $ for $n \geq N_1,$ and our $n$ is $\geq \max\{N_1,N_2\}.$ Thus, for $n \geq N$, \[-\epsilon < y_n - \ell < \epsilon \implies |y_n - \ell| < \epsilon,\] and thus $\lim y_n = \ell$, by definition.
\end{proof}

\begin{definition}[Increasing/Decreasing]
  A sequence $(x_n)$ is called \emph{increasing} if $x_{n+1} \geq x_n \forall n \in \mathbb{N}$, and is \emph{decreasing} if $x_{n_1} \leq x_n \forall n \in \mathbb{N}$.
\end{definition}

\begin{definition}[Bounded from above/below]
  A sequence $(x_n)$ is called \emph{bounded} from above if there exists some $M \in \mathbb{R} \st x_n \leq M \forall n \geq 1$.\\Sequence $(x_n)$ is bounded from below if there exists some $M \in \mathbb{R} \st x_n \geq M \forall n \geq 1$.
\end{definition}

\begin{theorem}[Monotone Convergence Theorem]
  The following relate to bounded above/below and increasing/decreasing sequences:\footnotemark
  \begin{enumerate}
    \item Let $(x_n)$ be an increasing sequence that is bounded from above. Then $(x_n)$ is converging.
    \item Let $(x_n)$ be a decreasing sequence that is bounded from below. then $(x_n)$ is converging.
  \end{enumerate}
\end{theorem}
\newpage
\footnotetext{Sketch: 1,2 are proven very similarly. For 1., take the set of all $x_n$ in the given sequence. Since the sequence is bounded, then so is the set, and so we can take its supremum. Use the $\epsilon$ definition of $\sup$ to show that this supremum is also the limit of the sequence (basically, a bunch of inequalities, and being careful with definitions). 2. follows identically but using the infimum.}

\begin{proof}[Proof (of 1)]
  Let $A = \{x_n: n \geq 1\}$. Since $(x_n)$ is bounded above by $M$, the set $A$ is bounded from above. Let $\alpha = \sup A$, which exists by AC.\\Let $\epsilon > 0$. Since $\alpha$ is the least upper bound for $A$, $\alpha - \epsilon$ is \textit{not} an upper bound of $A$ ($\alpha - \epsilon < \alpha$). Hence, there must exist some $N \in \mathbb{N}$ such that $\alpha - \epsilon < x_N$ (if it didn't exist, then $\alpha$ wouldn't be the supremum \dots). Then, for $n \geq N$, and since $(x_n)$ increasing, $$\alpha - \epsilon < x_N \leq x_n \leq \alpha.$$ Then, for all $n \geq N$, \[\alpha - \epsilon < x_n \leq \alpha \implies - \epsilon < x_n -\alpha \leq 0,\] and so $|x_n - \alpha| < \epsilon $ for $n \geq N$. By definition, $\alpha = \lim x_n$
\end{proof}

\begin{example}
  % TODO
  A sequence $(x_n)$ is called \emph{eventually increasing} if there exists some $N_0 \in \mathbb{N} \st \forall n \geq N_0 x_{n+1} \geq x_n$. If $(x_n)$ is eventually increasing and bounded from above, $\lim x_n = \alpha$ exists.
\end{example}

\begin{example}
  Let $(x_n)$ be a sequence defined recursively by $x_1 = \sqrt{2}$ and $x_{n+1} = \sqrt{2 + x_n}, n \geq 1$. So $x_2 = \sqrt{2 + \sqrt{2}}, x_3 = \sqrt{2 + \sqrt{2 + \sqrt{2}}} \cdots$, $x_n = 2 \cos \frac{\pi}{2^{n+1}}, n \geq 1$. Show that $\lim x_n = 2$.
  \begin{proof}[Proof]
    We will prove this using the Monotone Convergence Thm by showing that the $x_n$ is bounded from above and increasing, which implies that the limit exists. We will then find the actual limit.\\
    Recall that $n \geq 1, x_n \leq 2$. We will prove this by induction. Let $S \subseteq \mathbb{N}$ be the set of indices such that $x_n \leq 2$. Since $x_1 = \sqrt{2} < 2$, $1 \in S$. Now suppose some $n \in S$, ie $x_n \leq 2$. Then, we have that $x_{n+1} = \sqrt{2 + x_n} \leq \sqrt{2+2} = 2 \implies x_{n+1} \leq 2$. Thus, by induction, $n \in S \implies n+1 \in S \implies S = \mathbb{N}$, ie $x_{n} \leq 2 \forall n \in \mathbb{N}$. Thus, our sequence is bounded from above.\\
    We now prove that $(x_n)$ is increasing. Let $S \subseteq \mathbb{N} \st n \in S \iff x_{n+1} \leq x_n$. $x_2 = \sqrt{2 + \sqrt{2}} \geq \sqrt{2} = x_1 \implies x_1 \leq x_2 \implies 1 \in S$. Suppose $n \in S \implies x_{n+1} \geq x_n$. Then, $x_{n+2} = \sqrt{2 + x_{n+1}} \geq \sqrt{2 + x_n} = x_{n+1} \implies n+1 \in S$. Thus, $S = \mathbb{N}, $ so $x_{n+1} \geq x_{n} \forall n \in \mathbb{N}$.\\
    So the sequence $(x_n)$ is increasing and bounded from above, and thus $\exists \lim x_n = \alpha$. To find the value of $\alpha$, consider $x_{n+1} = \sqrt{2 + x_n}$, or $x_{n+1}^2 = 2 + x_n$. We can also write that $\alpha = \lim x_n = \lim x_{n+1}.\footnotemark$ We then have that $\lim x_{n+1} = \alpha \implies \lim x_{n+1}^2 = \alpha^2$, and thus $x_{n+1}^2 = 2+ x_n \implies \lim  x_{n+1}^2 = \lim (2+x_n) \implies \alpha^2 = 2 + \alpha  \implies \alpha = 2, -1$. $x_n \geq 0 \forall n$, by Order Limit Theorem, and so $\alpha \geq 0$ and thus $\alpha = 2$.
  \end{proof}
\end{example}

\footnotetext{Add proof}

\begin{example}
  Let $(x_n)$ be defined recursively by $x_1 = 2$ and $x_{n+1} = \frac{1}{2}\left(x_n + \frac{2}{x_n}\right)$ for $n \geq 1$. Then, $(x_n)$ is converging and $\lim x_n = \sqrt{2}$.
  \begin{proof}[Proof]
    We will show that $(x_n)$ bounded from below and decreasing, implying the limit exists. We will show that for $n$, $x_n \geq \sqrt{2}$. For $n = 1$, $2 \geq \sqrt{2}$. For $n > 1$, we will use $\frac{1}{2} (a+b) \geq \sqrt{ab}$ for $a,b\geq 0$. The prove it, square both sides and observe \[\frac{1}{4}(a^2 + 2ab + b^2) \geq ab \iff a^2 + 2ab + b^2 \geq 0 \iff (a-b)^2 \geq 0.\] We then have that $x_{n+1} = \frac{1}{2}(x_n + \frac{2}{x_n}) \geq \dots \geq \sqrt{2} \implies x_n \geq \sqrt{2} \forall n \geq 1$, ie, it is bounded from below.\\
    We will now show that the sequence is decreasing.
    \[x_n - x_{n+1} = x_n - \frac{1}{2}(x_n + \frac{2}{x_n}) = \frac{1}{2}x_n - \frac{1}{x_n} = \frac{1}{2x_n}(x_n^2 - 2).\]
    % TODO: finish
  \end{proof}
\end{example}





























\newpage
\section{Appendix}
\subsection{Tutorials}
\subsubsection{Tutorial I (Sept 13)}

\begin{enumerate}
  \item We say $n$ odd if $\exists k$, $n = 2k+1$. Prove that the product of two odds is odd.
  \begin{proof}[Proof]
    Take two odd integers, $n_1 = 2k+1$ and $n_2 = 2j+1$. The product $n_1 \times n_2 = (2k+1)(2j+1) = 4kj+2(k+j)+1$. We have, then \[\underbrace{4kj+2(k+j)}_{\text{even}} + 1.\] Even + odd = odd, thus odd.
  \end{proof}

  \item \textbf{Proof by Contrapositive:} $P \implies Q \equiv \neg Q \implies \neg P$. 
  Let $q \in \mathbb{Q}$. Prove: If $x \in \mathbb{R}\setminus\mathbb{Q}$, then $q+x$ is irrational.
  \begin{proof}[Proof (contrapositive)]
    Let $q+x$ be rational. The sum of rationals is rational, and thus $q,x \in \mathbb{Q}$, and thus $x \notin \mathbb{R} \setminus \mathbb{Q}$.
  \end{proof}

  \item \textbf{Proofs by Induction}
  \begin{enumerate}
    \item Prove that $1^3 + 2^3 + \dots + n^3 = \left(\frac{n(n+1)}{2}\right)^2$.
    \begin{proof}
      Let $P_n$ be the statement that $1^3 + \dots = \left(\frac{n(n+1)}{2}\right)^2$. $P_0$ holds as $1 = \frac{(1)(2)}{2}^2 = 1$. Let $P_n$ hold: \[1^3 + 2^3 + \dots + n^3 = \left(\frac{n(n+1)}{2}\right)^2\]
      Adding $(n+1)^3$ to both sides:
      \[
        1^3 + 2^3 + \dots + n^3 + (n+1)^3 = \left(\frac{n(n+1)}{2}\right)^2 + (n+1)^3
      \]
      Focusing on the RHS:
      \begin{align*}
        \left(\frac{n(n+1)}{2}\right)^2 + (n+1)^3 &= (n+1)^2\left(\frac{n^2}{4} + (n+1)\right)\\
        &= (n+1)^2\left(\frac{n^2 + 4n + 4}{4}\right)\\
        &= (n+1)^2\left(\frac{(n+2)^2}{4}\right)\\
        &= \left(\frac{(n+1)(n+2)}{2}\right)^2
        & \equiv P_{n+1}
      \end{align*}
      Thus, by AI, $P_n$ holds for all $n \in \mathbb{N}$.
    \end{proof}
    \item We have an $8\times 8$ checker board. We remove the top-left and bottom-right squares. Prove that the remaining board cannot be covered by $2\times 1$ dominoes.
    \begin{proof}[Proof]
      Note that every domino must cover a black square and a white square. However, the board is missing 2 white squares (say). Thus, there are 62 squares (32 black, 30 white), and we would need \emph{exactly} 31 dominos (62/2). Each requires 1 black, 1 white tile, and thus we will run out of white squares before we reach our 31 dominos, and thus we cannot cover the board.
    \end{proof}
    \item Take $F_n$ to represent the $n$th Fibonacci number. Let $\varphi = \frac{1+\sqrt{5}}{2}$. Show that $F_n > \varphi^{n-2} \forall n \geq 3$.
    \begin{proof}[Proof]
      Let $P_n$ represent the "truth" of the given statement. $P_3: F_3 = F_2 + F_1 = 1 + 1 =2$. $\varphi^{1} = \varphi$; clearly $2 > \frac{1+\sqrt{5}}{2}$. Note that we should also prove $P_4, P_5$ for use in our induction.

      $P_4: (\frac{1+\sqrt{5}}{2})^2 = \frac{1 + 2 \sqrt{5} + 5}{4} = \frac{6+2\sqrt{5}}{4} < 3$.

      $P_5: (\frac{1+\sqrt{5}{2}})^3 \dots < 5$

      Take $P_{n-1}, P_{n}$ to hold, ie $F_{n-1}>\varphi^{n-3}$ and $F_n > \varphi^{n-2}$.
      \begin{align*}
        F_{n+1} = F_n + F_{n-1} &> \varphi^{n-2} + \varphi^{n-3}\\
        &= \varphi^{n-3}(\underbrace{\varphi+1}_{=\varphi^2})\\
        &= \varphi^{n-1},
      \end{align*} as desired, Noting that $\varphi + 1 = \frac{1+\sqrt{5}}{2} + 1 = \frac{1 + \sqrt{5}+2}{2} = \dots \varphi^2$.
    \end{proof}
    \item $a_1 = 1, a_2 = 8, a_n = a_{n-1} + 2 a_{n-2}$. Prove $a_n = 3 \cdot 2^{n-1} + 2 (-1)^n$.
    \begin{proof}[Proof]
      $a_1 = 1 =  3 \cdot 2^0 + 2(-1)^1 =3 -2 = 1$
      $a_2 = 8 = 3\cdot 2^1 + 2(-1)^2 = 6+2 = 8$
      So, $P_1, P_2$ holds. Assume $P_n$, $P_{n+1}$ holds. Then, we have $a_n = 3 \cdot 2^{n-1} + 2(-1)^n$ and so:
      \begin{align*}
        a_{n+1} &= 3 \cdot 2 ^{n-1} + 2(-1)^{n} + 2 \cdot \left(3\cdot 2^{n-2}+2(-1)^{n-1}\right)\\
        &= \dots = 3\cdot 2^n + 2(-1)^{n+1}
      \end{align*}
      Thus, proven.
    \end{proof}
  \end{enumerate}
  \item Show $A \setminus (B \setminus A) = A$.
  \begin{proof}[Proof]
    Let $x \in A \setminus(B \setminus A)$. $x$ must be in $A$, but not $B \setminus A$. Thus, $x$ is in $A$, but not in $B$. Thus, LHS $\subseteq$ RHS.

    Let $x \in A$. Thus, $x \notin B \setminus A$, and thus $x \in A \setminus (B\setminus A)$, and so $A \subseteq A \setminus (B \setminus A)$. Thus, LHS = RHS.

  \end{proof}

  \item $A_n = \{nk : k \in \mathbb{N}\}, n \geq 2$. Find $\bigcup_{n=2}^\infty An \bigcap_{n=2}^\infty A_n$.
  \begin{proof}
    \begin{align*}
      \bigcup_{n=2}^\infty A_n & = \bigcup \{2k, 3k, 4k, \dots\} = \{n : n \geq 2, n \in \mathbb{N}\} = \mathbb{N}\setminus \{1\}\\
      \bigcap_{n=2}^\infty A_n &= \varnothing \textit{ consider just } n=2, n=3 \text{ cases...}
    \end{align*}
  \end{proof}
\end{enumerate}

\subsection{Important}
\begin{figure*}[!ht]
  \centering
  \includegraphics[width=0.8\textwidth]{cardinal.png}
  \caption{Important!}
  \label{fig:important}
\end{figure*}

\end{document}