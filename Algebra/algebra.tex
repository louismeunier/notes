\documentclass[12pt]{article}
\usepackage{amsthm}
\usepackage{libertine}
\usepackage[margin=0.15in]{geometry}
\usepackage{amsmath,amssymb}
\usepackage{multicol}
\usepackage[shortlabels]{enumitem}
\usepackage{siunitx}
\usepackage{setspace}
\usepackage{cancel}
\usepackage{graphicx}
\usepackage{pgfplots}
\usepackage{listings}
\usepackage{tabularx}
\usepackage{titlesec}
\usepackage{thmtools}
\usepackage{thm-restate}
\usepackage[colorlinks=true, linkcolor=darkgray]{hyperref}
\usepackage{cleveref}
\usepackage[]{csquotes}
\usepackage[createShortEnv]{proof-at-the-end}

% makes theorems, definitions, etc. "restatable" as shown
% can add more with same format as you wish
\renewcommand*{\proofname}{}

\declaretheorem[thmbox=S,name=Definition,numberwithin=section]{definition}
\declaretheorem[thmbox=S,name=Axiom,numberwithin=section]{axiom}
\declaretheorem[thmbox=S,name=Lemma,numberwithin=section]{lemma}
\declaretheorem[thmbox=S,name=Theorem,numberwithin=section]{theorem}
\declaretheorem[name=Example,numberwithin=section]{example}
\declaretheorem[name=Exercise,numberwithin=section]{exercise}

\newEndThm[no proof here, restate, text proof={Examples}, one big link={\emph{See more}}]{definitionEnd}{definition}
\newEndThm[no proof here, restate, one big link = {\emph{Proof}}]{lemmaEnd}{lemma}
\newEndThm[no proof here, restate, one big link = {\emph{(Solution)}}]{exerciseEnd}{exercise}
\newEndThm[no proof here, restate]{axiomEnd}{axiom}

% \newEndThm[proof here, restate]{theoremEnd}{theorem}

% makes "quoted" text actually look correct
\MakeOuterQuote{"}

% page footer
\newpagestyle{mypage}{%
    \footrule
    \setfoot{\small\textcolor{gray}{§\thesubsection}}{\small\textcolor{gray}{\textit{\sectiontitle: \textbf{\subsectiontitle}}}}{\textcolor{gray}{\small p. \thepage}}
}

% title page settings
\newcommand{\pageauthor}{Louis Meunier}
\newcommand{\pagetitle}{Algebra I, II}
\newcommand{\pagesubtitle}{MATH235}

% black square for qed symbol
\renewcommand{\qedsymbol}{$\blacksquare$}

\titleformat{\section}
{\centering\normalfont\Large\bfseries}
{\thesection}{1em}{}

\begin{document}
\setstretch{2.25}
\noindent
\begin{center}
    \begin{tabularx}{\textwidth} { 
        >{\raggedright\arraybackslash}X 
        >{\raggedleft\arraybackslash}X}
    \LARGE \pageauthor \\
    \LARGE \textbf{\pagetitle} & \LARGE \textbf{\pagesubtitle}\\
    \end{tabularx}\\
    \rule[2ex]{0.8\textwidth}{1pt}
\end{center}

\setstretch{1.5}
\tableofcontents

% "enables" footer with section+subsection, etc. just comment it out if you don't want it
\pagestyle{mypage}

% makes sections a very dark gray + centered
\titleformat{\section}
{\color{darkgray}\centering\normalfont\Large\bfseries}
{\color{darkgray}\thesection}{1em}{}

% need to change margins and such here for rest of document
% kind of messy but what can you do
\newpage
% modify these as you wish
\newgeometry{margin=0.5in, top=0.4in, bottom=0.75in}
\parskip=0.5em

\section{Fundamentals}
\subsection{Sets}
\subsubsection{Definition}
A \textbf{set} can be considered as a collection of elements; more intuitively, you can consider something a set if you can determine wheteher a given object belongs to it. Typically sets are defined as $A = \{1, 2, \dots\}$, by a property $A = \{x \,|\, x\%2 = 0\}$, or with an appropriate verbal description.

\subsubsection{Set Operations}
There are a number of ways to "combine" sets:
\begin{itemize}
  \item \textbf{Union}: $A \cup B = \{x \,|\, x \in A \text{ or } x \in B\}$
  \item \textbf{Intersection}: $A \cap B = \{x \,|\, x \in A \text{ and } x \in B\}$
  \item \textbf{Difference}: $A \setminus B = \{x \,|\, x \in A \text{ and } x \notin B\}$
\end{itemize}

\begin{lemma}
  $A = (A\setminus B)\cup(A\cap B)$
\end{lemma}
\begin{proof}[Proof]
  To prove set equivalencies, we must prove that both RHS $\subseteq$ LHS and LHS $\subseteq$ RHS; meaning, the LHS and RHS are subsets of each other, and are thus equal.
  
  First, to prove LHS $\subseteq$ RHS, let $a \in A$. If $a \notin B$, then $a \in A\setminus B$, and $a \in$ RHS. Else, if $a \in B$, then $a \in A \cap B$ and $a \in$ RHS. Thus, LHS $\subseteq$ RHS.

  Next, to prove RHS $\subseteq$ LHS, let $a \in $ RHS. If $a \in A \setminus B$, then $a \in A=$ LHS. Else, $a \in A \cap B$, and thus $a \in A=$ LHS. Thus, RHS $\subseteq$ LHS.
  Since LHS $\subseteq$ RHS and RHS $\subseteq$ LHS, LHS = RHS.
  \qed
\end{proof}

\subsubsection{Indexed Sets}

Let $I$ be a set. If for every $i \in I$, we have a set $B_i$, we say that we have a \textit{collection} of sets $B_i$ indexed by $I$. We write $\{B_i : i \in I\}$.

\begin{example}
  Let $I = \{1, 2, 3\}$, and $B_i = \{1,2,3,4\}\setminus \{i\}$ ($B_i$ is the set of all numbers from 1 to 4, excluding $i$), for $i \in I$. We thus have $B_1 = \{2, 3, 4\}$ (etc.).

  This concept of indexing allows us to introduce repeated unions/intersections. For instance, we can write \[\bigcup_{i \in I} B_i = B_1 \cup B_2 \cup B_3 = \{1,2,3,4\}.\]
  Similarly, \[\bigcap_{i \in I} B_i = \{4\}.\footnotemark\]
\end{example}
\footnotetext{You can somewhat consider these "large" unions/intersections as analogous to summations $\Sigma$ and products $\Pi$.}

\begin{example}
  Let $I = \mathbb{R}$, and $B_i = [i, \infty] = \{r \in \mathbb{R} : r \geq i\}$. Then, $\bigcup_{i \in \mathbb{R}} B_i = \mathbb{R}$ and $\bigcap_{i \in \mathbb{R}} B_i = \emptyset$.
\end{example}
\newpage
\section{Appendix}
% \printProofs

\end{document}