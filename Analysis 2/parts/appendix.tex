\newpage
\section{Appendix}

\subsection{Notes from Tutorials}

\begin{theorem}
    Let $(X,d)$ be a compact metric space.\footnotemark  Let $C(X) := \{ f : X \to \mathbb{R} : f \text{ continuous}\}$ be a vector space. Take the uniform norm $\norm{f} := \sup_{x \in X} \abs{f(x)}$ on $C(x)$. Then, $(C(x), \norm{\bullet})$ is complete.\footnotemark
\end{theorem}

\footnotetext{In this proof, the compactness is necessary for the norm to be well-defined.}

\footnotetext{In this way, this becomes a Banach Space: a complete, normed vector space.}

\begin{proof}
    Denote the "canonical norm" $\rho(f, g) := \norm{f - g}$.

    Let $(f_n) \in C(X)$ be a Cauchy sequence. Then, $\forall \epsilon > 0, \exists N \in \mathbb{N} : \forall m, n \geq N, \rho(f_n, f_m) < \epsilon$.
    
    Fix $x \in X$, noting that \[
    \abs{f_n(x) - f_m(x)} \leq \sup_{y\in X} \abs{f_n(y) - f_m(y)} = \rho(f_n, f_m) < \epsilon. \quad \ast^1
    \]
    Define, for this fixed $x$, a sequence \emph{in $\mathbb{R}$} $\{f_n(x)\}_{n \in \mathbb{N}}$. By $\ast^1$, we have that this sequence is Cauchy in $\mathbb{R}$, but as $\mathbb{R}$ complete, $f_n(x)$ hence converges, to some limit we call $f(x) := \lim_{n\to \infty} f_n(x)$. Note that $x$ is still fixed at this point; these are but real numbers we are working with here.

    Now, as $x$ was completely arbitrary, we can repeat this process for all of $X$, and define a function $f : X \to \mathbb{R}$ where $f(x) := \lim_{n \to \infty} f_n(x)$.

    For a fixed $x$, we have that $f_m(x) \to f(x)$ as $m \to \infty$. This implies:
    \begin{align*}
        0 \leq \lim_{m \to \infty} \abs{f_n(x) - f_m(x)}& \leq \lim_{m \to \infty} \epsilon = \epsilon\\
        &\implies \abs{f_n(x) - f(x)} \leq \epsilon \forall n \geq N\\
        &\implies \rho(f_n, f) = \sup_{x \in X}\abs{f_n(x) - f(x)} \leq \epsilon \implies f_n \to f
    \end{align*}
    It remains to show that $f \in C(X)$. Let $c \in X$ and $\epsilon > 0$, and the corresponding $N \in \mathbb{N} : \rho(f_n, f) < \frac{\epsilon}{3} \forall n \geq N$. By construction, $f_N \in C(X)$, and is thus continuous at $c$. This gives that $\exists \delta > 0 : \abs{f_N(x) - f_N(c)} < \frac{\epsilon}{3}$ whenever $d(x, c) < \delta$. \footnote{Be careful here, there are three different metrics going on; $\rho$ from the vector space, $d$ from the underlying metric space, and $\abs{\cdots}$ from $\mathbb{R}$.}

    Hence, if $d(x, c) < \delta$, we have \begin{align*}
        \abs{f(x) - f(c)} \leq \abs{f(x) - f_N(x)} + \abs{f_N(x) - f_N(c)} + \abs{f_N(c) - f(c)}\\
        \leq \rho(f, f_N) + \frac{\epsilon}{3} + \frac{\epsilon}{3}\\
        < \frac{\epsilon}{3} + \frac{\epsilon}{3} + \frac{\epsilon}{3} = \epsilon,
    \end{align*}
   hence $f$ continuous at $c$, which was completely arbitrary, and thus $f \in C(X)$.
\end{proof}

\begin{theorem}
    Let $(X,d)$-complete. Let $\{F_n\}$ be a decreasing family of non-empty closed sets with $\lim_{n \to \infty} \diam(F_n) = 0$. Then, $\exists z:$ $\bigcap_{n \in \mathbb{N}} F_n = \{z\}$.
\end{theorem}

\begin{theorem}
    Let $(X, d)$-complete, and $f : X \to X$ an "expanding map", such that $d(x, y) \leq d(f(x), f(y)) \forall x, y \in X$. Then, $f$ is a surjective isometry, ie, $f(X) = X$ and $d(f(x), f(y)) = d(x, y) \forall x, y \in X$.
\end{theorem}

\begin{lemma}
    Differentiable $\implies$ Continuous.
\end{lemma}

\begin{proof}
    Let $f: I \to \mathbb{R}$, and $c \in I$ arbitrary. Notice that $\forall x \neq c \in I$, $f(x) - f(c) = (x-c) \frac{f(x)-f(c)}{x-c}$. Hence, \begin{align*}
        \lim_{ x \to c} (f(x) - f(c)) &= \lim_{ x \to c}(x-c) \frac{f(x)-f(c)}{x-c}\\
        &= \lim_{x \to c} (x- c) \cdot \lim_{x \to c} \frac{f(x) - f(c)}{x-c}\\
        &= 0 \cdot f'(x) = 0\\
        &\implies \lim_{x \to c} f(x) = f(c),
    \end{align*}
    hence $f$ continuous, noting that the splitting of the limits is valid as both are defined.
\end{proof}

\begin{example}
    Let $f: \mathbb{R} \to \mathbb{R}, f(x) := \begin{cases}
        x^2 & x \in \mathbb{Q}\\
        0 & x \notin \mathbb{Q}
    \end{cases}$

    \noindent\underline{Claim:} $f$ discontinuous at all $x \neq 0$.

    \begin{proof}
        Let $x \neq 0 \in \mathbb{R}$. By density of $\mathbb{Q} \subseteq \mathbb{R}$, there exist sequences $(r_n) \in \mathbb{Q}$ s.t. $r_n \to x$ and $(z_n) \in \mathbb{R} \setminus \mathbb{Q}$ s.t. $z_n \to x$. Then:\begin{align*}
            \lim_{n \to \infty} f(r_n) = \lim r_n^2 = x^2\\
            \lim_{n \to \infty} f(z_n) = \lim 0,
        \end{align*}
        hence $f$ discontinuous by the sequential criterion at $x \neq 0$.
    \end{proof}
    
    \noindent\underline{Claim:} $f'(0) = 0$. 
    \begin{proof}
        Let $\epsilon > 0$ and $\delta = \epsilon$. Notice that $f(x) \leq x^2 \forall x$. Then, we have that $\forall \abs{x} < \delta$, \begin{align*}
            \abs{\frac{f(x) - f(0)}{x-0} - 0} &= \abs{\frac{f(x)}{x}} \\
            &\leq \abs{\frac{x^2}{x}} = \abs{x} < \delta = \epsilon 
        \end{align*}
    \end{proof}
\end{example}

\begin{definition}
    Let $f: I \to \mathbb{R}$. A point $c \in I$ is a local max (resp min) if $\exists \delta > 0 \st f(x) \leq f(c)$ (resp $f(x) \geq f(c)$) $\forall x \in (c - \delta, c + \delta) \cap I$.
\end{definition}

\begin{lemma}
    Let $f : I \to \mathbb{R}$ be differentiable at $c \in I^\circ$. If $c$ a local extrema of $f$, then $f'(c) = 0$.
\end{lemma}
\begin{proof}
    Assume wlog that $c$ a local max; if a local min, take $\tilde f :=-f$ and continue.

    Since $I^\circ$ open, $\exists \delta_1 > 0 : (c- \delta_1, c + \delta_1) \subseteq I^\circ \subseteq I$. We also have that $\exists \delta_2 > 0 : f(x) \leq f(c) \forall x \in (c - \delta_2, c + \delta_2) \cap I$, by $c$ an extrema.

    Let $\delta := \min \{\delta_1, \delta_2\}$. Then, we have both $(c - \delta, c + \delta) \subseteq I$ and $f(x) \leq f(c) \forall x \in (c - \delta, c + \delta)$. 
    
    Since $f'(c)$ exists, $\lim_{x \to c^+} \frac{f(x) - f(c)}{x-c} = \lim_{x \to c^-} \frac{f(x) - f(c)}{x - c}$. But we have from the property of being a maximum that \begin{align*}
        \lim_{x \to c^+} \frac{f(x) - f(c)}{x-c} \geq 0, \qquad \lim_{x \to c^-} \frac{f(x) - f(c)}{x - c}\leq 0,
    \end{align*}
    hence, as these two limits must agree, they must equal $0$ and thus $f'(c) = 0$.
\end{proof}

\subsection{Miscellaneous}

\begin{example}[Rudin, Chapter 7: Differentiability]
    \begin{enumerate}
        \item Let $f$ be defined $\forall x \in \mathbb{R}$, and suppose that $\abs{f(x) - f(y)} \leq (x - y)^2, \forall x, y \in \mathbb{R}$. Prove that $f$ is constant.\footnotemark
        \begin{proof}
            Let $x > y \in \mathbb{R}$. Then, as $\abs{x - y} = x - y$, we have \begin{align*}
                \abs{f(x) - f(y)} \leq (x - y)^2 &\implies \abs{\frac{f(x) - f(y)}{x - y}} \leq x - y = \abs{x - y} \to 0 \text{ as } y \to x\\
                &\implies \abs{\frac{f(x) - f(y)}{x - y}} \to 0
            \end{align*}
            This implies, then, that $f'(x)$ is defined $\forall x \in \mathbb{R}$, and moreover, that $f'(x) = 0 \forall x \in \mathbb{R}$. We conclude, then, that $f(x)$ constant $\forall x \in \mathbb{R}$.
        \end{proof}
    
        \item Suppose $f'(x) > 0$ in $(a, b)$. Prove that $f$ is strictly increasing in $(a, b)$, and let $g$ be its inverse function. Prove that $g$ is differentiable, and that \[
        g'(f(x)) = \frac{1}{f'(x)}   \quad (a < x < b).
        \]
        \begin{proof}
            Fix $x > y \in (a, b)$. Then, by the mean value theorem, $\exists z \in (x, y) : f'(z) = \frac{f(x) - f(y)}{x - y}$. Since $f'(z) > 0$, it follows that \begin{align*}
                \frac{f(x) - f(y)}{x - y} > 0 \implies f(x) - f(y) > x - y > 0 \implies f(x) > f(y),
            \end{align*}
            hence, $f$ increasing, as $x > y$ arbitrary.

            Let now $g := f^{-1}$.
            % TODO
        \end{proof}
    \end{enumerate}
\end{example}

\footnotetext{Note that this means that $f$ \emph{Hölder continuous} with constant $\alpha = 2$. Indeed, Hölder continuous functions with $\alpha > 1$ are always constant by a similar proof. For $0 < \alpha \leq 1$, we have the inclusion continuously differentiable $\implies$ Lipschitz $\implies$ $\alpha-$Hölder $\implies$ uniformly continuous $\implies$ continuous.}