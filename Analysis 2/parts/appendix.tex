\newpage
\section{Appendix}

\subsection{Notes from Tutorials}

\begin{theorem}
    Let $(X,d)$ be a compact metric space.\footnotemark  Let $C(X) := \{ f : X \to \mathbb{R} : f \text{ continuous}\}$ be a vector space. Take the uniform norm $\norm{f} := \sup_{x \in X} \abs{f(x)}$ on $C(x)$. Then, $(C(x), \norm{\bullet})$ is complete.\footnotemark
\end{theorem}

\footnotetext{In this proof, the compactness is necessary for the norm to be well-defined.}

\footnotetext{In this way, this becomes a Banach Space: a complete, normed vector space.}

\begin{proof}
    Denote the "canonical norm" $\rho(f, g) := \norm{f - g}$.

    Let $(f_n) \in C(X)$ be a Cauchy sequence. Then, $\forall \epsilon > 0, \exists N \in \mathbb{N} : \forall m, n \geq N, \rho(f_n, f_m) < \epsilon$.
    
    Fix $x \in X$, noting that \[
    \abs{f_n(x) - f_m(x)} \leq \sup_{y\in X} \abs{f_n(y) - f_m(y)} = \rho(f_n, f_m) < \epsilon. \quad \ast^1
    \]
    Define, for this fixed $x$, a sequence \emph{in $\mathbb{R}$} $\{f_n(x)\}_{n \in \mathbb{N}}$. By $\ast^1$, we have that this sequence is Cauchy in $\mathbb{R}$, but as $\mathbb{R}$ complete, $f_n(x)$ hence converges, to some limit we call $f(x) := \lim_{n\to \infty} f_n(x)$. Note that $x$ is still fixed at this point; these are but real numbers we are working with here.

    Now, as $x$ was completely arbitrary, we can repeat this process for all of $X$, and define a function $f : X \to \mathbb{R}$ where $f(x) := \lim_{n \to \infty} f_n(x)$.

    For a fixed $x$, we have that $f_m(x) \to f(x)$ as $m \to \infty$. This implies:
    \begin{align*}
        0 \leq \lim_{m \to \infty} \abs{f_n(x) - f_m(x)}& \leq \lim_{m \to \infty} \epsilon = \epsilon\\
        &\implies \abs{f_n(x) - f(x)} \leq \epsilon \forall n \geq N\\
        &\implies \rho(f_n, f) = \sup_{x \in X}\abs{f_n(x) - f(x)} \leq \epsilon \implies f_n \to f
    \end{align*}
    It remains to show that $f \in C(X)$. Let $c \in X$ and $\epsilon > 0$, and the corresponding $N \in \mathbb{N} : \rho(f_n, f) < \frac{\epsilon}{3} \forall n \geq N$. By construction, $f_N \in C(X)$, and is thus continuous at $c$. This gives that $\exists \delta > 0 : \abs{f_N(x) - f_N(c)} < \frac{\epsilon}{3}$ whenever $d(x, c) < \delta$. \footnote{Be careful here, there are three different metrics going on; $\rho$ from the vector space, $d$ from the underlying metric space, and $\abs{\cdots}$ from $\mathbb{R}$.}

    Hence, if $d(x, c) < \delta$, we have \begin{align*}
        \abs{f(x) - f(c)} \leq \abs{f(x) - f_N(x)} + \abs{f_N(x) - f_N(c)} + \abs{f_N(c) - f(c)}\\
        \leq \rho(f, f_N) + \frac{\epsilon}{3} + \frac{\epsilon}{3}\\
        < \frac{\epsilon}{3} + \frac{\epsilon}{3} + \frac{\epsilon}{3} = \epsilon,
    \end{align*}
   hence $f$ continuous at $c$, which was completely arbitrary, and thus $f \in C(X)$.
\end{proof}

\begin{theorem}
    Let $(X,d)$-complete. Let $\{F_n\}$ be a decreasing family of non-empty closed sets with $\lim_{n \to \infty} \diam(F_n) = 0$. Then, $\exists z:$ $\bigcap_{n \in \mathbb{N}} F_n = \{z\}$.
\end{theorem}

\begin{theorem}
    Let $(X, d)$-complete, and $f : X \to X$ an "expanding map", such that $d(x, y) \leq d(f(x), f(y)) \forall x, y \in X$. Then, $f$ is a surjective isometry, ie, $f(X) = X$ and $d(f(x), f(y)) = d(x, y) \forall x, y \in X$.
\end{theorem}

\begin{lemma}
    Differentiable $\implies$ Continuous.
\end{lemma}

\begin{proof}
    Let $f: I \to \mathbb{R}$, and $c \in I$ arbitrary. Notice that $\forall x \neq c \in I$, $f(x) - f(c) = (x-c) \frac{f(x)-f(c)}{x-c}$. Hence, \begin{align*}
        \lim_{ x \to c} (f(x) - f(c)) &= \lim_{ x \to c}(x-c) \frac{f(x)-f(c)}{x-c}\\
        &= \lim_{x \to c} (x- c) \cdot \lim_{x \to c} \frac{f(x) - f(c)}{x-c}\\
        &= 0 \cdot f'(x) = 0\\
        &\implies \lim_{x \to c} f(x) = f(c),
    \end{align*}
    hence $f$ continuous, noting that the splitting of the limits is valid as both are defined.
\end{proof}

\begin{example}
    Let $f: \mathbb{R} \to \mathbb{R}, f(x) := \begin{cases}
        x^2 & x \in \mathbb{Q}\\
        0 & x \notin \mathbb{Q}
    \end{cases}$

    \noindent\myuline{Claim:} $f$ discontinuous at all $x \neq 0$.

    \begin{proof}
        Let $x \neq 0 \in \mathbb{R}$. By density of $\mathbb{Q} \subseteq \mathbb{R}$, there exist sequences $(r_n) \in \mathbb{Q}$ s.t. $r_n \to x$ and $(z_n) \in \mathbb{R} \setminus \mathbb{Q}$ s.t. $z_n \to x$. Then:\begin{align*}
            \lim_{n \to \infty} f(r_n) = \lim r_n^2 = x^2\\
            \lim_{n \to \infty} f(z_n) = \lim 0,
        \end{align*}
        hence $f$ discontinuous by the sequential criterion at $x \neq 0$.
    \end{proof}
    
    \noindent\myuline{Claim:} $f'(0) = 0$. 
    \begin{proof}
        Let $\epsilon > 0$ and $\delta = \epsilon$. Notice that $f(x) \leq x^2 \forall x$. Then, we have that $\forall \abs{x} < \delta$, \begin{align*}
            \abs{\frac{f(x) - f(0)}{x-0} - 0} &= \abs{\frac{f(x)}{x}} \\
            &\leq \abs{\frac{x^2}{x}} = \abs{x} < \delta = \epsilon 
        \end{align*}
    \end{proof}
\end{example}

\begin{definition}
    Let $f: I \to \mathbb{R}$. A point $c \in I$ is a local max (resp min) if $\exists \delta > 0 \st f(x) \leq f(c)$ (resp $f(x) \geq f(c)$) $\forall x \in (c - \delta, c + \delta) \cap I$.
\end{definition}

\begin{lemma}
    Let $f : I \to \mathbb{R}$ be differentiable at $c \in I^\circ$. If $c$ a local extrema of $f$, then $f'(c) = 0$.
\end{lemma}
\begin{proof}
    Assume wlog that $c$ a local max; if a local min, take $\tilde f :=-f$ and continue.

    Since $I^\circ$ open, $\exists \delta_1 > 0 : (c- \delta_1, c + \delta_1) \subseteq I^\circ \subseteq I$. We also have that $\exists \delta_2 > 0 : f(x) \leq f(c) \forall x \in (c - \delta_2, c + \delta_2) \cap I$, by $c$ an extrema.

    Let $\delta := \min \{\delta_1, \delta_2\}$. Then, we have both $(c - \delta, c + \delta) \subseteq I$ and $f(x) \leq f(c) \forall x \in (c - \delta, c + \delta)$. 
    
    Since $f'(c)$ exists, $\lim_{x \to c^+} \frac{f(x) - f(c)}{x-c} = \lim_{x \to c^-} \frac{f(x) - f(c)}{x - c}$. But we have from the property of being a maximum that \begin{align*}
        \lim_{x \to c^+} \frac{f(x) - f(c)}{x-c} \geq 0, \qquad \lim_{x \to c^-} \frac{f(x) - f(c)}{x - c}\leq 0,
    \end{align*}
    hence, as these two limits must agree, they must equal $0$ and thus $f'(c) = 0$.
\end{proof}

\subsection{Miscellaneous}

\begin{example}[Rudin, Chapter 7: Differentiability]
    \begin{enumerate}
        \item Let $f$ be defined $\forall x \in \mathbb{R}$, and suppose that $\abs{f(x) - f(y)} \leq (x - y)^2, \forall x, y \in \mathbb{R}$. Prove that $f$ is constant.\footnotemark
        \begin{proof}
            Let $x > y \in \mathbb{R}$. Then, as $\abs{x - y} = x - y$, we have \begin{align*}
                \abs{f(x) - f(y)} \leq (x - y)^2 &\implies \abs{\frac{f(x) - f(y)}{x - y}} \leq x - y = \abs{x - y} \to 0 \text{ as } y \to x\\
                &\implies \abs{\frac{f(x) - f(y)}{x - y}} \to 0
            \end{align*}
            This implies, then, that $f'(x)$ is defined $\forall x \in \mathbb{R}$, and moreover, that $f'(x) = 0 \forall x \in \mathbb{R}$. We conclude, then, that $f(x)$ constant $\forall x \in \mathbb{R}$.
        \end{proof}
    
        \item Suppose $f'(x) > 0$ in $(a, b)$. Prove that $f$ is strictly increasing in $(a, b)$, and let $g$ be its inverse function. Prove that $g$ is differentiable, and that \[
        g'(f(x)) = \frac{1}{f'(x)}   \quad (a < x < b).
        \]
        \begin{proof}
            Fix $x > y \in (a, b)$. Then, by the mean value theorem, $\exists z \in (x, y) : f'(z) = \frac{f(x) - f(y)}{x - y}$. Since $f'(z) > 0$, it follows that \begin{align*}
                \frac{f(x) - f(y)}{x - y} > 0 \implies f(x) - f(y) > x - y > 0 \implies f(x) > f(y),
            \end{align*}
            hence, $f$ increasing, as $x > y$ arbitrary.

            Let now $g := f^{-1}$.
            % TODO
        \end{proof}
    \end{enumerate}
\end{example}

\footnotetext{Note that this means that $f$ \emph{Hölder continuous} with constant $\alpha = 2$. Indeed, Hölder continuous functions with $\alpha > 1$ are always constant by a similar proof. For $0 < \alpha \leq 1$, we have the inclusion continuously differentiable $\implies$ Lipschitz $\implies$ $\alpha-$Hölder $\implies$ uniformly continuous $\implies$ continuous.}

\subsection{Class Midterm Solutions}

\begin{question}
    Let $X$ be a topological space, and let $f, g : X \to \R$ be two continuous functions. Show that the set $\{x \in X : f(x) > g(x)\}$ is an open subset of $X$.

    \begin{proof}
        Let $A \defeq \{x \in X : f(x) > g(x)\}$. Letting $\varphi(x) \defeq f(x) - g(x) = (f-g)(x)$, then remark that $A \equiv \{x \in X : \varphi(x) > 0\}$, and since differences of continuous functions are continuous, $\varphi$ continuous. Letting $B \defeq (0, \infty) \subseteq \mathbb{R}$, then, we have that $A = \varphi^{-1}(B)$. But $B$ an open set, and the inverse images of open sets by continuous functions are open, hence $A$ open.
    \end{proof}
\end{question}


\begin{question}
\begin{enumerate}[label=(\alph*)]
    \item List three equivalent properties (definitions) of compact sets in metric spaces; you don't need to prove anything.
    \item Is the unit ball\footnotemark in the space $\ell^2$ of infinite sequences compact? Prove or disprove. You may use any of the properties from (a).
\end{enumerate}
\begin{proof}
    \begin{enumerate}[label=(\alph*)]
        \item Every open cover admits a finite subcover $\iff$ sequentially compact $\iff$ complete and totally bounded.
        \item Denote the closed unit ball centered at $(0, 0, \dots)$ in $\ell^2$, $B \defeq \{x \in \ell^2 : d_2^2(0, x) = \sum_{i=1}^\infty\abs{x_i} \leq 1 \}$. Consider the sequence of "unit sequences" \[
            \{e^n\}_{n \in \mathbb{N}} \in B, \qquad e^n_i  \defeq \delta_{in}.
        \]
        Then, for any $i \neq j$, $d_2(e^{n}, e^m) = \sqrt{2} > 1$. It follows that, although $e^n \in B$ for any $n$, there cannot exist a subsequence of $x^n$ that converges within $B$ (verify why this is!). Thus, $B$ cannot be sequentially compact and thus not compact.
    \end{enumerate}
\end{proof}
\end{question}
\footnotetext{Jakobson said in class this is supposed to be a closed ball.}

\begin{question}
    \begin{enumerate}[label=(\alph*)]
        \item Define a complete metric space.
        \item State (without proof) the contraction mapping theorem.
        \item Let $f : (0, 1) \to (0, 1)$ be defined by $f(x) = x/2$. Is $f$ a contraction?
        \item Does $f$ have a fixed point in the open interval $I = (0, 1)$? Does that contradict the contraction mapping theorem?
    \end{enumerate}

\begin{proof}
    \begin{enumerate}[label=(\alph*)]
        \item A complete metric space is a metric space in which every Cauchy sequence converges within that space.
        \item Let $(X, d)$ be a complete metric space, and let $f : X \to X$ be a contraction mapping, ie for any $x, y \in X$, $d(f(x),f(y)) \leq c \cdot d(x, y)$ for some $c \in (0, 1)$. Then, the contraction mapping states that $f$ has a unique fixed point $z \in X$, ie $f(z) = z$ and $\lim_{n\to \infty} f^{(n)}(x) = z$ for any $x$.
        \item For any $x, y \in (0, 1)$, we have \[
        d(f(x), f(y)) = \abs{f(x) - f(y)} = \abs{\frac{x-y}{2}} = \frac{1}{2} \abs{x - y} = c \cdot d(x, y),
        \]
        so $f$ indeed a contraction mapping with $c \defeq \frac{1}{2}$.
        \item We have that for any $x \in I$, $f^{(n)}(x) = \frac{x}{2^n}$ so $x$ a fixed point iff $\frac{x}{2^n} = \frac{x}{2^{n-1}}$ for some $n$, which is only possible if $x = 0$, but $0 \notin I$, so indeed $f$ has no fixed point in $I$. This is not a contradiction to the contraction mapping theorem since $I \defeq (0, 1)$ not complete (indeed, $\frac{1}{n} \in I \forall n$ but $\frac{1}{n} \to 0 \notin I$).
    \end{enumerate}
\end{proof}

\end{question}

\begin{question}
    Let $x = (x_1, x_2, \dots)$ and $y = (y_1, y_2, \dots)$ be infinite real sequences satisfying $\norm{x}_2 \leq 2$ and $\norm{y}_2 \leq 3$, where $\norm{\cdots}_2$ the $\ell^2$ norm.
    \begin{enumerate}[label=(\alph*)]
        \item State Holder's inequality and Minkowski inequality for sequences.
        \item Give an upper bound for $x \cdot y = \sum_{i} x_i y_i$, and for $\norm{x + y}$.
    \end{enumerate}
    \begin{proof}
        \begin{enumerate}[label=(\alph*)]
            \item Holder's inequality: for $p, q$ Holder conjugates and $x \in \ell^p, y \in \ell^q$ we have \[
           \abs{ \sum_{i=1} x_i y_i} \leq \norm{x}_p \norm{y}_q.
            \]
            Minkowski inequality: for $x, y \in \ell^p$, \[
            \norm{x + y}_p \leq \norm{x}_p + \norm{y}_p.    
            \]
            \item For $x, y$ as given; by Holders, $x \cdot y \leq \norm{x}_p \norm{y}_q = 2 \cdot 3 = 6$, and by Minkowski's, $\norm{x + y} \leq \norm{x} + \norm{y} = 2 + 3 = 5$, so $6, 5$ are upper bounds for $x \cdot y$, $\norm{x + y}$ respectively.
        \end{enumerate}
    \end{proof}
\end{question}


\begin{question}
    \begin{enumerate}[label=(\alph*)]
        \item State (without proof) Taylor's theorem.
        \item Let $f \in C^4([0, 2])$, and let $f'(1) = f''(1) = f'''(1) = 0$ while $f^{(4)}(1) = 2$. Use (a) to show that $f(x)$ has a local extremum at $x = 1$, and determine its type.
    \end{enumerate}

\begin{proof}
    \begin{enumerate}[label=(\alph*)]
        \item Let $I \defeq [a, b] \subseteq \R$ and let $f : I \to \R$ such that $f \in C^n(I)$, and $f^{(n+1)}(x)$ exists on $(a, b)$. Then, for $x_0 \in [a, b]$, there exists some $c \in (\min(x, x_0), \max(x, x_0))$ such that \[
        f(x) = f(x_0) + f'(x_0)(x-x_0) + \cdots + \frac{f^{(n)}(x_0)}{n!}(x-x_0)^n + \frac{f^{(n+1)}(c)}{(n+1)!}(x-x_0)^{n+1}.
        \]
        \item By Taylor's, for any $x \in [0, 2]$, there exists some $c$ between $x$ and $1$ such that \begin{align*}
            f(x) &= f(1)+\underbrace{f'(1)(\cdots) + f''(1)(\cdots) + f'''(1)(\cdots) }_{=0}+ \frac{f^{(4)}(c)}{4!}(x-1)^4\\
            &=f(1) + \frac{f^{(4)}(c)}{4!}(x-1)^4 \\
            &\implies f(x) - f(1) \geq \frac{f^{(4)}(c)}{4!}(x-1)^4 \forall x \in [0, 2]
        \end{align*}
        By continuity of $f^{(4)}$, there exists some neighborhood $V$ of $x_0 = 1$ such that $f^{(4)}(c)$ has the same sign of $f^{(4)}(1)$. So, for any $x \in V$, $\frac{f^{(4)}(c)}{4!} \geq 0$, since $\frac{f^{(4)(1)}}{4!} = \frac{2}{4!} \geq 0$. Thus, since $(x-1)^4 \geq 0$, we have that for such $x$ in $V$, \[
        f(x) - f(1)     \geq  0 \implies f(x) \geq f(1).
        \]
        Hence, we have a neighborhood of $1$ such that for all $x$ in the neighborhood $f(x) \geq f(1)$. It follows that $1$ a local minimum of $f$.
    \end{enumerate}
\end{proof}
\end{question}
