\subsection{Metrizability}

\begin{proposition}
    Different metrics can define the same topology.
\end{proposition}
\begin{example}
    \begin{enumerate}
        \item Different $\ell_p$ metrics in $\mathbb{R}^n$ (PSET 1)
        \item Let $(X, d)$ be a metric space. Then, \[
        \tilde{d}(x,y) := \frac{d(x,y)}{d(x,y)+1}
        \]
        is also a metric (the first two axioms are trivial), and defines the same topology. Note, moreover, that $\tilde{d}(x,y) \leq 1 \forall x, y$; this distance is bounded, and can often be more convenient to work with in particular contexts.
    \end{enumerate}
\end{example}

\begin{question}
    Suppose $(X_k, d_k)$ are metric spaces $\forall k \geq 1$. Then, we can define the product topology $\tau$ on  \[
        X := \prod_{k=1}^{\infty} X_k.    
        \]
        Does the product topology $\tau$ come from a metric? That is, is $\tau$ \emph{metrizable}?
\end{question}

\begin{remark}
    There do indeed exist examples of non-metrizable topological spaces; this question is indeed well-founded.
\end{remark}

\begin{proof}[Answer.]
    Let $\myuline{x}=(x_1, x_2, \dots, x_n, \dots), \myuline{y}=(y_1, y_2, \dots, y_n, \dots) \in \prod_{k=1}^\infty$ (where $x_i, y_i \in X_i$) be infinite sequences of elements. Then, for each metric space $X_k$ take the metric $$\tilde{d}_k(x_k, y_k) = \frac{d_k(x_k,y_k)}{1+d_k(x_k,y_k)}$$ (as in the example above). Then, we define \[
    D(\myuline{x}, \myuline{y}) = \sum_{k=1}^{\infty} \frac{\tilde{d}_k(x_k, y_k)}{2^k},
    \]
    noting that $D(\myuline{x}, \myuline{y}) \leq \sum_{k=1}^\infty \frac{1}{2^k} = 1$ (by our construction, "normalizing" each metric), hence this is a valid, \emph{converging} metric (which wouldn't otherwise be guaranteed if we didn't normalize the metrics). It remains to show whether this metric omits the same topology as $\tau$. 
\end{proof}

\subsection{Compactness, Connectedness}

\begin{definition}[Compact]
        A set $A$ in a topological space is said to be \emph{compact} if every cover has a finite subcover. That is, if \[
        A \subseteq \bigcup_{\alpha \in I} U_\alpha-\text{open},    
        \]
        then $\exists \{\alpha_1, \dots, \alpha_n \in I\}$ such that $A \subseteq \bigcup_{i=1}^n U_{\alpha_i}$.
\end{definition}

\begin{proposition}\label{prop:closedcompact}
    A closed interval $[a, b]$ is compact.
\end{proposition}

\begin{proof}
    If\footnote{This proof is adapted from that of Theorem 27.1 in Munkre's Topology, an identical theorem but applied to more general ordered topologies.} $a = b$, this is clear. Suppose $a < b$, and let $[a, b] \subseteq \bigcup_{i \in I} U_i =: \mathcal{U}$ be an arbitrary cover. Then, we proceed in the following steps:
    \begin{enumerate}
        \item \textbf{Claim:} Given $x \in [a, b], x \neq b$, $\exists y\in [a, b] \st [x, y]$ has a finite subcover.
        
        Let $x \in [a, b]$, $x \neq b$. Then, $\exists U_\alpha \in \mathcal{U} : x \in U_\alpha$. Since $U_\alpha$ open, and $x \neq b$, we further have that $\exists c \in [a, b] \st [x, c) \subseteq U_\alpha$. 

        Now, let $y \in (x, c)$; then, the interval $[x, y] \subseteq [x, c) \subseteq U_\alpha$, that is, $[x, y]$ has a finite subcover.

        \item Define $C := \{y \in [a, b]: y > a, [a, y] \text{ has a finite subcover}\}$. We note that 
        \begin{itemize}
            \item $C \neq \varnothing$; taking $x = a$ in Step 1. above, we have that $\exists y \in [a, b]$ such that $[a, y]$ has a finite step cover, so this $y \in C$.
            \item $C$ bounded; by construction, $\forall y \in C, a < y \leq c$.
        \end{itemize}
        Thus, we can validly define $c:= \sup C$, noting that $a < c \leq b$. Ultimately, we wish to prove that $c = b$, completing the proof that $[a, b]$ has a finite subcover.

        \item \textbf{Claim:} $c \in C$.
 
        Let $U_\beta \in \mathcal{U} : c \in U_\beta$. Then, by the openness of $U_\beta$, $\exists d \in [a, b] \st (d, c] \subseteq U_\beta$.

        Supposing $c \notin C$, then $\exists z \in C$ such that $z \in (d, c)$; if one did not exist, then this would imply that $d$ was a smaller upper bound that $c$, a contradiction. Thus, $[z , c] \subseteq (d, c] \subseteq U_\beta$.

        Moreover, we have that, given $z \in C$, $[a, z]$ has a finite subcover; call it $U_z \subseteq \mathcal{U}$. This gives, then: \[
        [a, c] = [a, z] \cup [z, c]  \subseteq U_z \cup U_\beta.
        \]
        But this is a finite subcover of $[a, c]$, contradicting the fact that $c \notin C$. We conclude, then, that $c \in C$ after all.

        \item \textbf{Claim:} $c = b$.

        Suppose not; then, since we have $c \leq b$, then assume $c < b$. Then, applying Step 1. with $x = c$ (which we can do, by our assumption of $c \neq b$!), then we have that $\exists y > c \st [c, y]$ has a finite subcover, call this $U_y \subseteq \mathcal{U}$. 
        
        Moreover, we had $c \in C$, hence $[a, c]$ has a finite subcover, call this $U_c \subseteq \mathcal{U}$.

        Then, this gives us that \[
        [a, y] = [a, c] \cup [c, y] \subseteq U_c \cup U_y,
        \]
        that is, $[a, y]$ has a finite subcover, and so $y \in C$. But recall that $y > c$; hence, this a contradiction to $c$ being the least upper bound of $C$. We conclude that $c = b$, and thus $[a, b]$ has a finite subcover, and is thus compact.
    \end{enumerate}
    % Then, suppose \[
    % [a, b] \subseteq \bigcup_{\alpha \in I} U_\alpha-\text{open}.    
    % \]
    % Then, it must be that $a \in U_{\alpha_1}$ for some $\alpha_1$; hence, $\exists \epsilon > 0$ $(a - \epsilon, a + \epsilon) \subseteq U_{\alpha_1}$. Let $c = a + \frac{\epsilon}{2}$, then $[a, c] \subseteq U_{\alpha_1}$. Then, $[a,c]$ has a finite subcover; it is covered by the single open set $U_{\alpha_1}$.

    % Let, then, $l = $ least upper bound for all $c\leq b \st [a, c]$ has a finite subcover. Then, $l \in [a,b]$ and so $\exists \beta \in I$ such that $l \in U_\beta$. It follows that $\exists \epsilon > 0 \st [l - \epsilon, l] \subseteq U_\beta$. 

    % By our definition of $l$, $\exists c_j \st d(l, c_j) < \frac{1}{j} \st [a, c_j]$ is contained in a finite union of $U_\alpha$'s. Let $c_j \in [l-\epsilon, l]$. Then, $[a, c_j] \subseteq \bigcup_{i=1}^{k} U_{\alpha_i}$, and then, the interval $[a, l] \subseteq \bigcup_{i=1}^{k}U_{\alpha_1} \cup U_{\beta}$. 
    
    % But we can "choose" $U_\beta$ such that $[l, l + \epsilon] \subseteq U_\beta$, and so we'd further have that $[a, l + \epsilon] \subseteq \bigcup_{i=1}^{k}U_{\alpha_1} \cup U_{\beta}$; but we have that $L + \epsilon > l$, and we have a contradiction to $l$ being the least upper bound for which this holds, unless $l = b$ itself. Then, we cannot "increase" further, and hence $l = b$ and so $[a, l] = [a, b] \subseteq$ a finite union of open sets.
\end{proof}

\begin{remark}
    A similar proof shows that $[a, b]$ is \emph{connected}; we cannot cover it by two disjoint open sets.
\end{remark}

\begin{theorem}[On Compactness]\label{thm:compactnessrn}
    Let $A \subseteq \mathbb{R}^n$. Then, $A$ compact $\iff$ $A$ closed and bounded.
\end{theorem}

\begin{proposition}\label{prop:productspacecompact}
    If $X, Y$ are compact topological spaces, then $X \times Y$ is compact.    
\end{proposition}

\begin{remark}
    By induction, if $X_1, \dots, X_n$ compact, so is $\prod_{i=1}^{n} X_i$.
\end{remark}

\begin{proposition}\label{prop:subsetcompact}
    A closed subset of a compact topological space is compact in the subspace topology.
\end{proposition}

\begin{proof}(Of \cref{thm:compactnessrn})

    \noindent($\impliedby$) If $A \subseteq \mathbb{R}^n$ closed and bounded, then $A \subseteq [-R, +R]^n$ for some $R > 0$ (it is contained in some "$n$-cube"). Then, we have that $[-R, R]$ is compact, by \cref{prop:closedcompact}, \cref{prop:productspacecompact}, and \cref{prop:subsetcompact}, $A$ itself compact.
    
    \noindent($\implies$) Suppose $A \subseteq \mathbb{R}^n$ is compact. Then, $\bigcup_{x \in A} B(x, \epsilon)$ for some $\epsilon > 0$ is an open cover of $A$. As $A$ compact, there must exist a finite subcover of this cover, $A \subseteq \bigcup_{i = 1}^N B(x_i, r_i)$. Let $R := \max_{i=1}^{N} (\norm{x_i} + r_i)$. Then, $A \subseteq \overline{B(0, R)}$, that is, $A$ is bounded.

    Now, suppose $x$ is a limit point of $A$. Then, any neighborhood of $x$ contains a point in $A$, so $\forall r > 0, B(x, r) \cap A \neq \varnothing$, and so $\overline{B}(x, r)$ also contains a point of $A$ for any $r > 0$. 
    
    Now, suppose $x \notin A$ (looking for a contradiction). Then, \[
    U :=\bigcup_{r > 0} U_r :=  \bigcup_{r > 0} (\mathbb{R}^n \setminus \overline{B(x, r)}) = \mathbb{R}^n \setminus \{x\}
    \]
    is an open cover for the set $A$. $A$ being compact implies that $U$ has an finite subcover such that $A \subset U_{r_1} \cup U_{r_2} \cup \dots \cup U_{r_N}$. Let $r_0 = \min_{i=1}^N r_i$. Then, $A \subseteq U_{r_0}$, and $A \cap B(x, r_0) = \varnothing$; but this is a contradiction to the definition of a limit point, hence any limit point $x$ is contained in $A$ and $A$ is thus closed by definition.
\end{proof}

\begin{proposition}
    Compact $\implies$ sequentially compact; that is, every sequence in a compact set has a convergent subsequence. 
\end{proposition}