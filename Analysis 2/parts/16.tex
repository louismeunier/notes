\section{Riemann Integral}

\subsection{Introduction}
\begin{definition}[Partitions]
    A \emph{partition} is a division of an interval $(a, b)$, denoted \[
    \mathcal{P} = \{a = x_0, x_1, \dots, x_{n-1}, x_n = b\}    .
    \]
    We define $\diam (\mathcal{P})  := \max_{n} \abs{x_{i} - x_{i-1}}$.

    A \emph{marked partition}, denoted $\dot{\mathcal{P}}$, is one in which, for each interval we choose some $t_i \in (x_i, x_{i+1}]$.
\end{definition}

\begin{definition}[Riemann Sum]
    We denote \[
    S(f,\dot{\mathcal{P}}) = \sum_{i=1}^n f(t_i)(x_{i}-x_{i-1}).
    \]
\end{definition}

\begin{definition}[Riemann Integrable]
    A function $f$ is \emph{Riemann Integrable} if $[a, b]$ if $S(f, \dot{\mathcal{P}}) \to L$ as $\diam(\dot{\mathcal{P}}) \to 0$ for any choice of $t_i \in [x_i, x_{i+1}]$.

    That is, $\forall \epsilon > 0, \exists \delta : $ if $\diam(\mathcal{P}) < \delta$, then for any choice of $t_i \in [x_i, x_{i+1}]$ we have $\abs{L - S(f, \dot{\mathcal{P}})} < \epsilon$.
\end{definition}

\begin{proposition}
    \begin{enumerate}
        \item If $L$ exists, it is unique.
        \item The integral is linear in $f(x)$; if $\int_a^b f(x)\dd{x}$ and $\int_a^b g(x) \dd{x}$ exist, then $\int_a^b (c_1 f+ c_2 g)\dd{x} = c_1\int_a^b f \dd{x} + c_2 \int_a^b g \dd{x}$.
        \item If $f \leq g$ are Riemann integrable on $[a, b]$, then $\int_a^b f(x) \dd{x} \leq \int_a^b g(x) \dd{x}$.
    \end{enumerate}
\end{proposition}

\begin{proposition}
    If $f(x)$ integrable on $[a, b]$, the $f(x)$ is bounded on $[a, b]$.
\end{proposition}

\begin{proof}
    Suppose $\int_a^b f$ exists. Let $\epsilon > 0$, and $\delta$ such that if $\diam (\dot{\mathcal{P}}) < \delta$ then $\abs{L - S(f, \dot{\mathcal{P}})}$. Let $\epsilon = 1$. Then, $S(f, \dot{\mathcal{P}}) \leq \abs{L} + 1$.

    Let $Q = \{a = x_0, \dots, x_n = b\}$ be a partition of $[a, b]$ such that $\diam (Q) < \delta$. Suppose towards a contradiction that $f$ is not bounded on $[a, b]$. Then, $f$ is unbounded on at least one interval $[x_i, x_{i+1}]$, say, on $[x_k, x_{k+1}]$. Let $t_i = x_i$ for $i \neq k$ and choose $t_k \in [x_k, x_{k+1}]$ such that $\abs{f(t_k)}(x_{k+1} - x_k) > \abs{L} + 1 + \abs{\sum_{i \neq k} f(t_i)(x_{i+1} - x_i)}$ (which we can do by assumption of $f$ being unbounded).

    By assumption, $\abs{S(f, \dot{Q})} \leq \abs{L} + 1$, but we have that \begin{align*}
        S(f, \dot{Q}) = \underbrace{\sum_{i \neq k} f(t_i)(x_{i+1} - x_{i})}_{:=N} + \abs{f(t_k)}(x_{k+1}-x_k) > 2N + \abs{L} + 1,
    \end{align*}
    contradiction.
\end{proof}

\subsection{Cauchy Criterion}

\begin{proposition}[Cauchy Criterion for Integrability]
    $f \in \mathcal{R}[a, b] \iff \forall \epsilon > 0, \exists \delta > 0 : $ if $\dot{P}$ and $\dot{Q}$ are tagged partitions of $[a, b] \st \diam{\dot{P}} < \delta$ and $\diam{\dot{Q}} < \delta$, then $\abs{S(f, \dot{P}) - S(f, \dot{Q})} < \epsilon$.\footnotemark
\end{proposition}
\footnotetext{Note that $\mathcal{R}[a, b]$ is the set of all real-valued functions integrable on the interval $[a, b]$.}

\subsection{Squeeze Theorem}

\begin{theorem}
    Let $f:[a, b] \to \mathbb{R}$. Then $\int_a^b f$ exists $\iff \forall \epsilon > 0, \exists \alpha_\epsilon(x), \omega_\epsilon(x) \in \mathcal{R}[a, b]$, $\alpha_\epsilon \leq f \leq \omega_\epsilon$, and \[
    \int_a^b (\omega_\epsilon - \alpha_\epsilon) < \epsilon    
    \]
\end{theorem}
\begin{proof}
    If $f \in \mathcal{R}[a, b]$ then take $\alpha_\epsilon = f = \omega_\epsilon$. 
    
    Conversely, let $\epsilon > 0$. Since $\alpha_\epsilon, \omega_\epsilon \in \mathcal{R}[a, b]$, then, $\exists \delta > 0$ such that for any tagged partition with $\dim \dot{P} < \delta$, then $\abs{S(\alpha_\epsilon, \dot{P}) - \int_a^b \alpha_\epsilon} < \epsilon$ and $\abs{S(\omega_\epsilon, \dot{P}) - \int_a^b \omega_\epsilon} < \epsilon$, thus \[
    \int_a^b \alpha_\epsilon - \epsilon < S(\alpha_\epsilon, \dot{P}) \leq S(f, \dot{P})\leq S(\omega_\epsilon, \dot{P}) < \int_a^b \omega_\epsilon + \epsilon.    
    \]
    Let $\dot{Q}$ be any other tagged partition with $\diam \dot{Q} < \delta$; then, the same inequality holds ie $\int_a^b \alpha_\epsilon - \epsilon < S(f, \dot{Q}) < \int_a^b \omega_\epsilon + \epsilon$. Subtracting one from the other, we see that \begin{align*}
        \abs{S(f, \dot{P}) - S(f, \dot{Q})} < \int_a^b \omega_\epsilon - \int_a^b \alpha_\epsilon + 2 \epsilon < 3 \epsilon,
    \end{align*}
    and thus $f \in \mathcal{R}[a,b]$ by Cauchy Criterion.
\end{proof}