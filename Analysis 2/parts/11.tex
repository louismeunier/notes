\begin{example}[Complete Metric Space Example: $L^p$ norm]
    Let $f \in C([a, b])$. We define the norm \[
    \norm{f}_p := \left(\int_a^b \abs{f(x)}^p \dd{x}\right)^\frac{1}{p}.
    \]
    As desired, $\norm{f}_p \geq 0$; $\norm{f}_p = 0 \iff f \equiv 0$; $\norm{c \cdot f}_p  = c \cdot \norm{f}_p$.

    Hölder's and Minkowski's inequalities for functions also hold; for $\frac{1}{p} + \frac{1}{q} = 1$, $1 \leq p, q \leq \infty$, \[
    \int \abs{fg} \leq \norm{f}_p \cdot\norm{g}_q; \quad \norm{f + g}_p \leq \norm{f}_p + \norm{g}_q,   
    \]
    respectively.

    We similarly have the $L^\infty$ norm, namely, for a function $f: [a,b] \to \mathbb{R}$, \[
    \norm{f}_\infty = \sup_{x \in [a,b]}\abs{f(x)},     
    \]
    which obeys all the necessary properties as well.

    Let $f_n \to f$ in $C([a, b])$, wrt $\norm{\cdots}_\infty$, where $\{f_n\}_{n \in \mathbb{N}}$ a sequence of functions. Namely, we say that $$\forall \epsilon > 0, \exists N \in \mathbb{N} : \forall n \geq N, \sup_{x \in [a,b]} \abs{f_n(x) - f(x)} < \epsilon.$$
    If this holds, we say that $f_n$ \emph{uniformly converges}. 

    We say that $f_n(x) \to f(x)$ \emph{pointwise} on $[a,b]$ if $\forall x \in [a, b], f_n (x) \to f(x)$. Note that uniform convergence implies pointwise convergence, but not the converse.
\end{example}

\begin{theorem}
    Suppose $f_n(x)$ continuous, and $f_n(x) \to f(x)$ uniformly on $[a, b]$. Then, $f(x)$ also continuous on $[a, b]$.
\end{theorem}

\begin{proof}
    Fix $\epsilon > 0$, $x_0 \in [a, b]$. We have that $\exists N : n \geq N, \abs{f_n(x) - f(x)} < \frac{\epsilon}{3}, \forall x \in [a, b]$.

    Let $n \geq N$. $f_n(x)$ continuous at $x_0$, hence $\exists \delta(x_0) > 0 : \abs{y - x_0} \implies \abs{f_n(y) - f_n(x_0)} < \frac{\epsilon}{3}$. We have \begin{align*}
        \abs{f(x_0) - f(y)} &\leq \abs{f(x_0) - f_n(x_0)} + \abs{f_n(x_0) - f_n(y)} + \abs{f_n(y) - f(y)}\\
        &\leq \frac{\epsilon}{3} + \frac{\epsilon}{3} + \frac{\epsilon}{3} = \epsilon,
    \end{align*}
    completing the proof.
\end{proof}

\begin{remark}
    This does not hold with pointwise convergence.
\end{remark}

\begin{remark}
    We will prove later that $C([a,b])$ is complete for $\norm{f}_\infty$, but not for arbitrary $\norm{f}_p$, $1 \leq p < +\infty$. To "complete" $C([a, b])$ for $p \neq \infty$, we will need to consider measurable functions and redefine our notion of integration.
\end{remark}

% ! hint: how to cover the cantor set. step 1: covered by 2/3. step 2: covered by 2/3*2/3, etc, converges to 0. in the end, we'll have 2^n inervals * length (1/3)^n -> 0.