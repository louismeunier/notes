\subsection{Path Components, Connected Components}
\begin{remark}
    Remark that if a metric space $X$ is not connected, then we can write $X = U \cup V$ where $U, V$ are open, nonempty and disjoint. It follows, then, that $U = V^C$ (and vice versa) and hence $U,V$ are both open and closed.
\end{remark}

\begin{definition}[Connected Component]
    A connected component of $x \in X$ is the largest connected subset of $X$ that contains $x$.
\end{definition}

\begin{example}
    Let $X = (0, 1) \cup (1, 2)$. Here, we have two connected components, $(0,1)$ and $(1, 2)$
\end{example}

\begin{example}[Middle Thirds Cantor Set]
    Let $C_0 := [0, 1]$, and given $C_n$, define $C_{n+1} := \frac{1}{3}\left(C_{n} \cup (2 +C_{n})\right)$ for $n \geq 0$. $C_\infty$ is totally disconnected.
\end{example}

\begin{definition}[Path Component]
    A \emph{path component} $P(x)$ of $x \in X$ is the largest path connected subset of $X$ that contains $x$.
\end{definition}

\begin{proposition}
    $P(x) = \{x \in X : \exists \text{ conintuous path } \gamma: [0, 1] \to X : \gamma(0) = x, \gamma (1) = y\}$.
\end{proposition}

\begin{remark}
    Where we "start" a path does not matter. We write $x \sim y$ if $\exists \gamma$ from $x$ to $y$; this is an equivalence relation on the elements of $X$.
\end{remark}

\begin{lemma}
    If $P(x) \cap P(y) \neq \varnothing$, then $P(x) = P(y)$.
\end{lemma}

\begin{proof}
    $P(x) \cap P(y) \neq \varnothing \implies \exists z : x \sim z \wedge y \sim z \implies x \sim y$.
\end{proof}

\begin{lemma}
    If $A \subseteq X$ is connected, then $\overline{A}$ is also connected.
\end{lemma}

\begin{lemma}
    Suppose $A \subseteq X$ is both open and closed. Then, if $C\subseteq X$ is connected and $C \cap A \neq \varnothing$, then $C \subseteq A$.
\end{lemma}
\begin{proof}
    If $A$ is both open and closed, then $C \cap A$ is both open and closed in $C$. If $C \cap A^C \neq \varnothing$, then this is also open and closed in $C$. Hence, we can write $C = (C \cap A) \cup (C \cap A^C)$, that is, a disjoint union of two nonempty open sets, contradicting the connectedness of $C$. Hence, $C \cap A^C = \varnothing$, and so $C \subseteq A$.
\end{proof}

\begin{proposition}
    Let $\{C_\alpha\}_{\alpha \in I}$ be a collection of nonempty connected subspaces of $X$ s.t. $\forall \alpha, \beta \in I, C_\alpha \cap C_\beta \neq \varnothing$. Then, $\cup_{\alpha \in I} C_\alpha$ is connected.
\end{proposition}

\begin{proposition}
    Suppose each $x \in X$ has a path-connected neighborhood. Then, the path components in $X$ are the same as the connected components in $X$.
\end{proposition}

\subsubsection{Cantor Staircase Function}
\begin{definition}[An Explicit Definition]
    Let $x \in C : x = 0. a_1 a_2 a_3 \dots$ (base 3), ie $a_j = \begin{cases}
        0\\
        2
    \end{cases}$. Define \[
    f(x) = \begin{cases}
        \sum \frac{a_j/2}{2^j} & x \in C\\
        \text{extend by continuity} & x \notin C.
    \end{cases}
    \]
    That is, if $x \notin C$, set $f(y)= \sup_{x \in C, x < y} f(x) = \inf_{x \in C, x > y} f(x)$.
\end{definition}

\begin{definition}[Complement Definition]
    To construct the complement of the Cantor set, begin with $[0, 1]$ and at a step $n$, we remove $2^n$ open intervals from this interval. $f(x)$ will be constant on each of these intervals with values $\frac{k}{2^n}$ where $k$ odd and $0 < k < 2^n$. Extend by continuity to all $x \in C$.
\end{definition}

\begin{remark}
    \href{https://en.wikipedia.org/wiki/Cantor_function}{Wikipedia's} explanation of this is far better than whatever this definition is trying to say.
\end{remark}
