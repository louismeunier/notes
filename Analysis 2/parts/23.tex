\begin{theorem}
    Suppose $\sum_{i=1}^{\infty} \vec{v}_i$, $\vec{v}_i \in \R^n$, converges, but $\sum_{i=1}^\infty \norm{\vec{v}_i} = + \infty$. Then, the set of rearranged sums $\sum_{i=1}^\infty \vec{v}_{\sigma(i)}$ for each $\sigma : \mathbb{N} \leftrightarrow \mathbb{N}$ permutation form an \emph{affine subspace } of $\R^n$.
\end{theorem}

\subsection{Tests for Absolute Convergence}

\begin{proposition}
    Let $x_n, y_n$ be sequences and $r \defeq \lim_{n \to \infty} \abs{\frac{x_n}{y_n}}$.
    \begin{enumerate}
        \item If $r \neq 0$, $\sum_{n=1}^\infty x_n$ converges absolutely iff $\sum_{n=1}^\infty y_n$ converges absolutely.
In addition, if $0 < r_1 \defeq \liminf \abs{\frac{x_n}{y_n}} \leq \limsup \abs{\frac{x_n}{y_n}} =: r_2 < + \infty$, this still holds.
        \item If $r = 0$, and if $\sum y_n$ converges absolutely, so does $\sum x_n$.
    \end{enumerate}
\end{proposition}

\begin{proposition}[Root Test]
    If there $\exists r < 1$ such that $\abs{x_n}^{1/n} \leq r$ for sufficiently large $n \geq K$, then $\sum_{n=K}^\infty \abs{x_n} \leq \sum_{n=K}^\infty r^{-n}$ converges.

    If $\abs{x_n}^{1/n} \geq n$ for $n \geq K$, $\sum x_n$ does not converge absolutely.
\end{proposition}

\begin{proposition}[Ratio Test]
    Let $x_n \neq 0$. If $\exists 0 < r < 1$, $\abs{\frac{x_{n+1}}{x_n}} \leq r$ for sufficiently large $n$, $\sum x_n$ absolutely convergent.

    If $\abs{\frac{x_{n+1}}{x_n}} \geq 1$ for $n \geq K$, $\sum x_n$ diverges.
\end{proposition}

\begin{proposition}[Integral Test]
    Let $f(x) \geq 0$ be non-increasing/non-decreasing function of $x \geq 1$. Then $\sum_{k=1}^\infty f(k)$ converges $\iff \lim_{k\to \infty} \int_1^k f(x) \dd{x}$ finite.
\end{proposition}

\begin{proposition}[Raube's Test]
    Let $x_n \neq 0$. 
    \begin{enumerate}
        \item Suppose $\exists a > 1 \st \abs{\frac{x_{n+1}}{x_n}} \leq 1 - \frac{1}{n}$, $n \geq K$. Then $\sum x_n$ converges absolutely.
        \item If $\exists a \leq 1 \st \abs{\frac{x_{n+1}}{x_n}} \geq 1 - \frac{1}{n}$, $n \geq K$. Then $\sum x_n$ does not converge absolutely.
    \end{enumerate}
\end{proposition}

\begin{corollary}
    Let $a \defeq \lim n(1 - \abs{\frac{x_{n+1}}{x_n}})$, if such a limit exists. Then, if $a > 1$, $\sum x_n$ converges absolutely, and if $a < 1$, $\sum x_n$ does not.
\end{corollary}

\subsection{Tests for Non-Absolute Convergence}
\begin{proposition}[Alternating Series]
    If $x_n > 0$, $x_{n+1} \leq x_n$, $\lim_{n \to \infty} x_n = 0 \implies \sum (-1)^n x_n$ converges.
\end{proposition}

\begin{lemma}[Abel's Lemma]
    Let $x_n,y_n \in \R$. Let $s_0 \defeq 0, s_n \defeq \sum_{k=1}^n y_k$. Then, for $m > n$, \[
    \sum_{k=n+1}^m x_k y_k = x_m s_m - x_{n+1} s_{n+1}     + \sum_{k=n+1}^m (x_k - x_{k+1}) s_k
    \]
\end{lemma}