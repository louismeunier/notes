\subsection{Dual Spaces}

\begin{definition}[Dual Space]
For a vector space $V$ over a field $\field$, linear transformations from $V \to \field$ (where we view $\field$ as a one-dimensional vector space over $\field$) are called \emph{linear functionals}. The space of linear functionals (namely, $\Hom(V, \field)$) is denoted $V^\ast$, and called the \emph{dual space} of $V$.
\end{definition}

\begin{proposition}
    If $V$ is finite dimensional, $\dim(V^\ast) = \dim(V)$.\footnotemark
\end{proposition}
\footnotetext{This does \emph{not} hold for infinite dimensional spaces.}
\begin{proof}
    For finite dimensional $V$, we know that $\dim(\Hom(V, \field)) = \dim(V) \cdot \dim(\field) = \dim(V)$, hence $\dim(V^*) = \dim(V)$. In the same notation with which we proved this originally in \cref{prop:construction}; fix a basis $\beta \defeq \{v_1, \dots, v_n\}$ for $V$ and the standard basis $\gamma \defeq \{1\}$ for $\field$, and defined $\beta^\ast \defeq \{f_1, \dots, f_n\}$, where $f_i \defeq T_{v_i, 1} : V \to \field$ maps $v_i \mapsto 1$ and every other basis vector to $0_\field$.
\end{proof}

\begin{remark}
The basis $\beta^\ast$ for $V^\ast$ is called the \emph{dual basis}. Explicitly, we have:
\end{remark}

\begin{corollary}
    Let $V$ be a finite dimensional vector space over $\field$ and let $\beta \defeq \{v_1, \dots, v_n\}$ be a basis for $V$. Then,\[
    \beta^\ast \defeq \{f_1, \dots, f_n\}    
    \]
    is a basis for $V^\ast$. Moreover, for each linear functional $f \in V^\ast$, \[    
    f = \sum_{i=1}^n f(v_i)\cdot f_i.
    \]
\end{corollary}

\begin{proof}
    \myuline{Linear indepedence:} let $a_1 f_1 + \cdots + a_n f_n = 0_{V^\ast}=: 0$. Then,
    \[
    (a_1f_1 + \cdots + a_n f_n)(v_i) = a_if_i(v_i) = a_i \cdot 1 = a_i \implies a_i = 0,    
    \]
    hence $\beta^\ast$ indeed linearly independent.

    \myuline{Spanning:} let $f \in V^\ast$. We claim that $f = \sum_{i = 1}^n f(v_i) f_i$. It suffices to show these two sides are equal on the basis vectors, as linear transformations are determined by their effect on basis vectors. We have:
    \begin{align*}
       \left( \sum_{i=1}^n f(v_i) f_i\right)(v_j) = \sum_{i=1}^n f(v_i) f_i(v_j) = \sum_{i=1}^n f(v_i) \cdot \delta_{ij} = f(v_j),
    \end{align*}
    as desired.\footnote{Where $\delta_{ij} \defeq \begin{cases}
        1 & i = j\\
        0 & i \neq j
    \end{cases}$ is the Kronecker delta.}
\end{proof}

\begin{example}
    \begin{enumerate}
    \item Let $V \defeq \field^n$ and $\beta \defeq \{v_1, \dots, v_n\}$ be a basis for $\field^n$, viewed as column vectors, and let $\beta^\ast \defeq \{f_1, \dots, f_n\}$ be the dual basis for $V^\ast$. Recall that $f_i : \field^n \to \field$, hence $f_i \defeq L_{A_i}$ for some matrix $A_i \in M_{1 \times n}(\field) \defeq$ space of $1 \times n$ row vectors. Hence, $A_i = e_i^t$.
    \item Consider $V^{\ast \ast}$, the dual of the dual. If $V$ is finite-dimensional, then as $\dim(V)= \dim(V^\ast)$, we have $\dim(V) = \dim(V^\ast) = \dim(V^{\ast \ast})$, ie, they are (abstractly) isomorphic.
    
    We have that $T : V \to V^\ast, v_i \mapsto f_i$ is an isomorphism; we define an explicit isomorphism to $V^{\ast \ast}$ below.
\end{enumerate}
\end{example}

\begin{definition}
Let $V$ be an arbitrary vector space over $\field$. For each $x \in V$, define $\hat x\in V^{\ast \ast}$ by $\hat x : V^\ast \to \field$, where $\hat{x}(f) \defeq f(x)$.
\end{definition}

\begin{remark}
    Note that $\hat{x}$ is linear.
\end{remark}

\begin{theorem}
    The map $x \mapsto \hat{x} : V \to V^{\ast \ast}$ is a linear injection. In particular, if $V$ is finite dimensional, it is an isomorphism.
\end{theorem}

\begin{proof}
    Let $x \in V$ and suppose $\hat{x} = 0_{V^{\ast \ast}}$. Let $\beta$ be a basis for $V$ and $\beta^\ast$ its dual basis. Let $x = a_1 v_1 + \cdots + a_n v_n$ for $v_i \in \beta, a_i \in \field$. Let $f_i$ such that $f_i(v_j) = \delta_{ij} v_j$. Then, \begin{align*}
        \hat{x}f_i = f_i (x) = f_i(a_1 v_1 + \cdots a_n v_n) = a_i = 0,
    \end{align*}
    hence, $a_i = 0 \forall i$. Hence, $x = 0$, and thus $\hat{x}$ has a trivial kernel and is thus injective.
\end{proof}

