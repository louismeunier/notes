\begin{example}
    \begin{align*}
        \begin{matrix}
            3x_1 + &2x_2 + &3x_3 - &2 x_4 &= 1\\
            x_1 + &x_2 + &x_3& &= 3\\
            x_1 + &2x_2 + &x_3 - &x_4 &= 2
        \end{matrix} 
        \quad \rightsquigarrow \quad A \defeq \begin{pmatrix}
            3 & 2 & 3 & -2\\
            1 & 1 & 1 & 0\\
            1 & 2 & 1 & -1
        \end{pmatrix}, \vec{b} \defeq \begin{pmatrix}
            1\\
            3\\
            2
        \end{pmatrix},
    \end{align*}
    so we have agumented matrix \begin{align*}
        (A \vert b) = \begin{pmatrix}
            3 & 2 & 3 & -2 & \vert &1 \\
            1 & 1 & 1 & 0 &\vert &3\\
            1 & 2 & 1 & -1 & \vert& 2
        \end{pmatrix} \quad \overset{\text{Gaussian Elimination}}{\rightsquigarrow} \quad \begin{pmatrix}
            1 & 0 & 1 & 0 & \vert &1\\
            0 & 1 & 0 & 0 & \vert & 2\\
            0 & 0 & 0 & 1 & \vert &3
        \end{pmatrix},
    \end{align*}
    so $r \defeq \rank (A) = 3$ and $\nullity(L_A) = 4 - 3 = 1$, so we expect a solution as a particular solution plus an ideal (the kernel). Rewriting, we see that \begin{align*}
        \begin{matrix}
            x_1  & & &+x_3 &= 1\\
            & x_2 & & &= 2\\
            & & & x_4 &= 3
        \end{matrix} \quad &\implies \quad \begin{pmatrix}
            x_1\\
            x_2\\
            x_3\\
            x_4
        \end{pmatrix} = \begin{pmatrix}
            1 - t_1\\
            2\\
            t_1\\
            3
        \end{pmatrix} = \begin{pmatrix}
            1\\
            2\\
            0\\
            3
        \end{pmatrix} + t_1\begin{pmatrix}
            -1\\
            0\\
            1\\
            0
        \end{pmatrix},
    \end{align*}
    where $t_1 \in \field$ arbitrary. Moreover, since setting $t_1 = 0$ gives that $\vec{v} \defeq (1, 2, 0, 3)^t$ a solution, then $t_1 (-1, 0, 1, 0)^t$ is a solution to the homogeneous system $A \vec{x} = \vec{0}$, ie, $\vec{u} \defeq (-1, 0, 1, 0)^t$ is a basis for the kernel of $\ker(L_A)$.
\end{example}

\begin{theorem}
    For any system $A \vec{x} = \vec{b}$, using Gaussian elimination we obtain another system $A_1 \vec{x}=\vec{b_1}$ where $(A_1 \vert \vec{b_1})$ is the reduced echelon form of $(A \vert \vec{b})$. Then:
    \begin{enumerate}
        \item $A \vec{x} = \vec{b}$ has a solution $\iff \rank(A_1 \vert \vec{b_1}) = \rank(A_1) = \sharp $ of non-zero rows of $A_1$.
        \item If a solution exists, then, denoting $r \defeq \rank(A)$ and $n \defeq \sharp$ columns of $A$, we have the general solution to $A \vec{x} = \vec{b}$ of the form $$\vec{v} + t_1 \vec{u}_1 + \cdots + t_{n-r} \vec{u}_{n-r}$$ where $\vec{v} \in \field^n$ and $\{\vec{u}_1, \dots, \vec{u}_{n-r}\}$ a basis for $\ker(L_A) = $ space of solutions to $A \vec{x} = \vec{0}$.
    \end{enumerate}
\end{theorem}

\begin{proof}
    We will only prove 1.

    Recall that $A \vec{x} = \vec{b}$ has a solution $\iff \vec{b} \in \im(L_A) = \Span(\text{columns of } A) \iff  \Span(\text{columns of } A) = \Span(\text{columns of } (A \vert b)) \iff \rank(A) = \rank((A \vert b))$.
\end{proof}

\begin{corollary}
    The system $A \vec{x} = \vec{b}$ has a solution $\iff$ in the reduced echelon form $(A_1\vert \vec{b}_1)$ of the augmented matrix, we do not have a pivot in the last column.
\end{corollary}

\begin{lemma}\label{lemma:elementarypreservationofdependence}
    Let $B \in M_{m \times n}(\field)$ be obtained from $A \in M_{m \times n}(\field)$ via a row operation. Then, for all $a_1, \dots, a_n \in \field$, $$a_1 A^{(1)} + \cdots + a_n A^{(n)} = \vec{0} \iff a_1 B^{(1)} + \cdots + a_n B^{(n)} = \vec{0}.$$
    In particular, columns in $A$ are linearly (in)dependent iff the corresponding columns in $B$ are linearly (in)dependent.
\end{lemma}
\begin{proof}
    Left as a (homework) exercise.
\end{proof}

\begin{lemma}
    Let $B$ be the reduced row echelon form of $A \in M_{m \times n}(\field)$. Then:
    \begin{enumerate}
        \item $\sharp$ non-zero rows of $B= \rank(B) = \rank(A) =: r$.
        \item For each $i = 1, \dots, r$, denote by $j_i$ the pivot of the $i$th row. Then, $B^{(j_i)} = e_i \in \field^m$. In particular, $\{B^{(j_1)}, \dots, B^{(j_r)}\}$ is linearly independent.
        \item Each column of $B$ without a pivot is in the span of the previous columns.
    \end{enumerate}
\end{lemma}
\begin{proof}
   Follows from the definition of rref.
\end{proof}

\begin{corollary}
    The rref of a matrix is unique.
\end{corollary}
\begin{proof}
    Left as a (homework) exercise.
\end{proof}
