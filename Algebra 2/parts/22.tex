\subsubsection{Application to Finding Inverse Matrix}

If $A \in M_{n}(\field)$ is invertible, then $A = E_1 \cdot \cdots \cdot E_k$ for some elementary matrices $E_i$, so $A^{-1} = E_k^{-1} \cdot \cdots \cdot E_1^{-1} \cdot I_n$.

Consider the augmented matrix $(A \vert I_n)$. Remark that $B \cdot (A \vert I_n) = (BA \vert BI_n)$, and in particular, $E_k^{-1}\cdots E_1^{-1}\cdot (A \vert I_n) = (I_n \vert A^{-1})$, ie, there are row operations that turn $(A \vert I_n)$ to $(I_n \vert A^{-1})$.

\begin{theorem}
    Let $A \in M_n(\field)$ be invertible.
    \begin{enumerate}
        \item There are row operations that turn $(A \vert I_n)$ into $(I_n \vert A^{-1})$.
        \item If row operations turn $(A \vert I_n)$ into $(I_n\vert B)$ then $B = A^{-1}$.
    \end{enumerate}
\end{theorem}

\subsubsection{Solving Systems of Linear Equations}

\begin{definition}
    For matrices $A_1, A_2 \in M_{m \times n}(\field)$ and $\vec{b}_1, \vec{b}_2 \in \field^m$, the systems of linear equations $A_1 \cdot \vec{x} = \vec{b}_1$ and $A_2 \cdot \vec{x} = \vec{b}_2$ are called \emph{equivalent} if their sets of solutions are equal. 
    
    In particular, any two systems with no solutions are equivalent.
\end{definition}
\begin{proposition}
    If $G \in \GL_m(\field)$ and $A \in M_{m \times n}(\field), \vec{b} \in \field^m$, then $G \cdot A \vec{x} = G \cdot \vec{b}$ is equivalent to $A \vec{x} = \vec{b}$
\end{proposition}

\begin{proof}
    Multiply both sides from the left by $G^{-1}$.
\end{proof}

\begin{corollary}
    Row operations applied to $(A \vert b)$ do not change the solution set of $A \vec{x} = \vec{b}$.
\end{corollary}

% \begin{array}[t][{@{}ccc|c@{}}]
%     1 & 2 & -1 & 1 \\
%     2 & -3 & 1 & 4 \\
%     0 & 1 & 2 & 0 \\
%    \end{array}
\begin{definition}[ref/rref]
    Let $B \in M_{m \times n}(\field)$. We say $B$ is in \emph{row echelon form} if \begin{enumerate}
            \item All zero rows are at the bottom, ie each nonzero row is above each zero row;
            \item The first nonzero entry (called a pivot) of each row is the only nonzero entry in its column;
            \item The pivot of each row appears to the right of the pivot of the previous row.
        \end{enumerate}
        If all pivots are $1$, then we say that $B$ is in \emph{reduced row echelon form}.
    % \[
    %         B = \begin{pmatrix}
    %             0 & \cdots & 0 & \boxtimes & \ast & \cdots & \ast & 0 & \ast & \cdots 0 & \ast\\
    %             0 & \cdots & 0 & \boxtimes & \ast & \cdots & \ast & 0 & \ast & \cdots 0 & \ast
    %         \end{pmatrix}    
    %     \]
\end{definition}

\begin{theorem}[Gaussian Elimination Theorem]
    There is a sequence of row operations of types 1. and 3. that bring any matrix $A \in M_{m \times n}(\field)$ to a row echelon form. Moreover, applying row operations of type 2. to a matrix in row echelon form results in a reduced row echelon form.
\end{theorem}