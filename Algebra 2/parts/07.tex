\begin{theorem}\label{thm:basesofequalcardinality}
    For every vector space $V$ over a field $\field$, any two bases $\mathcal{B}_1,\mathcal{B}_2$ are equinumerous/of equal size/cardinality, ie, there is a bijection between $\mathcal{B}_1$ and $\mathcal{B}_2$.
\end{theorem}

\begin{remark}
    We will only prove this for vector spaces that admit a finite basis.
\end{remark}

\begin{lemma}[Steinitz Substitution]\label{lemma:steinitz}
    \defvectorspace . Let $Y \subseteq V$ be a finite, linearly independent set and let $Z \subseteq V$ be a finite spanning set. Then:
    \begin{enumerate}
        \item $k:=\abs{Y} \leq \abs{Z} =: n$
        \item There is $Z' \subseteq Z$ of size $n -k$ s.t. $Y \cup Z'$ is still spanning.
    \end{enumerate}
\end{lemma}

\begin{proof}
    We prove by induction on $k$.

    $k = 0$ gives that $Y = \varnothing$, and so $Z' = Z$ itself works ($Z' \cup Y = Z$) as a spanning set.

    Suppose the statement holds for some $k \geq 0$. Let $Y$ be an independent set such that $\abs{Y} = k + 1$, ie \[
    Y := \{y_1, y_2, \dots, y_k, y_{k+1}\}, \quad y \in V.
    \]
    By our inductive assumption, we can consider $Y' := \{y_1, \dots, y_k\} \subseteq Y$ of size $k$, to obtain a set \[
    Z' = \{z_1, z_2, \dots, z_{n-k}\} \subseteq Z, \st Y' \cup Z' = \{y_1, \dots, y_k, z_1, \dots, z_{n-k}\}
    \]
    is spanning. As this is spanning, we can write $y_{k+1}$ as a linear combination of vectors in $Y' \cup Z'$, ie \[
    y_{k+1} = a_1 y_1 + \dots + a_k y_k + b_1 z_1 + \dots + b_{n-k} z_{n-k}, \quad a_i, b_j \in \field.
    \]
    It must be that at least one of $b_j$'s must be nonzero; if they were all zero, then $y_{k+1}$ would simply be a linear combination of vector $y_i$ giving that $y_{k+1}$ linearly dependent, contradicting our construction of $Y$ linearly independent.

    Assume, wlog, $b_{n-k} \neq 0$. Then, we can write \[
    z_{n-k} = b_{n-k}^{-1}y_{k+1}-b_{n-k}^{-1}a_1y_{1}-\cdots - b_{n-k}^{-1}a_ky_k - b_{n-k}^{-1}b_1 z_1 - \dots - b_{n-k}^{-1}b_{n-k-1}z_{n-k-1},    
    \]
    and hence $$z_{n-k} \in \Span \{y_1, \dots, y_{k+1}, z_1, \dots, z_{n-k-1}\} = \Span \left(\underbrace{\{y_1, \dots, y_{k+1}\}}_{Y} \cup \underbrace{\{z_1, \dots, z_{n-k-1}\}}_{:= Z''} \right).$$
    We had that $Y' \cup Z'$ was spanning, and $(Y' \cup Z') \setminus (Y \cup Z'') = \{z_{n-k}\} \subseteq \Span (Y \cup Z'')$, and we thus have that $Y \cup Z''$ is also spanning.
\end{proof}

\begin{corollary}[Finite Basis Case for \cref{thm:basesofequalcardinality} ]
    Let $V$ be a vector space that admits a finite basis. Then, any two bases of $V$ are equinumerous.
\end{corollary}

\begin{proof}
    Let $Y, Z$ be two finite bases for $V$. Then, $Y$ is independent and $Z$ is spanning, so by \nameref{lemma:steinitz}, $\abs{Y} \leq \abs{Z}$. OTOH, $Z$ is independent, and $Y$ is spanning, so by \nameref{lemma:steinitz}, $\abs{Z} \leq \abs{Y}$, and we conclude that $\abs{Y} = \abs{Z}$. Let $n := \abs{Y}$.

    It remains to show that there exist no infinite bases for $V$; it suffices to show that there is no independent set of size $n+1$. To this end, let $I \subseteq V$ such that $\abs{I} = n+1$ be an independent set. $Y$ is still spanning, hence, by the substitution lemma, $n + 1 \leq n$, a contradiction. Hence, $I$ as defined cannot exist and so any basis of $V$ must be of size $n$.
\end{proof}

\begin{definition}[Dimension]
    \defvectorspace . The \emph{dimension} of $V$, denote $$\dim (V)$$ as the cardinality/size of any basis for $V$. We call $V$ \emph{finite dimensional} if $\dim (V)$ is a natural number, i.e. $V$ admits a finite basis. Otherwise, we say $V$ is infinite dimensional.
\end{definition}

\begin{corollary}[of \nameref{lemma:steinitz}]\label{cor:corofstein}
    Let $V$ be a finite dimensional vector space over $\field$ and denote $n := \dim (V)$. Then: 
    \begin{enumerate}
        \item Every linearly independent subset $I \subseteq V$ has size $\leq n$;
        \item Every spanning set $S \subseteq V$ for $V$ has size $\geq n$;
        \item Every independent set $I$ can be completed to a basis to $V$, ie, there exists a basis $B$ for $V$ s.t. $I \subseteq B$.
    \end{enumerate}
\end{corollary}

\begin{proof}
    Fix a basis $B$ for $V$, $\abs{B} =: n$.
    \begin{enumerate}
        \item If $I$ is a independent set, then because $B$ spanning, \nameref{lemma:steinitz} gives $\abs{I} \leq \abs{B}$.
        \item If $S$ spanning for $V$, then because $B$ is linearly independent, \nameref{lemma:steinitz} gives $\abs{B} \leq \abs{S}$.
        \item Let $I$ be an independent set. Then, because $B$ is spanning, \nameref{lemma:steinitz} gives $B' \subseteq B$ of size $n - \abs{I}$ s.t. $I \cup B'$ is spanning. Moreover, $\abs{I \cup B'} \leq n$, and by 2. it must have size $\geq n$, and thus has size precisely $n$ and is thus a minimally spanning set and thus a basis.
    \end{enumerate}
\end{proof}

\begin{corollary}[Monotonicity of Dimension]
    \defvectorspace . For any subspace $W \subseteq $, $\dim W \leq \dim V$, and $$\dim W = \dim V \iff W = V.$$
\end{corollary}

\begin{proof}
    We build a maximally independent subset of $W$ as follows; let $S_0 := \varnothing$. If this is maximal for $W$, we are done. Otherwise, $\exists w_0 \in W \st S_1 := S_0 \cup \{w_0\}$ is independent. If $S_1$ is maximal, we are done. Else, $\exists w_2 \in W \st S_2 := S_1 \cup \{w_1\}$ is linearly independent, and so one for at most $n$ steps, since $\dim V = n$, and so by 1. of \cref{cor:corofstein}, every independent set has $\leq n$ elements. This constructs a maximally independent set in $W$, which would then be a basis for $W$, and thus $\dim W \leq \dim V$.

    
\end{proof}