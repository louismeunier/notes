\begin{example}
    \begin{enumerate}
        \item $A \defeq \begin{pmatrix}
            4 & 0 & 1\\
            2 & 3& 2\\
            1 & 0 & 4   
        \end{pmatrix}$, so $L_A : \field^3 \to \field^3$. Then, \begin{align*}
            p_A(t) = - \det\begin{pmatrix}
                4 - t & 0 & 1\\
                2 & 3 - t & 2\\
                1 & 0 & 4- t
            \end{pmatrix} = -(4-t)(3-t)(4-t) + 1 \cdot (3- t) \cdot 2 = - (t-5)(t-3)^2.
        \end{align*}
        Supposing $\text{char}(\field)\neq 2$ ie $3 \neq 5$, then we have two distinct eigenvalues $\lambda_1 = 5, \lambda_2 = 3$ with $m_a(5) = 1, m_a(3) = 2$, so the polynomial splits (regardless of $\field$). We have that $1 \leq m_g(5) \leq m_a(5) = 1$, so $m_g(5) = m_a(5) = 1$. We need only to check that $m_g(3) = 2$; but we have that \begin{align*}
            m_g(3) &= \nullity(L_A - 3\cdot I) = 3 - \rank(L_A - 3\cdot I) = 3 - \rank(A - 3I)\\
            &= 3 - \rank\begin{pmatrix}
                1 & 0 & 1\\
                2 & 0 & 2\\
                1 & 0 & 1
            \end{pmatrix}  = 3 - 1 = 2 = m_a(3),
        \end{align*}
        so $A$ indeed diagonalizable. A conjugate of $A$ that is diagonal is $D \defeq \begin{pmatrix}
            5 & 0 & 0 \\
            0 & 3 & 0\\
            0 & 0 & 3
        \end{pmatrix}$, and if $v_1$ an eigenvector for $\lambda_1 = 5$ and $v_2, v_3$ are linearly independent eigenvectors for $\lambda_2 = 3$, then \[
        Q \defeq \begin{pmatrix}
            \vert & \vert & \vert \\
            v_1 & v_2 & v_3\\
            \vert & \vert & \vert 
        \end{pmatrix}  = [I_3]_\beta^\alpha,
        \] 
        where $\alpha \defeq \{e_1, e_2, e_3\}$ and $\beta \defeq \{v_1, v_2, v_3\}$, is such that \[
        D = Q^{-1} A Q.
        \]

        In the case that $\text{char}(\field) = 2$, $3 = 5$ so we hae a single eigenvalue $\lambda = 1 = 3 = 5$ with $m_a(1) = 3$. But we still have that $\rank(A - I) = \rank \begin{pmatrix}
            1& 0 & 1\\
            0 & 0 & 0\\
            1 & 0 & 1
        \end{pmatrix} = 1$ so $m_g(1) = 2 < 3$, hence $A$ is not diagonalizable.

        \item Let $T : \field^2 \to \field^2$ be a rotation by ninety degrees, so $T(e_1) = e_2$ and $T(e_2) = - e_1$. Then, $T = L_A$ with \[
        A = [T]_\alpha = \begin{pmatrix}
            \vert & \vert\\
            e_2 & -e_1\\
            \vert & \vert
        \end{pmatrix}    = \begin{pmatrix}
            0 & - 1\\
            1 & 0
        \end{pmatrix},
        \]
        with $\alpha$ the standard basis. Then \begin{align*}
            p_T(t) = p_A(t) = -\det \begin{pmatrix}
                -t & -1 \\
                1 & -t
            \end{pmatrix} = t^2 + 1,
        \end{align*}
        which doesn't split over $\field\defeq\R$, but does over $\field\defeq\mathbb{C}$ or any $\field$ with characteristic 2 where $t^2 + 1 = (t+1)^2$.

        When $\field \defeq \C$, $p_T(t) = (t - i)(t + i)$ so we have 2 distinct eigenvalues with each having algebraic multiplicity 1, hence both have geometric multiplicity of 1 and thus $T$ is diagonalizable.

        When $\text{char}(\field) = 2$, we have a single eigenvalue $\lambda = 1$, with \begin{align*}
            m_g(1) = \nullity(T - I) =  2 - \rank(T - I) = 2 - \rank\begin{pmatrix}
                -1 & -1 \\
                1 &  -1
            \end{pmatrix} = 2 - \rank \begin{pmatrix}
                1 & 1\\
                1 & 1
            \end{pmatrix} = 1 < 2 = m_a(1),
        \end{align*}
        so $T$ is not diagonalizable.
    \end{enumerate}
\end{example}

\begin{remark}
    From the previous two examples, regard that the issue of diagonalizability is a field-related issue; not only because of the "splittability" of polynomials, but because of characteristic.
\end{remark}

\subsection{\texorpdfstring{$T$}{T}-cyclic Vectors and the Cayley-Hamilton Theorem}

\begin{definition}[$T$-cyclic subspace]
    Let $V$ be any vector space, $T : V \to V$ a linear operator, and $v \in V$. The \emph{$T$-cyclic subspace} of/generated by $v$ is the space $$\Span(\{v, T(v), T^2(v), \dots,\}) = \Span(\{T^n(v) : n \in \mathbb{N}\}).$$
\end{definition}

\begin{remark}
    Note that $T$-cyclic subspaces are $T$-invariant. In a sense, $T$-cyclic subspaces are "minimal $T$-invariant subspaces". Recall too that the characteristic polynomial of $T$ restricted to $T$-invariant subspaces divides the characteristic polynomial of $T$ by \cref{lemma:charpolynomialdiv}.
\end{remark}

\begin{lemma}\label{lemma:forcayley}
    Let $V$ be finite dimensional, $T : V \to V$ linear, and $v \in V$. Let $W \defeq $ the $T$-cyclic subspace generated by $v$.
    \begin{enumerate}
        \item $\{v, T(v), \dots, T^{k-1}(v)\}$ is a basis for $W$, where $k \defeq \dim(W)$.
        \item Since $T^{k}(v) \in \Span(\{v, T(v), \dots, T^{k-1}(v)\})$, we have a unique representation $T^k(v) = a_0 v + a_1T(v) + \cdots  + a_{k-1}T^{k-1}(v)$. Then, \[
        p_{T_W}(t) = t^k - a_{k-1}t^{k-1} - \cdots - a_1 t - a_0    
        \]
    \end{enumerate}
\end{lemma}

\begin{proof}
    Left as homework.

    Hint for 2.: use $\beta \defeq \{v, \dots, T^{k-1}(v)\}$ representation of $[T_W]_\beta$.
\end{proof}

\begin{remark}
    Note that if $V$ itself $T$-cyclic for some $v$, then $T$ "satisfies" its own characteristic polynomial. Indeed, $p_T(t) = t^n - a_{n-1}t^{n-1} - \cdots - a_0$ and so \[
    p_T(T) \defeq T^n - a_{n-1}T^{n-1} - \cdots - a_0 I_V 
    \]
    is equal to $0$ on $v$, and hence on all vectors $u \in V$ since $V = \Span (\{v, T(v), \dots, T^{n-1}(v)\})$ because $$p_T(T)(T^{i})(v) = T^{n+i}(v) - a_{n-1}T^{n-1+i}(v) - \cdots - a_0 T^{i}(v) = (T^{i} \circ p_T(T))(v) = T^{i}(p_T(v)) = T^{i}(0) = 0.$$ Even more generally, we have that this is true in general, precisely:
\end{remark}

\begin{theorem}[Cayley-Hamilton Theorem]
    Let $V$ be finite dimensional and $T : V \to V$ be linear. Then $T$ satisfies its own characteristic polynomial $p_T(t) = t^{n} + a_{n-1}t^{n-1} + \cdots + a_0$, ie \[
    p_T(T) = T^n + a_{n-1}T^{n-1} + \cdots + a_0I_V \equiv 0_V.    
    \]
\end{theorem}

\begin{proof}
    Fix $v \in V$. Let $W \defeq$ $T$-cyclic subspace generated by $v$, so $p_{T_W}(t) \vert p_T(t)$, ie $p_T(t) = q(t) \cdot p_{T_W}(t)$. Hence $p_T(T) = q(T) \circ p_{T_W}(T)$, and thus \begin{align*}
        p_T(T)(v) = q(T)(p_{T_W}(T)(v)) \overset{\text{\cref{lemma:forcayley}}}{=} q(T)(0) = 0.
    \end{align*}
\end{proof}

\begin{corollary}[Cayley-Hamilton for Matrices]
    For every $A \in M_n(\field)$, $p_A(A) = 0$.
\end{corollary}

\begin{proof}
    Follows immediately from $[L_A]_\alpha = A$.
\end{proof}