\section{Introduction}
\begin{remark}
    This course is about vector spaces and linear transformations between them; a vector space involves multiplication by \emph{scalars}, where the scalars come from some field. We recall first examples of fields, then vector spaces, as a motivation, before presenting a formal definition.
\end{remark}

\subsection{Vector Spaces}

\begin{remark}
    Much of this is recall from \href{https://notes.louismeunier.net/Algebra 1/algebra.pdf}{Algebra 1}.
\end{remark}

\begin{example}[Examples of Fields]
    \begin{enumerate}
        \item $\mathbb{Q}$; the field of rational numbers.
        \item $\mathbb{R}$; the field of real numbers; $\mathbb{Q} \subseteq \mathbb{R}$.
        \item $\mathbb{C}$; the field of complex numbers; $\mathbb{Q} \subseteq \mathbb{R} \subseteq \mathbb{C}$.
        \item $\mathbb{F}_p \equiv \zmod{p} \equiv \{0,1, \dots, p-1\}; the (unique) field of $p$ elements, where $p$ prime.\footnote{where $a +_p b \defeq $ remainder of $\frac{a+b}{p}$, $a \cdot_p b \defeq$ remainder of $\frac{a\cdot b}{p}$.}$
        \begin{enumerate}
            \item $p =2$; $\mathbb{F}_2 \equiv \{0, 1\}$.
            \item $p = 3$; $\mathbb{F}_3 \equiv \{0, 1, 2\}$.
            \item $\cdots$
        \end{enumerate}
    \end{enumerate}
\end{example}

\begin{remark}
Throughout the course, we will denote an abstract field as $\field$.
\end{remark}

\begin{example}[Examples of Vector Spaces]
    \begin{enumerate}
        \item $\mathbb{R}^3 \defeq \{(x, y, z) : x, y, z \in \mathbb{R}\}$. We can add elements in $\mathbb{R}^3$, and multiply them by real scalars.
        \item  $\mathbb{F}^n \defeq \underbrace{\mathbb{F} \times \mathbb{F} \times \cdots \mathbb{F}}_{n \text{ times}} \defeq \{(a_1, a_2, \dots, a_n) : a_i \in \mathbb{F}\}$, where $n \in \mathbb{N}$\footnotemark; this is a generalization of the previous example, where we took $n = 3$, $\mathbb{F} = \mathbb{R}$. Operations follow identically; addition:
        \begin{align*}
            (a_1, a_2, \dots, a_n) + (b_1, b_2, \dots, b_n) \defeq (a_1 + b_1, a_2 + b_2, \dots, a_n + b_n)
        \end{align*}
        and, taking a scalar $\lambda \in \mathbb{F}$, multiplication:
        \[
            \lambda \cdot (a_1, a_2, \dots, a_n) \defeq (\lambda \cdot a_1, \lambda \cdot a_2, \dots, \lambda \cdot a_n).
            \]
        We refer to these elements $(a_1, \cdots, a_n)$ as \emph{vectors} in $\mathbb{F}^n$; the vector for which $a_i = 0 \forall i$ is the \emph{$0$ vector}, and is the additive identity, making $\mathbb{F}^n$ an abelian group under addition, that admits multiplication by scalars from $\mathbb{F}$.
        \item $C(\mathbb{R}) \defeq \{f : \mathbb{R} \to \mathbb{R} : f \text{ continuous}\}$. Here, we have the constant zero function as our additive identity ($x \mapsto 0 \forall x$), and addition/scalar multiplication of two continuous real functions are continuous.
        \item $\field[t] \defeq \{a_0 + a_1 t + a_2 t^2 + \cdots + a_n t^n: a_i \in \field \forall i, n \in \mathbb{N}\}$, ie, the set of all polynomials in $t$ with coefficients from $\field$. Here, we can add two polynomials; \[
            (a_0 + a_1 t + \cdots + a_n t^n) + (b_0 + b_1 t + \cdots + b_m t^m) \defeq \sum_{i=0}^{\max \{n, m\}}(a_i + b_i)t^i,
        \]
        (where we "take" undefined $a_i$/$b_i$'s as 0; that is, if $m > n$, then $a_{m - n}, a_{m - n + 1}, \dots, a_m$ are taken to be $0$). Scalar multiplication is defined \[
        \lambda \cdot (a_0 + a_1 t + a_2 t^2 + \cdots + a_n t^n) \defeq \lambda a_0 + \lambda a_1 t + \lambda a_2 t^2 + \cdots + \lambda a_n t^n.
        \]
        Here, the zero polynomial is simply $0$ (that is, $a_i = 0 \forall i$).
    \end{enumerate}
\end{example}

\footnotetext{Where we take $0 \in \mathbb{N}$, for sake of consistency. Moreover, by convention, we define $\mathbb{F}^0$ (that is, when $n = 0$) to be $\{0\}$; the trivial vector space.}

\begin{definition}[Vector Space]
    A \emph{vector space} $V$ \underline{over} a field $\field$ is an \emph{abelian group} with an operation denoted $+$ (or $+_V$) and identity element\footnotemark denoted $0_V$, equipped with \emph{scalar multiplication} for each scalar $\lambda \in \field$ satisfying the following axioms:
    \begin{enumerate}
        \item $1 \cdot v = v$ for $1 \in \mathbb{F}, \forall v \in V$.
        \item $\alpha\cdot(\beta \cdot v) = (\alpha \cdot \beta)v, \forall \alpha, \beta \in \field, v \in V$.
        \item $(\alpha + \beta) \cdot v = \alpha \cdot v + \beta \cdot v, \forall \alpha, \beta \in \field, v \in V$.
        \item $\alpha \cdot (u + v) = \alpha \cdot u + \alpha \cdot v, \forall \alpha \in \field, u, v \in V$.
    \end{enumerate}
    We refer to elements $v \in V$ as \emph{vectors}.
\end{definition}
\footnotetext{The "zero vector".}

\begin{proposition}
    For a vector space $V$ over a field $\field$, the following holds:
    \begin{enumerate}
        \item $0 \cdot v = 0_V, \forall v \in V$ (where $0\defeq0_\field$)
        \item $-1 \cdot v = -v, \forall v \in V$ (where $1\defeq1_\field$)\footnotemark
        \item $\alpha \cdot 0_V = 0_V, \forall \alpha \in \field$
    \end{enumerate}
\end{proposition}

\footnotetext{NB: "additive inverse"}

\begin{proof}
    \begin{enumerate}
        \item $0 \cdot v = (0 + 0)\cdot v = 0 \cdot v + 0 \cdot v \implies 0 \cdot v = 0_V$ (by "cancelling" one of the $0 \cdot v$ terms on each side).
        % \item $0 = 0 + 0$ (by definition in $\field$) $\implies 0\cdot v = (0 + 0) \cdot v \overset{\text{axiom 3.}}{\implies} 0 \cdot v = 0 \cdot v + 0 \cdot v \overset{\text{group, inverses exist}}\implies (0 \cdot v) + (0 \cdot v)^{-1} = (0 \cdot v) + (0 \cdot v)^{-1} + 0 \cdot v \implies 0_V = 0 \cdot v$.
        \item $v + (-1 \cdot v) = (1 \cdot v + (-1) \cdot v) = (1 - 1)\cdot v = 0 \cdot v = 0_V \implies (-1 \cdot v) = -v$.
        \item $\alpha \cdot 0_V = \alpha \cdot (0_V + 0_V) = \alpha \cdot 0_V + \alpha \cdot 0_V \implies \alpha \cdot 0_V = 0_V$ (by, again, cancelling a term on each side).
    \end{enumerate}
\end{proof}