\begin{theorem}[First Isomorphism Theorem for Vector Spaces]
    Let $V, W$ be vector spaces over $\field$. Let $T : V \to W$ be a linear transformation. Then, \[
    V/\ker(T) \cong \im(T),\]
    by the isomorphism given by $v + \ker(T) \mapsto T(v)$.
\end{theorem}

\begin{proof}
    From group theory, we know that $\hat{T} : V/\ker(T) \to \im (T)$, where $\hat{T}(v + \ker(T)) \defeq T(v)$ is well-defined, and is an isomorphism of abelian groups. We need only to check that $\hat{T}$ is linear, namely, that is respects scalar multiplication. We have \begin{align*}
        \hat{T}(a \cdot (v + \ker(T))) &= \hat{T}((a \cdot v) + \ker (T))\\
        &= T(av) = a \cdot T(v)\\
        &= a \hat{T}(v + \ker(T)),
    \end{align*}
    as desired.
\end{proof}

\subsection{The Space \texorpdfstring{$\Hom(V, W)$}{Hom(V, W)}}

\begin{definition}[Homomorphism Space]
    For vector spaces $V, W$ over $\field$, let $\Hom(V, W)$ (also denoted $\ell (V, W)$) denote the set of all linear transformations from $V$ to $W$. We can turn this into a vector space over $\field$ as follows:
    \begin{enumerate}
        \item \textit{Addition of linear transformations:} for $T_0, T_1 \in \Hom(V, W)$, define \[
            (T_0 + T_1) : V \to W, \quad v \mapsto T_0(v) + T_1(v).
        \]
        $(T_0 + T_1)$ is clearly a linear transformation, as the linear combination of linear transformations $T_0, T_1$.
        \item \textit{Scalar multiplication of linear transformations:} for $T \in \Hom(V, W)$, $a \in \field$, define \[
        (a \cdot T) : V \to W, \quad v \mapsto a \cdot T(v),    
        \]
        which is again clearly linear in its own right.
    \end{enumerate}
\end{definition}

\begin{proposition}
    Endowed with the operations described above, $\Hom(V, W)$ is a vector space over $\field$.
\end{proposition}

\begin{proof}
    Follows easily from the definitions.
\end{proof}

\begin{theorem}[Basis for $\Hom(V, W)$]\label{thm:basisforhom}
    For vector spaces $V, W$ over $\field$ and bases $\beta, \gamma$ for $V, W$ resp., the following set \[
    \{T_{v, w} = v \in \beta, w \in \gamma\},
    \]
    is a basis for $\Hom(V, W)$, where for each $v \in \beta$ and $w \in \gamma$, $T_{v, w} \in \Hom(V, W)$ defined as the unique linear transformation such that $$T_{v, w} (v') = \begin{cases}
        w & v' = v\\
        0_W & v' \neq v \iff \in \beta \setminus \{v\}
    \end{cases}.$$
\end{theorem}

\begin{proof}
    Left as a (homework) exercise.
\end{proof}



\begin{corollary}
        If $V, W$ finite dimensional, then $\dim(\Hom(V, W)) = \dim(V) \cdot \dim(W)$.
\end{corollary}

\begin{proposition}\label{prop:construction}
    Let $\beta = \{v_1, \dots, v_n\}$, $\gamma = \{w_1, \dots, w_m\}$ be bases for $V, W$ resp. Then, by \cref{thm:basisforhom}, \[
        \{T_{v_i, w_j} : i \in \{1, \dots, n\}, j \in \{1, \dots, m\}\}
    \]
    is a basis for $\Hom(V, W)$, and it has $n \cdot m$ vectors by construction.
\end{proposition}

\subsection{Matrix Representation of Linear Transformations, Finite Fields}

Consider a linear transformation $T : \field^n \to \field^m$ between finite fields. We know that $T$ is uniquely determined by its value of basis vectors, so fix the standard bases \[
\beta = \{e_1^{(n)}, \dots, e_n^{(n)}\}=\{v_1, \dots, v_n\},
\]
and note that $T$ is determined by $\{T(v_1), \dots, T(v_n)\} \subseteq \field^m$. 

\begin{remark}
    We denote vectors in $\field^n$ as column vectors, ie $\begin{pmatrix}
        a_1\\
        \vdots\\
        a_n
    \end{pmatrix} \in \field^n$.
\end{remark}

Each $T(v_i)$ is a column vector in $\field^m$, and we an put these into a $m \times n$ matrix, namely:\footnote{Where $[T]$ denotes a matrix named "$T$".}

\begin{align*}
    [T] \defeq \begin{pmatrix}
        \vert & & \vert \\
        T(v_1) & \cdots & T(v_n) \\
        \vert & & \vert 
    \end{pmatrix} = \underbrace{\begin{pmatrix}
        a_{11} & \cdots & a_{1n}\\
        \vdots & \ddots & \vdots\\
        a_{m1} & \cdots & a_{mn}
    \end{pmatrix}}_{n}
    % ! hacky but pretty
    \hspace{-0.6em}
    % \begin{rightcase}
        \\\\\\
    % \end{rightcase}\text{\scriptsize{$m$}}
\end{align*}

We call this the \emph{matrix representation} of $T$ in the standard bases. The operation of multiplying an $m \times n$ matrix and a $n \times 1$ vector is precisely defined so that \begin{proposition}\label{prop:tofvequalmatt}
    $T(v) = [T] \cdot v$ for all $v \in \field^n$.
\end{proposition}

\begin{proof}
    Let $v = \begin{pmatrix}
        x_1\\
        \vdots\\
        x_n
    \end{pmatrix},$ where $v = x_1 v_1 + \cdots + x_n v_n$. Then \begin{align*}
        T(v) = x_1 T(v_1) + \cdots + x_n T(v_n)\\
        T(v_i) = \begin{pmatrix}
            a_{1i}\\
            \vdots\\
            a_{mi}
        \end{pmatrix}
    \end{align*}
    so
    \begin{align*}
        T(v) = \begin{pmatrix}
            a_{11} \cdot x_1 + \cdots + a_{1n} \cdot x_n\\
            \ddots \\
            a_{m1} \cdot x_1 + \cdots + a_{mn} \cdot x_n
        \end{pmatrix} = [T]\cdot v  
    \end{align*}
\end{proof}

\begin{definition}
    For a given $m \times n$ matrix $A$ over $\field$, define $L_A : \field^n \to \field^m$ by $L_A (v) \defeq A \cdot v$, where $v$ is viewed as an $n \times 1$ column. It follows from definition that the $L_A$ is linear.

    In other words, every $T \in \Hom(\field^n, \field^m)$ is equal to $L_A$ for some $A$.
\end{definition}