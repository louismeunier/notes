\subsubsection{Application to Matrix Rank}

\begin{definition}[Matrix Rank/C-Rank,R-Rank]
    For a matrix $A \in M_{m\times n} (\field)$, we define \[
    \rank(A) := \rank(L_A)    
    \]
    and the \emph{column rank} of \[
    \crank(A) := \text{size of maximal indep. subset of columns } \{A^{(1)}, \dots, A^{(n)}\}
    \]
    and \emph{row rank} of \[
    \rrank(A) := \text{size of maximal indep. subset of rows } \{A_{(1)}, \dots, A_{(m)}\}.
    \]
\end{definition}

\begin{remark}
    Notice that $\rank(A) = \crank(A)$.
\end{remark}

\begin{corollary}
    \[
    \rank(A) = \rank(A^t) = \rrank(A)
    \]
\end{corollary}

\begin{proof}
    We know already that $\rank(A^t) = \crank(A^t) = \rrank(A)$, as remarked previously, hence we need only to show that $\rank(A^t) = \rank(A)$. But $A = [L_A]$ and $A^t = [L_{A^t}] = [L_A]^t = [L_A^t]$. Thus, $\rank(A) = \rank(L_A) = \rank(L_A^t) = \rank(A^t)$.
\end{proof}

\begin{corollary}
    $$\rank(A) = \crank(A) = \rrank(A), \quad \forall A \in M_{m \times n}(\field)$$
\end{corollary}

\section{Elementary Matrices, Matrix Operations}
\subsection{Systems of Linear Equations}

We can write a system of $m$ equations of $n$ unknowns $x_i$
\begin{align*}
    \begin{cases}
        a_{11}x_1 + \cdots + a_{1n}x_n = b_1\\
        \qquad \ddots \qquad \ddots \qquad \ddots \\
        a_{m1}x_1 + \cdots + a_{mn}x_n = b_m
    \end{cases}
\end{align*}
succinctly as a matrix equation \[
A \cdot \vec{x} = \vec{b},    
\]
where $A := (a_{ij}) \in M_{m \times n}(\field)$, $\vec{x} = \begin{pmatrix}
    x_1\\
    \vdots\\
    x_n
\end{pmatrix}$, and $\vec{b} := \begin{pmatrix}
    b_1\\
    \vdots\\
    b_m
\end{pmatrix} \in \field^m$. Hence, $\vec{x}$ solves $A \vec{x} = \vec{b} \iff L_{A}(\vec{x}) = \vec{b} \iff \vec{x} \in L_{A}^{-1}(\vec{b})$. In other words, a solution exists iff $\vec{b} \in \im(L_A) = \Span(A^{(1)}, \dots, A^{(n)})$. In particular, when $\vec{b} = \vec{0}$, a solution always exists, $\vec{x} = \vec{0}$. We call $A \cdot \vec{x} = \vec{0}$ the \emph{homogeneous system of equations} of $A$. 

It follows that $A \cdot \vec{x} = \vec{0}$ has nonzero solutions $\iff \ker(L_A)$ non-trivial. Moreover, if $A \cdot \vec{x} = \vec{b}$ and $A \cdot \vec{y} = \vec{0}$, then $A\cdot (\vec{x} + \vec{y}) = \vec{b}$ as well by linearity.

\begin{proposition}\label{prop:kernelcoset}
    For $A \in M_{m \times n}(\field)$ and $b \in \im(L_A)$ the set of solutions to $A\vec{x} = \vec{b}$ is precisely the coset $\vec{v} + \ker(L_A)$ where $\vec{v} \in \field^n$ is a particular solution to $A \vec{x} = \vec{b}$; $A \vec{v} = \vec{b}$.
\end{proposition}

\begin{proof}
    $\vec{v} + $ an element of $\ker(L_A)$ is a solution to $A \vec{x} = \vec{b}$. Conversely, if $\vec{v}, \vec{w}$ are solutions to $A \vec{x} = \vec{b}$, then $A \cdot (\vec{v} - \vec{w}) = \vec{b} - \vec{b} = \vec{0}$ so $\vec{v} - \vec{w} \in \ker(L_A)$, thus $\vec{w} = \vec{v} + (\vec{v} - \vec{w}) \in \vec{v} + \ker(L_{A})$.
\end{proof}

\begin{corollary}
    If $m < n$ and $A \in M_{m \times n}(\field)$, then there is always a nonzero solution to the homogeneous equation $A \vec{x} = \vec{0}$
\end{corollary}

\begin{proof}
    $\nullity{(L_A)} = n - \rank(L_A) = n - \dim(\im(L_A)) \geq n - m > 0$ hence $\ker(L_A)$ nontrivial.
\end{proof}