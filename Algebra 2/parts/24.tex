% \begin{center}
%     \begin{tikzpicture}
%     \pic[
%     braid/.cd,
%     every strand/.style={ultra thick},
%     strand 1/.style={solarized-magenta},
%     strand 2/.style={solarized-green},
%     strand 3/.style={solarized-blue},
%     ] {braid={s_1 s_2^{-1} s_1 s_2^{-1} s_1 s_2^{-1}}};
%     \end{tikzpicture}
% \end{center}
% % https://mirror.quantum5.ca/CTAN/graphics/pgf/contrib/braids/braids.pdf

\subsection{Determinant}

The determinant, denoted $\det(A)$, of a square matrix $A \in M_n(\field)$ is a scalar from $\field$, meant to equal $0$ iff $A$ is not invertible.

\begin{proposition}
    $A \in M_{n}(\field)$ is invertible $\iff$ the columns of $A$  are linearly independent $\iff$ the rows of $A$  are linearly independent $\iff$ $\rank(A) = n$
\end{proposition}
\begin{proof}
    $A$ invertible $\iff$ $L_A$ invertible $\iff$ $L_A$ bijection $\iff$ $L_A$ surjection $\iff$ $\rank(L_A) = \rank(A) = n$
\end{proof}

\begin{example}
    Let $A \in M_3(\R)$, $A = \begin{pmatrix}
       - & v_1 & -\\
       - & v_2 & -\\
       - & v_3 & -
    \end{pmatrix}$. If $\{v_1, v_2, v_3\}$ linear dependent, then $\dim(\Span(v_1, v_2, v_3)) \leq 2$, which happens iff the parallelepiped formed with sides $v_1, v_2, v_3$ is contained in a plane (is "flat"), iff the parallelepiped is a parallelogram, ie, has 0 volume. As such, we can make the notion of volume dependent on the orientation of $v_1, v_2, v_3$ such that permuting $v_1, v_2, v_3$ changes the sign of the volume. This gives us the idea of an "oriented volume", which we can define as our determinant. This has a clear meaning in $\R$, but it remains to show how we can generalize this to arbitrary fields, where such a "volume" does not have a concrete meaning.
\end{example}

We now aim to derive a general formula for the determinant of a matrix over an arbitrary field by observing several key characteristics of our parallelepiped constructed above, and using these to define a unique determinant formula with geometric motivations.

\begin{center}
\textbf{Observation 1}\\
\textit{Scaling a vector in a parallelepiped scales the volume of the parallelepiped by the same scalar.}
\end{center}

\begin{definition}[multilinear form]
    A function $\delta: M_n(\field) \to \field$ is called (row) multilinear, or $n$-linear, if it is linear in every row, i.e. for each $i = 1, \dots, n$, \[
    \delta \begin{pmatrix}
        - &v_1& -\\
        & \vdots & \\
        - & v_{i-1} & -\\
        - & c \cdot\vec{x} + \vec{y}&-\\
        - & v_{i+1} & -\\
        & \vdots & \\
        - & v_n & -
    \end{pmatrix}  =  c \cdot \delta \begin{pmatrix}
            - &v_1& -\\
            & \vdots & \\
            - & v_{i-1} & -\\
            - & \vec{x} &-\\
            - & v_{i+1} & -\\
            & \vdots & \\
            - & v_n & -
        \end{pmatrix} + \delta \begin{pmatrix}
            - &v_1& -\\
            & \vdots & \\
            - & v_{i-1} & -\\
            - & \vec{y} &-\\
            - & v_{i+1} & -\\
            & \vdots & \\
            - & v_n & -
        \end{pmatrix}.
    \]
\end{definition}

\begin{example}
    \begin{enumerate}
        \item $\delta(A) \defeq a_{11}\cdot a_{22}\cdot \dots \cdot a_{nn}$ is $n$-linear.
        \item Fix $j \in \{1, \dots, n\}$. The function $\delta_j(A) \defeq a_{1j}\cdot a_{2j} \cdot \dots \cdot a_{nj}$ is $n$-linear.
        \item[$^\ast 3.$] However, $\tr(A) \defeq \sum_{i=1}^n a_{ii}$ is \emph{not} $n$-linear; scalar multiplication fails.
    \end{enumerate}
\end{example}

\begin{proposition}
    For an $n$-linear form $\delta : M_n(\field) \to \field$, if $A \in M_n(\field)$ has zero row, then $\delta(A) = 0$.
\end{proposition}

\begin{proof}
    $\delta(A) = \delta\left( \begin{pmatrix}
        \vec{0}\\
        \vdots
    \end{pmatrix}\right) = \delta\left(\begin{pmatrix}
        \vec{0}\\
        \vdots
    \end{pmatrix}+\begin{pmatrix}
        \vec{0}\\
        \vdots
    \end{pmatrix}\right) = \delta\left(\begin{pmatrix}
        \vec{0}\\
        \vdots
    \end{pmatrix}\right) + \delta\left(\begin{pmatrix}
        \vec{0}\\
        \vdots
    \end{pmatrix}\right) = \delta(A) + \delta (A) \implies \delta(A) = 0$.
\end{proof}


\begin{center}
    \textbf{Observation 2}\\
    \textit{If two sides of the parallelepiped are equal, then the volume is 0 (the shape is "flat").}
\end{center}

% TODO swap definition and proposition
\begin{definition}[Alternating]
    A $n$-linear form $\delta : M_n(\field) \to \field$ is called \emph{alternating} if $\delta(A) = 0$ for any matrix $A$ whose two equal rows.
\end{definition}

\begin{proposition}\label{prop:deltaalternatingswapping}
    Let $\delta : M_n(\field) \to \field$ be an alternating $n$-linear form. Then, if $B$ is obtained from $A$ by swapping two rows, then $\delta(B) = - \delta(A)$.
\end{proposition}

\begin{proof}
    It suffices to show that swapping two consecutive rows changes the sign of the result. Suppose $B$ is obtained from $A$ by swapping rows 1 and 2, namely
    \[
    B = \begin{pmatrix}
        - & A_{(2)} & - \\
        - & A_{(1)} & - \\
        & \vdots & 
    \end{pmatrix}.  
    \]
    Then, \[
    \delta \begin{pmatrix}
        - & A_{(1)} + A_{(2)} & -\\
        - & A_{(1)}+ A_{(2)} & -\\
        & \vdots & 
    \end{pmatrix}   = 0,  
    \]
    since its first two rows are equal; OTOH,
    \[
        \delta \begin{pmatrix}
            - & A_{(1)} + A_{(2)} & -\\
            - & A_{(1)}+ A_{(2)} & -\\
            & \vdots & 
        \end{pmatrix}   = \delta(A) + \delta(B),    
    \]
    so $\delta(B) = - \delta(A)$.
\end{proof}

\begin{proposition}
    A multilinear form $\delta : M_n(\field) \to \field$ is alternating $\iff$ $\delta(A) = 0$ for every matrix $A$ with two equal consecutive rows.
\end{proposition}

\begin{proof}
    Left as a (homework) exercise.
\end{proof}



\begin{center}
    \textbf{Observation 3}\\
    \textit{If $v_i = e_i$ for $i = 1, \dots, n$, ie, our parallelepiped is the unit cube, then the volume, aptly, equals 1; it is "normalized".}
\end{center}
