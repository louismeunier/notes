\begin{proof}[Proof (Attempt)](Of \cref{thm:vectorspacebases})
    We will try to "inductively" build a maximally independent set, as follows:
    
    Begin with an empty set $S_0 := \varnothing$, and iteratively add more vectors to it. Let $v_0 \in V$ be a non-zero vector, and let $S_1 := \{v_0\}$.

    If $S_1$ is maximal, then we are done. Otherwise, there exists a new vector $v_1 \in V \setminus S_1 \st S_2 := \{v_0, v_1\}$ is still independent.

    If $S_2$ is maximal, then we are done. Otherwise, there exists a new vector $v_2 \in V \setminus S_2 \st S_3 := \{v_0, v_1, v_2\}$ is still independent.

    Continue in this manner; this would take arbitrarily many finite, or even infinite, steps; we would need some "choice function" that would "allow" us to choose any particular $i$th vector $v_i$. 
    
    We can make this construction precise via the Axiom of Choice and transfinite induction (on ordinals); alternatively, we will prove a statement equivalent to the Axiom of Choice, Zorn's Lemma.
\end{proof}

\begin{remark}
    Before stating Zorn's Lemma, we introduce the following terminology.
\end{remark}

\begin{axiom}[Axiom of Choice]
    Let $X$ be a set of nonempty sets. Then, there exists a choice function $f$ defined on $X$ that maps each set of $X$ to an element of that set.
\end{axiom}

\begin{definition}[Inclusion-Maximal Element]
    A \emph{inclusion-maximal} element of $I$ is a set $S \in I$ s.t. there is no strict super set $S' \supsetneq S \st S' \in I$. 
\end{definition}

\begin{definition}[Chain]
    Let $X$ a set. Call a collection $\mathcal{C} \subseteq \pset{X}$ a \emph{chain} if any two $A, B \in \mathcal{C}$ are comparable, ie, $A \subseteq B$ or $B \subseteq A$.
\end{definition}

\begin{definition}[Upper Bound]
    An \emph{upper bound} of a collection $\tau \subseteq \pset{X}$ is a set $U \subseteq X \st U \supseteq J \forall J \in \tau$; $U$ contains the union of all sets in $J$.
\end{definition}

\begin{example}[Of The Previous Definitions]
    Let $X := \mathbb{N}, I := \{\varnothing, \{0\}, \{1, 2\}, \{1, 2, 3\}\} \subseteq \pset{\mathbb{N}}$.

    The maximal elements of $I$ would be $\{0\}$ and $\{1, 2, 3\}$.

    Chains would include $\mathcal{C}_0 := \{\varnothing, \{1, 2 \}, \{1,2,3\}\}, \mathcal{C}_1 := \{\varnothing, \{0\}\}, \mathcal{C}_2 := \{\varnothing\}$ (or any set containing a single element).

    The sets $\{0, 1, 2, 3\}$ and $\{0, 1, 2, 3, 4, 5\}$ are upper bounds for $I$, while neither is an element of $I$. The set $\{1, 2, 3\}$ is an upper bound for $\mathcal{C}_0$. A chain $\{\varnothing, \{0\}, \{0,1\}, \{0,1,2\}, \dots\}$ has an upper bound of $\mathbb{N}$.
\end{example}

\begin{lemma}[Zorn's Lemma]\label{lemma:zorns}
    Let $X$ be an ambient set and $I \subseteq \pset{X}$ be a nonempty collection of subsets of $X$. If every chain $\mathcal{C} \subseteq I$ has an upper bound in $I$, then $I$ has a maximal element.
\end{lemma}

\begin{proof}["Proof"]
    This is equivalent to the Axiom of Choice; proving it is beyond the scope of this course :(.
\end{proof}

\begin{proof}[Proof of \cref{thm:vectorspacebases}, cnt'd]
    We obtain a maximal independent set using Zorn's Lemma.

    Let $I$ be the collection of all linearly independent subsets of $V$. $I$ is nonempty; $\varnothing \in I$, as is $\{v\} \in I$ for any nonzero $v \in V$. To apply Zorn's, we need to show that every chain $\mathcal{C}$ if sets in $I$ has an upper bound in $I$; that is, every linearly independent set has an upper bound that itself is linearly independent.

    Let $\mathcal{C}$ be a chain in $I$. Let $S := \bigcup \mathcal{C}$ be the union of all sets in $\mathcal{C}$. To show $S$ is linearly independent, it suffices to show that every finite subset $\{v_1, \dots, v_n\} \subseteq S$ is linearly independent. Let $S_i \in \mathcal{C}$ be s.t. $v_i \in S_i$ for each $i$. Because $\mathcal{C}$ a chain, for each $i,j$ we have either $S_i \subseteq S_j$ or $S_j \subseteq S_i$, and so we can order $S_1, \dots, S_n$ in increasing order w.r.t $\subseteq$. This implies, then, there is a maximal $S_{i_0} \st S_{i_0} \supseteq S_i \forall i \in \{1, \dots, n\}$. Moreover, we have that $\{v_1, \dots, v_n\} \in S_{i_0}$, and that $S_{i_0}$ is linearly independent and thus $\{v_1, v_2, \dots, v_n\}$ is also linearly independent. 

    Thus, as we can apply Zorn's Lemma, we conclude that $I$ has a maximal element, ie, there is a maximal independent set, and thus a $V$ indeed has a basis.
\end{proof}

\begin{theorem}\label{thm:basesofequalcardinality}
    For every vector space $V$ over a field $\field$, any two bases $\mathcal{B}_1,\mathcal{B}_2$ are of equal size/cardinality, ie, there is a bijection between $\mathcal{B}_1$ and $\mathcal{B}_2$.
\end{theorem}