\section{Inner Product Spaces}

\subsection{Introduction: Inner Products, Norms, Basic Properties}

For this section, $\field$ will always be either $\R$ or $\C$.

\begin{definition}[Inner Product]
    Let $V$ be a vector space over $\field$. An \emph{inner product} on $V$ is a function \begin{align*}
        V \times V  \to \field, \qquad (u, v) \mapsto \langle u, v\rangle,
    \end{align*}
    satisfying, for all $u, v,w \in V$ and $\alpha \in \field$, 
    \begin{enumerate}
        \item Linear in the first coordinate:
        \begin{enumerate}
            \item $\iprod{u + v, w} = \iprod{u,w} + \iprod{v, w}$
            \item $\iprod{\alpha u, v} = \alpha \cdot \iprod{u, v}$
        \end{enumerate}
        \item Skew-symmetric:
        \begin{enumerate}
            \item $\iprod{u, v} = \overline{\iprod{v, u}}$
        \end{enumerate}
        \item $\iprod{u, u} \geq 0$, and equal to $0$ iff $u = 0_V$.
    \end{enumerate}
    $V$ together with $\iprod{.,.}$ is called an \emph{inner product space}.
\end{definition}

Unless otherwise stated, all vector spaces $V$ should be considered as an inner product space from here on.
\begin{remark}
 Note that the third requirement is well-defined; that is, it follows from 2. that $\iprod{u,u} \in \R$, since $\iprod{u, u} =\overline{\iprod{u, u}}$, ie $\iprod{u, u}$ is equal to its own complex conjugate, which is only possible if its imaginary part is precisely 0. So, it makes sense to require it to be geq 0 (if it was complex, this would be meaningless).
\end{remark}

\begin{definition}
    Let $\iprod{.,.}$ be an inner product on $V$. The \emph{norm} associated to this inner product is defined \[
    \norm{v} \defeq \sqrt{\iprod{v,v}}, \quad v \in V.
    \]
    We call $v \in V$ a \emph{unit vector} if $\norm{v} = 1$. For $v \in V, v \neq 0$, we call $\norm{v}^{-1} \cdot v$ the \emph{normalization of $v$}.
\end{definition}

\begin{remark}
    \emph{Never} work with a norm directly; working with the square of the norm is far easier.
\end{remark}

\begin{proposition}
    Let $V$ be an inner product space. For each $u, v, w \in V$ and $\alpha \in \field$,
    \begin{enumerate}
        \item Conjugate linearity in the second coordinate holds:
        \begin{enumerate}
            \item $\iprod{u, v + w} = \iprod{u, v} + \iprod{u, w}$
            \item $\iprod{u, \alpha v} = \overline{\alpha} \iprod{u, v}$
        \end{enumerate}
        \item $\norm{\alpha\cdot  v} = \abs{\alpha} \cdot \norm{v}$
        \item $\iprod{v, 0_V} = 0 = \iprod{0_V, v}$
    \end{enumerate}
\end{proposition}
\begin{proof}
    1.(a), (b) follow from skew-symmetry.

    For 2., we have $\norm{\alpha v}^2 = \iprod{\alpha v, \alpha v} = \alpha \cdot \overline{\alpha} \iprod{v, v} = \abs{\alpha}^2 \cdot \norm{v}^2$.

    For 3., follows from $\iprod{0_V, v} + \iprod{0_V, v} = \iprod{0_V, v}$.
\end{proof}

\begin{example}
    \begin{enumerate}
   \item For $V \defeq \field^n$, the standard inner product is the "dot product"; for $\vec{x} \defeq (x_1, \dots, x_n), \vec{y} \defeq (y_1, \dots, y_n)$, \begin{align*}
            \iprod{\vec{x}, \vec{y}} \defeq \vec{x} \cdot \vec{y} \defeq \sum_{i=1}^n x_i \overline{y_i},
        \end{align*}
        which gives \begin{align*}
            \norm{\vec{x}} = \sqrt{\sum_{i=1}^n \abs{x_i}^2},
        \end{align*}
        that is, the standard Euclidean norm.
        \begin{proposition}
            For $\field \defeq \R$ and $\vec{x}, \vec{y} \in \R^n$, $\vec{x} \cdot \vec{y} = \norm{\vec{x}} \norm{\vec{y}} \cos \alpha$, where $\alpha$ the angle from $\vec{x}$ to $\vec{y}$.
        \end{proposition}
        \item If $\iprod{.,.}$ an inner product on $V$ and $r$ a positive real, then $\iprod{.,.}_r \defeq r \cdot \iprod{.,.}$ is also an inner product.
        \item Let $V \defeq C[0,1]$. Define for $f, g \in V$, \[
        \iprod{f, g}    \defeq \int_0^1 f(t) \cdot \overline{g(t)} \dd{t}.
        \]
        \item Let $V \defeq \field[t]_n$. For $f(t) \defeq a_0 + a_1 t + \cdots + a_n t^n, g(t) \defeq b_0 + b_1 t + \cdots + b_n t^n$, define \begin{align*}
            \iprod{f, g}_1 \defeq \sum_{i=0}^n a_i \overline{b_i},
        \end{align*}
        and \begin{align*}
            \iprod{f, g}_2 \defeq \int_0^1 f(t) \overline{g(t)} \dd{t}.
        \end{align*}
        These are both inner products.
        \item For $A \in M_{n \times m}(\field)$, let $A^\ast \defeq \overline{A}^t$ the \emph{conjugate transpose of} $A$.\footnotemark  For $V \defeq M_n(\field)$ and $A, B \in V$, define \[
        \iprod{A, B} \defeq \tr(B^\ast \cdot A).
        \]
        It is left as a (homework) exercise to verify that this is a well-defined inner product.
    \end{enumerate}
\end{example}

\footnotetext{Where $\overline{A} \defeq (\overline{a_{ij}})$.}

